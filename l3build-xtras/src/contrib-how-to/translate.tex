\documentclass[10pt, a4paper]{../main/main}

\usepackage[utf8]{inputenc}
\usepackage[T1]{fontenc}

\usepackage[french]{babel, varioref}

\usepackage{enumitem}
\frenchsetup{StandardItemLabels=true}

\usepackage{tabularray}

\usepackage[lang = french]{tutodoc}


\usepackage{../deco/deco.cls}
\usepackage{../admonitions/admonitions.cls}
\usepackage{../inenglish/inenglish.cls}
\usepackage{../macroenv/macroenv.cls}

%\geometry{showframe}


% == FORDOC == %

\NewDocumentCommand{\mailsubject}{m}%  <-- Translate me!
  {sujet \tdocquote{\texttt{tutodoc - CONTRIB - #1}}}

% Source: https://tex.stackexchange.com/a/424061/6880

\newcommand{\FTdirO}{}
\def\FTdirO(#1,#2,#3){%
  \FTfile(#1,{\color{blue!40!black}\faFolderOpen\hspace{-.35pt}}{\hspace{0.2em}#3})
  (tmp.west)++(0.8em,-0.4em)node(#2){}
  (tmp.west)++(1.5em,0)
  ++(0,-1.3em)
}

\newcommand{\FTdirC}{}
\def\FTdirC(#1,#2,#3){%
  \FTfile(#1,{\color{blue!40!black}\faFolder\hspace{.75pt}}{\hspace{0.2em}#3})
  (tmp.west)++(0.8em,-0.4em)node(#2){}
  (tmp.west)++(1.5em,0)
  ++(0,-1.3em)
}

\newcommand{\FTfile}{}
\def\FTfile(#1,#2){%
  node(tmp){}
  (#1|-tmp)++(0.6em,0)
  node(tmp)[anchor=west,black]{\tt #2}
  (#1)|-(tmp.west)
  ++(0,-1.2em)
}

\newcommand{\FTroot}{}
\def\FTroot{tmp.west}

\newcommand\contribtranslatedirtree{
  \begin{tikzpicture}%
    \draw[color=black, thick]
% en        : parent = \FTroot
% normal dir: (parentID, currentID, label)
% file      :       (parentID, label)
      \FTdirO(\FTroot,root,translate){
        \FTdirC(root,changes,changes){
        }
        \FTdirO(root,en,en) {
          \FTdirC(en,api,api) {
          }
          \FTdirC(en,doc,doc) {
          }
        }
        \FTdirC(root,fr,fr){
        }
        \FTdirC(root,status,status){
          \FTdirO(status,en,en) {
            \FTfile(en,api.yaml)
            \FTfile(en,manual.yaml)
          }
          \FTdirC(status,fr,fr){
          }
        }
        \FTfile(root,README.md)
        \FTfile(root,LICENSE.txt)
      };
  \end{tikzpicture}
}


\begin{document}

%\section{Contribuer}

\subsection{Compléter les traductions}

\begin{tdocnote}
    L'auteur de \thisproj\ gère les versions françaises et anglaises des traductions.
\end{tdocnote}


\begin{tdoccaut}
    Même si nous allons expliquer comment traduire les documentations, il semble peu pertinent de le faire, car l'anglais devrait suffire de nos jours.%
    \footnote{
        L'existence d'une version française est une simple conséquence de la langue maternelle de l'auteur de \thisproj.
    }
\end{tdoccaut}

\begin{figure}[ht]
	\centering
    \contribtranslatedirtree\
    \caption{Vue simplifiée du dossier des traductions}
    \label{tutodoc-contrib-translate-dir}
\end{figure}


Les traductions sont organisées grosso-modo comme dans la figure \ref{tutodoc-contrib-translate-dir} où seuls les dossiers importants pour les traductions ont été \tdocquote{ouverts}\,.%
\footnote{
    Cette organisation était celle du 5 octobre 2024, mais elle reste d'actualité.
}
\textbf{Un peu plus bas, la section \ref{tutodoc-contrib-translate} donne les démarches à suivre pour ajouter de nouvelles traductions.}


\subsubsection{Les dossiers \texttt{fr} et \texttt{en}}

Ces deux dossiers, gérés par l'auteur de \thisproj, ont la même organisation ; ils contiennent des fichiers faciles à traduire même si l'on n'est pas un codeur.


\subsubsection{Le dossier \texttt{changes}}

Ce dossier est un outil de communication où sont indiqués les changements importants sans s'attarder sur les modifications mineures propres à une ou plusieurs traductions.



\subsubsection{Le dossier \texttt{status}}

Ce dossier permet de savoir où en sont les traductions du point de vue du projet. Tout se passe via des fichiers \verb#YAML# bien commentés, et lisibles par un non codeur.



\subsubsection{Les fichiers \texttt{README.md} et \texttt{LICENSE.txt}}

Le fichier \texttt{LICENSE.txt} est bien nommé, tandis que le fichier \texttt{README.md} reprend en anglais les points importants de ce qui est dit dans cette section sur les nouvelles traductions.



\subsubsection{De nouvelles traductions}
\label{tutodoc-contrib-translate}

\begin{tdocimp}
    Le dossier \verb#api# contient les traductions relatives aux fonctionnalités de \thisproj.
    Vous y trouverez des fichiers de type \verb#TXT# à modifier via un éditeur de texte, ou de code, mais non via un traitement de texte.
    Les contenus de ces fichiers utilisent des lignes commentées en anglais pour expliquer ce que fera \thisproj\ ; ces lignes commencent par \verb#//#\,. Voici un extrait de ce type de fichier où \textbf{les traductions se font après chaque signe \,$=$\, sans toucher à ce qui se trouve avant}, car ce morceau initial est utilisé en interne par le code de \thisproj..

    \tdocsep
    \vspace{-10pt}
    \begin{verbatim}
  // #1: year  in format YYYY like 2023.
  // #2: month in format MM   like 04.
  // #3: day   in format DD   like 29.
  date = #1-#2-#3

  // #1: the idea is to produce one text like
  //     "this word means #1 in English".
  in_EN = #1 in english\end{verbatim}
\end{tdocimp}


\begin{tdocnote}
    Le dossier \verb#doc# est réservé aux documentations. Il contient des fichiers de type \verb#TEX# compilables directement pour une validation en temps réel des traductions faites.
\end{tdocnote}


\begin{tdocwarn}
    Ne partir que de l'un des dossiers \verb#fr# et \verb#en#, car ceux-ci sont de la responsabilité de l'auteur de \thisproj.
\end{tdocwarn}


\medskip


\emph{\textbf{Imaginons que vous souhaitiez ajouter le support de l'italien en partant de fichiers rédigés en anglais.}}%
\footnote{
    Comme indiqué plus haut, il n'y a pas de besoin réel du côté du dossier \texttt{doc} de la documentation.
}


\paragraph{Méthode 1 : \git.}

\begin{enumerate}
    \item Obtenir tout le dossier du projet via \thisrepo\,.
    Ne pas passer via la branche \verb#main# qui sert à figer les dernières versions stables des projets du dépôt unique \thismonorepo\,.

    \item Dans le dossier \verb#tutodoc/contrib/translate#, créer une copie \verb#it# du dossier \verb#en#\,, le nom court de la langue étant documenté dans
    \href{https://en.wikipedia.org/wiki/IETF_language_tag#List_of_common_primary_language_subtags}%
         {la page \tdocquote{IIETF language tag}}
    de \texttt{Wikipédia}.

    \item Une fois la traduction achevée dans le dossier \verb#it# , il faudra la communiquer via \thisrepo\ en usant d'un classique \verb#git push#\,.
\end{enumerate}


\paragraph{Méthode 2 : communiquer par courriels.}

\begin{enumerate}
    \item Via un courriel de \mailsubject{en FOR italian}\,, demander une version des traductions anglaises (noter l'emploi du nom anglais de la nouvelle langue).
    Bien respecter le sujet du courriel, car l'auteur de \thisproj\ automatise le pré-traitement de ce type de courriels.

    \item Vous recevrez un dossier nommé \verb#italian# contenant la version anglaise des dernières traductions.
    Ce dossier sera le lieu de votre contribution.

    \item Une fois la traduction achevée, il faudra compresser votre dossier \verb#italian# au format \verb#zip# ou \verb#rar# avant de l'envoyer par courriel avec le \mailsubject{italian}\,.
\end{enumerate}

\end{document}
