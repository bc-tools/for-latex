\documentclass[12pt, a4paper]{article}

\usepackage[utf8]{inputenc}
\usepackage[T1]{fontenc}

\usepackage[french]{babel, varioref}

\usepackage{enumitem}
\frenchsetup{StandardItemLabels=true}


\usepackage[lang = french]{../main/main}
\usepackage{../listing/listing}
\usepackage{../rem-exa/rem-exa}

% TESTING LOCAL IMPLEMENTATION %

\usepackage{macroenv}


\begin{document}

\section{Indiquer des packages, des macros ou des environnements}

Voici ce qu'il est possible de taper de façon sémantique.

\begin{bdoclatex}[sbs]
\bdocpack{monpackage} est pour...

\bdocmacro{unemacro} permet de...

\bdocenv{env} sert à...
\end{bdoclatex}


\begin{bdocrem}
    L'intérêt des macros précédentes vis à vis de l'usage de \bdocmacro{bdocinlatex} est l'absence de coloration.
    De plus, la macro \bdocmacro{bdocenv} demande juste de taper le nom de l'environnement
    \footnote{
        De plus, \bdocenv{cetenv} utilise des espaces pour autoriser des retours à la ligne si besoin.
    }.
\end{bdocrem}


% -------------------- %


\section{Expliquer les préfixes ou les suffixes}

Pour expliquer les noms retenus rien de tel que d'indiquer et expliciter les courts préfixes et suffixes retenus.

\begin{bdoclatex}[sbs]
\bdocpre{sup} est relatif à...

\bdocprewhy{sup}{erbe} signifie...

\emph{\bdocprewhy{sup}{erbe} vient de...}
\end{bdoclatex}


\end{document}
