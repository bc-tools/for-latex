\documentclass[10pt, a4paper]{article}

\usepackage[utf8]{inputenc}
\usepackage[T1]{fontenc}

\usepackage[french]{babel, varioref}

\usepackage{enumitem}
\frenchsetup{StandardItemLabels=true}

\usepackage{tabularray}

\usepackage[lang = french]{tutodoc}


\usepackage[lang = french]{../main/main}
\usepackage{../listing/listing}
\usepackage{../focus/focus}

% TESTING LOCAL IMPLEMENTATION %

\usepackage{macroenv}


\begin{document}

\section{Indiquer des packages, des macros ou des environnements}

Voici ce qu'il est possible de taper de façon sémantique.

\begin{bdoclatex}[sbs]
\bdocpack{monpackage} est pour...

\bdocmacro{unemacro} permet de...

\bdocenv{env} sert à...

On a aussi :

\bdocenv[{[opt1]<opt2>}]{env}
\end{bdoclatex}


\begin{bdocrem}
    L'intérêt des macros précédentes vis à vis de l'usage de \bdocmacro{bdocinlatex} est l'absence de coloration.
    De plus, la macro \bdocmacro{bdocenv} demande juste de taper le nom de l'environnement
    \footnote{
        De plus, \bdocinlatex{\bdocenv{monenv}} produit \bdocenv{monenv} avec des espaces afin d'autoriser des retours à la ligne si besoin.
    }
    avec des éventuelles options en tapant les bons délimiteurs
    \footnote{
        Se souvenir que tout est possible ou presque dorénavant.
    }
    à la main.
\end{bdocrem}


\begin{bdocwarn}
    L'argument optionnel de la macro \bdocmacro{bdocenv} est copié-collé lors du rendu. Ceci peut donc parfois nécessiter d'utiliser des accolades protectrices comme dans l'exemple précédent. 
\end{bdocwarn}


% -------------------- %


\section{Origine d'un préfixe ou d'un suffixe}

Pour expliquer les noms retenus, rien de tel que d'indiquer et expliciter les courts préfixes et suffixes retenus. Ceci se fait facilement comme suit.

\begin{bdoclatex}[sbs]
\bdocpre{sup} est relatif à...

\bdocprewhy{sup.erbe} signifie...

\emph{\bdocprewhy{sup.er} pour...}
\end{bdoclatex}


\begin{bdocrem}
    Le choix du point pour scinder un mot permet d'utiliser des mots avec un tiret comme dans \bdocinlatex+\bdocprewhy{ca.sse-brique}+ qui donne \bdocprewhy{ca.sse-brique}.
\end{bdocrem}

\end{document}
