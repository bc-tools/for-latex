\documentclass[10pt, a4paper]{article}

\usepackage[utf8]{inputenc}
\usepackage[T1]{fontenc}

\usepackage[french]{babel, varioref}

\usepackage{enumitem}
\frenchsetup{StandardItemLabels=true}

\usepackage{tabularray}

\usepackage[lang = french]{tutodoc}


\usepackage[lang = french]{../main/main}
\usepackage{../macroenv/macroenv}
\usepackage{../inenglish/inenglish}
\usepackage{../showcase/showcase}
\usepackage{../listing/listing}
\usepackage{../focus/focus}

% TESTING LOCAL IMPLEMENTATION %

\usepackage{version-n-change}


\begin{document}

%\section{Indiquer les changements}

\subsection{Quoi de neuf ?}

Pour fournir des explications efficaces dans l'historique des changements, \bdocpack{bdoc} propose différents environnements pour indiquer rapidement et clairement ce qui a été fait
\footnote{
	L'utilisateur n'a pas besoin de tous les détails techniques.
}.


\begin{bdocexa}[Pour les nouveautés]
	\leavevmode
	
    \bdoclatexshow{examples/version-n-change/new.tex}
\end{bdocexa}


% ------------------ %


\begin{bdocexa}[Pour les mises à jour]
	\leavevmode

    \bdoclatexshow{examples/version-n-change/update.tex}
\end{bdocexa}


% ------------------ %


\begin{bdocexa}[Pour les réparations]
	\leavevmode

    \bdoclatexshow{examples/version-n-change/fix.tex}
\end{bdocexa}


% ------------------ %


\begin{bdocexa}[Thématiques aux choix]
	\leavevmode

    \bdoclatexshow{examples/version-n-change/topic.tex}
\end{bdocexa}

\end{document}

