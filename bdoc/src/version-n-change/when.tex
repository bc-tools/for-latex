\documentclass[10pt, a4paper]{article}

\usepackage[utf8]{inputenc}
\usepackage[T1]{fontenc}

\usepackage[french]{babel, varioref}
\frenchsetup{StandardItemLabels=true}

\usepackage{enumitem}

\usepackage{multicol}

\newcommand\thispack{\tdocpack{tutodoc}}


\usepackage[lang = french]{../main/main}
\usepackage{../macroenv/macroenv}
\usepackage{../inenglish/inenglish}
\usepackage{../showcase/showcase}
\usepackage{../listing/listing}
\usepackage{../focus/focus}

% TESTING LOCAL IMPLEMENTATION %

\usepackage{version-n-change}


\begin{document}

\section{Changements}

\subsection{À quel moment ?}


\begin{bdocexa}[Dater des nouveautés]
    Dater une nouveauté se fait via la macro \bdocmacro{bdocdate} comme dans l'exemple suivant où la date doit être donnée au format anglais \texttt{AAAA-MM-JJ}.

    \bdoclatexshow{examples/version-n-change/dating.tex}
\end{bdocexa}


% ------------------ %


\begin{bdocexa}[Versionner et dater des nouveautés]
    Associer un numéro de version à une nouveauté se fait via la macro \bdocmacro{bdocversion}, la couleur et la date étant des arguments optionnels.

    \bdoclatexshow{examples/version-n-change/versioning.tex}
\end{bdocexa}


\begin{bdocimportant}
    \leavevmode

    \begin{enumerate}
        \item Les macros \bdocmacro{bdocdate} et \bdocmacro{bdocversion} nécessitent deux compilations.

        \item Comme la langue indiquée pour cette documentation est le français, la date dans le rendu final est au format \texttt{JJ/MM/AAAA} alors que dans le code celle-ci devra toujours être donnée au format anglais \texttt{AAAA-MM-JJ}.
    \end{enumerate}
\end{bdocimportant}


\begin{bdocwarn}
    Seul l'emploi du format numérique \bdocinlatex+YYYY-MM-DD+ est vérifié
    \footnote{
        Techniquement, vérifier la validité d'une date, via \LaTeX3, ne présente pas de difficulté.
    },
    et ceci est un choix ! Pourquoi cela ? Tout simplement car dater et versionner des explications doit se faire de façon semi-automatisée afin d'éviter tout bug humain
    \footnote{
        L'auteur de \bdocpack{bdoc} est entrain de mettre en place un ensemble d'outils permettant une telle semi-automatisation.
    }.
\end{bdocwarn}

\end{document}
