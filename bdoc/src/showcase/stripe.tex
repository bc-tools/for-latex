\documentclass[10pt, a4paper]{article}

\usepackage[utf8]{inputenc}
\usepackage[T1]{fontenc}

\usepackage[french]{babel, varioref}
\frenchsetup{StandardItemLabels=true}

\usepackage{enumitem}

\usepackage{multicol}

\newcommand\thispack{\tdocpack{tutodoc}}


\usepackage[lang = french]{../main/main}
\usepackage{../macroenv/macroenv}
\usepackage{../listing/listing}
\usepackage{../focus/focus}

% TESTING LOCAL IMPLEMENTATION %

\usepackage{showcase}


\begin{document}

\section{Un rendu en situation réelle} \label{bdoc-showcase}

\subsection{Avec une bande colorée}

\begin{bdocexa}[Avec les textes par défaut]
    Il peut être utile de montrer un rendu réel directement dans un document, typiquement lorsque l'on fait une démo.
    Ceci se tape comme suit.

    \bdoclatexinput[code]{examples/showcase/default.tex}

    On obtient alors le rendu suivant
    \footnote{
        La bande est créée sans effort grâce au package \bdocpack{clrstrip}.
    }.

    \medskip

    
\begin{bdocshowcase}
    \bfseries Un peu de code \LaTeX.

    \bigskip

    \emph{\large Fin de l'affreuse démo.}
\end{bdocshowcase}
\end{bdocexa}


\begin{bdocrem}
    Voir la section \ref{bdoc-latexshow} page \pageref{bdoc-latexshow} pour obtenir facilement un code suivi de son rendu réel comme dans l'exemple précédent.
\end{bdocrem}

\begin{bdocnote}
    Les textes explicatifs s'adaptent à la langue choisie lors du chargement de \bdocpack{bdoc}.
\end{bdocnote}


% ------------------ %


\begin{bdocexa}[Changer la couleur et/ou les textes par défaut]
    \leavevmode

    \bdoclatexinput[code]{examples/showcase/customized.tex}

    Ceci produira ce qui suit.

    \medskip

    \begin{tdocshowcase}[before     = Mon début,
                     after      = Ma fin à moi,
                     col-stripe = red,
                     col-text   = orange!75!black]
    Bla, bla, bla, bla, bla, bla, bla, bla, bla, bla, bla, bla, bla...
\end{tdocshowcase}

\end{bdocexa}


\begin{bdocnote}
    Vous avez sûrement noté que l'on n'obtient pas un rouge pur : en coulisse les macros développables \bdocmacro{bdocbackcolor} et \bdocmacro{bdocdarkcolor} sont utilisées pour créer celles du fond et des titres respectivement à partir de la couleur proposée à \bdocenv{bdocshowcase}.
	Ces macros à un seul argument, la couleur choisie, admettent les codes suivants.

	\begin{bdoclatex}[code]
\NewExpandableDocumentCommand{\bdocbackcolor}{m}{#1!5}
\NewExpandableDocumentCommand{\bdocdarkcolor}{m}{#1!50!black}
	\end{bdoclatex}
\end{bdocnote}


% ------------------ %


\begin{bdocwarn}
    Avec la mise en forme par défaut, si le code \LaTeX\ commence par un crochet ouvrant, il faudra indiquer explicitement une option vide comme dans l'exemple suivant.

    \bdoclatexinput[code]{examples/showcase/hook.tex}

    Ceci produira ce qui suit.

    \medskip

    \begin{tdocshowcase}[]
    [Cela fonctionne...]
\end{tdocshowcase}

OU.

\begin{tdocshowcase}
    \string[Cela fonctionne aussi...]
\end{tdocshowcase}

\end{bdocwarn}


\begin{bdocnote}
    Il faut savoir qu'en coulisse la macro \bdocmacro{bdocruler} est utilisée.

    \begin{bdoclatex}[std]
        \bdocruler{Un pseudo-titre décoré}{red}
    \end{bdoclatex}
\end{bdocnote}


\end{document}
