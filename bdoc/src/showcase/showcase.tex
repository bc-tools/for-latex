\documentclass[12pt,a4paper]{article}

\usepackage[utf8]{inputenc}
\usepackage[T1]{fontenc}
\usepackage[french]{babel, varioref}

% USING OTHER TOOLS %

\usepackage[lang = FR]{../main/main}
\usepackage{../macroenv/macroenv}
\usepackage{../listing/listing}
\usepackage{../rem-exa/rem-exa}

% TESTING LOCAL IMPLEMENTATION %

\usepackage{showcase}


\begin{document}

\section{Un rendu en situation réelle}

\docexa[Avec les textes par défaut]

Il peut être utile de montrer un rendu réel directement dans un document. Ceci se tape comme suit.

\begin{doclatex-alone}
\begin{docshowcase}
    \bfseries Un peu de code \LaTeX.

    \bigskip

    \itshape Fin de la démo.
\end{docshowcase}
\end{doclatex-alone}

On obtient alors le rendu suivant
\footnote{
    La bande est créée sans effort grâce au package \docpack{clrstrip}.
}.

\medskip

\begin{docshowcase}
    \bfseries Un peu de code \LaTeX.

    \bigskip

    \itshape Fin de la démo.
\end{docshowcase}


\bigskip


Il faut savoir qu'en coulisse la macro \docmacro{docruler} est utilisée.

\begin{doclatex-flat}
\docruler{Un pseudo-titre décoré}
\end{doclatex-flat}


\begin{docrem}
    Les textes utilisés pour délimiter le rendu s'adaptent à la langue choisie lors du chargement du package \docpack{bdoc}.
\end{docrem}


% ------------------ %

\docexa[Changer les textes par défaut]

C'est assez simple à faire.

\begin{doclatex-alone}
\begin{docshowcase}[Mon début]%
                   [Ma fin]
    Bla, bla, bla, bla, bla, bla, bla, bla, bla, bla, bla, bla, bla...
\end{docshowcase}
\end{doclatex-alone}

On obtient alors ce qui suit.

\medskip

\begin{docshowcase}[Mon début]%
                   [Ma fin]
    Bla, bla, bla, bla, bla, bla, bla, bla, bla, bla, bla, bla, bla...
\end{docshowcase}

\end{document}
