\documentclass[12pt,a4paper]{article}

\usepackage[utf8]{inputenc}
\usepackage[T1]{fontenc}
\usepackage[french]{babel, varioref}

% USING OTHER TOOLS %

\usepackage[lang = FR]{../main/main}
\usepackage{../macroenv/macroenv}
\usepackage{../listing/listing}
\usepackage{../rem-exa/rem-exa}

% TESTING LOCAL IMPLEMENTATION %

\usepackage{showcase}


\begin{document}

\section{Un rendu en situation réelle}

\bdocexa[Avec les textes par défaut]

Il peut être utile de montrer un rendu réel directement dans un document. Ceci se tape comme suit.

\begin{bdoclatex-alone}
\begin{bdocshowcase}
    \bfseries Un peu de code \LaTeX.

    \bigskip

    \itshape Fin de la démo.
\end{bdocshowcase}
\end{bdoclatex-alone}

On obtient alors le rendu suivant
\footnote{
    La bande est créée sans effort grâce au package \bdocpack{clrstrip}.
}.

\medskip

\begin{bdocshowcase}
    \bfseries Un peu de code \LaTeX.

    \bigskip

    \itshape Fin de la démo.
\end{bdocshowcase}


\bigskip


Il faut savoir qu'en coulisse la macro \bdocmacro{bdocruler} est utilisée.

\begin{bdoclatex-flat}
\bdocruler{Un pseudo-titre décoré}
\end{bdoclatex-flat}


\begin{bdocrem}
    Les textes utilisés pour délimiter le rendu s'adaptent à la langue choisie lors du chargement du package \bdocpack{bdoc}.
\end{bdocrem}


% ------------------ %

\bdocexa[Changer les textes par défaut]

C'est assez simple à faire.

\begin{bdoclatex-alone}
\begin{bdocshowcase}[Mon début]%
                    [Ma fin]
    Bla, bla, bla, bla, bla, bla, bla, bla, bla, bla, bla, bla, bla...
\end{bdocshowcase}
\end{bdoclatex-alone}

On obtient alors ce qui suit.

\medskip

\begin{bdocshowcase}[Mon début]%
                   [Ma fin]
    Bla, bla, bla, bla, bla, bla, bla, bla, bla, bla, bla, bla, bla...
\end{bdocshowcase}

\end{document}
