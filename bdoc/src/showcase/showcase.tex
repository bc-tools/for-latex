\documentclass[12pt, a4paper]{article}

\usepackage[utf8]{inputenc}
\usepackage[T1]{fontenc}

\usepackage[french]{babel, varioref}

\usepackage{enumitem}
\frenchsetup{StandardItemLabels=true}


\usepackage[lang = french]{../main/main}
\usepackage{../macroenv/macroenv}
\usepackage{../listing/listing}
\usepackage{../rem-exa/rem-exa}

% TESTING LOCAL IMPLEMENTATION %

\usepackage{showcase}


\begin{document}

\section{Un rendu en situation réelle}

\begin{bdocexa}[Avec les textes par défaut] 
    Il peut être utile de montrer un rendu réel directement dans un document.
    Ceci se tape comme suit.
    
    \bdoclatexinput[code]{examples/showcase/default.tex}
    
	On obtient alors le rendu suivant
	\footnote{
    	La bande est créée sans effort grâce au package \bdocpack{clrstrip}.
	}.

	\medskip
	
	\begin{bdocshowcase}
    \bfseries Un peu de code \LaTeX.

    \bigskip

    \emph{\large Fin de l'affreuse démo.}
\end{bdocshowcase}
\end{bdocexa}


\begin{bdocrem}
	\leavevmode
	
	\begin{itemize}
	    \item Voir la section \ref{bdoc-latexshow} pour obtenir facilement un code suivi de son rendu réel comme dans l'exemple précédent.
	    
	    \item Il faut savoir qu'en coulisse la macro \bdocmacro{bdocruler} est utilisée.
	\end{itemize}

    \begin{bdoclatex}[std]
	    \bdocruler{Un pseudo-titre décoré}
	\end{bdoclatex}
\end{bdocrem}


% ------------------ %


\begin{bdocexa}[Changer les textes par défaut]
	\leavevmode

	\bdoclatexinput[code]{examples/showcase/text-changed.tex}

	On obtient alors ce qui suit.

	\medskip
	
	\begin{bdocshowcase}[Mon début]%
                    [Ma fin à moi]
    Bla, bla, bla, bla, bla, bla, bla, bla, bla, bla, bla, bla, bla...
\end{bdocshowcase}
\end{bdocexa}


\begin{bdocrem}
    Les textes utilisés par défaut  pour délimiter le rendu s'adaptent à la langue choisie lors du chargement du package \bdocpack{bdoc}.
\end{bdocrem}

\end{document}
