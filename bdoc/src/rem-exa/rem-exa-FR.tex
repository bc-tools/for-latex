\documentclass[12pt,a4paper]{article}

\usepackage[utf8]{inputenc}
\usepackage[french]{babel, varioref}

% USING OTHER TOOLS %

\usepackage[lang = FR]{../main/main}
\usepackage{../macroenv/macroenv}
\usepackage{../showcase/showcase}
\usepackage{../listing/listing}
\usepackage{../rem-exa/rem-exa}
\usepackage{../inenglish/inenglish}

% TESTING LOCAL IMPLEMENTATION %

\usepackage{rem-exa}


\begin{document}

\section{Faire une remarque}

Tout se passe via l'environnement \docenv{docrem} comme dans l'ex\-emple suivant.

\begin{doclatex}
    \begin{docrem}
        Juste une remarque...
    \end{docrem}
\end{doclatex}


% ------------------ %
% ------------------ %
% ------------------ %
% ------------------ %


\section{Exemples}

Comme les documentations se basent sur des cas d'utilisation, la macro \docmacro{docexa} est indispensable pour organiser visuellement les exemples.

\inputdoclatexreal{examples/rem-exa/exa.tex}


\begin{docrem}
    \leavevmode

    \begin{enumerate}
        \item La numérotation des exemples est remise à zéro dès qu'une section  de niveau au moins égale à une \docilatex|\subsubsection| est ouverte.

        \item Le mot \docquote{Exemple} sera traduit dans la langue indiquée lors du chargement du package \docpack{bdoc}.
    \end{enumerate}
\end{docrem}

\end{document}
