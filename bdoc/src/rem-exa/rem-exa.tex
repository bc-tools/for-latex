\documentclass[12pt, a4paper]{article}

\usepackage[utf8]{inputenc}
\usepackage[T1]{fontenc}

\usepackage[french]{babel, varioref}

\usepackage{enumitem}
\frenchsetup{StandardItemLabels=true}


\usepackage[lang = fr]{../main/main}
\usepackage{../macroenv/macroenv}
\usepackage{../showcase/showcase}
\usepackage{../listing/listing}
\usepackage{../rem-exa/rem-exa}
\usepackage{../inenglish/inenglish}

\usepackage{rem-exa}


\begin{document}

\section{Faire une remarque}

Tout se passe via l'environnement \bdocenv{bdocrem} comme dans l'ex\-emple suivant
\footnote{
	La mise en forme provient du package \bdocpack{amsthm}.
}.

\begin{bdoclatex}
    \begin{bdocrem}
        Juste une remarque...
    \end{bdocrem}
    
    \begin{bdocrem}[Mini titre]
        Utile ?
    \end{bdocrem}
\end{bdoclatex}


Il peut parfois être utile de revenir à la ligne dès le début du contenu. Voici comment faire (cette astuce reste valable pour les environnements présentés dans la section suivante).

\begin{bdoclatex}
    \begin{bdocrem}
        \leavevmode 
	 	  
        \begin{enumerate}
            \item Point 1.

            \item Point 2.
        \end{enumerate}
    \end{bdocrem}
\end{bdoclatex}


\begin{bdocrem}
	Le mot \bdocquote{Remarque} sera traduit dans la langue indiquée lors du chargement du package \bdocpack{bdoc}.
\end{bdocrem}


% ------------------ %
% ------------------ %
% ------------------ %
% ------------------ %


\section{Exemples}

L'environnement \bdocenv{docexa} et sa version étoilée permettent de proposer des exemples. Voici les différents emplois possibles.

\inputbdoclatexreal{examples/rem-exa/exa.tex}


\begin{bdocrem}
	\leavevmode 
	 	  
    \begin{enumerate}
        \item La numérotation des exemples est remise à zéro dès qu'une section  de niveau au moins égale à une \bdocinlatex|\subsubsection| est ouverte.

        \item Le mot \bdocquote{Exemple} sera traduit dans la langue indiquée lors du chargement du package \bdocpack{bdoc}.
    \end{enumerate}
\end{bdocrem}

\end{document}
