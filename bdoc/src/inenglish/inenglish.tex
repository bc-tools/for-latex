\documentclass[10pt, a4paper]{article}

\usepackage[utf8]{inputenc}
\usepackage[T1]{fontenc}

\usepackage[french]{babel, varioref}

\usepackage{enumitem}
\frenchsetup{StandardItemLabels=true}

\usepackage{tabularray}

\usepackage[lang = french]{tutodoc}


\usepackage[lang = french]{../main/main}
\usepackage{../macroenv/macroenv}
\usepackage{../listing/listing}
\usepackage{../focus/focus}

% TESTING LOCAL IMPLEMENTATION %

\usepackage{inenglish}


\begin{document}

\section{Cela veut dire quoi en \bdocquote{angliche}}

Penser aux non-anglophones est bien, même si ces derniers se font de plus en plus rares.

\begin{bdoclatex}
Cool et top signifient \bdocinEN*{cool} et \bdocinEN{top}.
\end{bdoclatex}


La macro \bdocmacro{bdocinEN} et sa version étoilée s'appuient sur \bdocmacro{bdocquote} : par exemple, \bdocquote{sémantique} s'obtient via \bdocinlatex|\bdocquote{sémantique}| .

\end{document}
