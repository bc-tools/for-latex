\documentclass[12pt, a4paper]{article}

\usepackage[utf8]{inputenc}
\usepackage[T1]{fontenc}

\usepackage[french]{babel, varioref}

\usepackage{enumitem}
\frenchsetup{StandardItemLabels=true}


\usepackage[lang = french]{../main/main}
\usepackage{../macroenv/macroenv}
\usepackage{../listing/listing}
\usepackage{../rem-exa/rem-exa}

% TESTING LOCAL IMPLEMENTATION %

\usepackage{inenglish}


\begin{document}

\section{Cela veut dire quoi en \bdocquote{angliche}}

Penser aux non-anglophones est bien, même si ces derniers se font de plus en plus rares.

\begin{bdoclatex-flat}
Cool et top signifient \bdocinEN*{cool} et \bdocinEN{top}.
\end{bdoclatex-flat}


La macro \bdocmacro{bdocinEN} et sa version étoilée s'appuient sur \bdocmacro{bdocquote} : par exemple, le code \bdocinlatex|\bdocquote{sémantique}| produit \bdocquote{sémantique}.


\begin{bdocrem}
    Les guillemets sont accessibles via les macros \bdocinlatex|\og| et \bdocinlatex|\fg| .
\end{bdocrem}


\end{document}
