\begin{bdoctopic}{Différentes langues}
    choix de la langue prise en compte par \bdocpack{bdoc}
    (seulement le français et l'anglais sont proposés pour le moment).
\end{bdoctopic}


% -------------------- %


\begin{bdoctopic}{Codes \LaTeX\ et leur rendus}
    \bdocmacro{inputbdoclatexafter},
    \bdocmacro{inputbdoclatexbefore}
    et
    \bdocmacro{inputbdoclatexreal}
    ont un argument optionnel pour changer le texte explicatif.
\end{bdoctopic}


% -------------------- %


\begin{bdoctopic}{Rendus réels}
    dorénavant la bande colorée contiendra aussi les lignes démarcatrices.
\end{bdoctopic}


% -------------------- %


\begin{bdoctopic}{Indiquer des changements}[t]
    \begin{itemize}
        \item \bdocenv{bdoctopic} : 
              l'option \texttt{ml} a été renommée \texttt{t}.

        \item \bdocmacro{bdocversion} :
              la date respecte le format de la langue gérée par \bdocpack{bdoc}.
    \end{itemize}
\end{bdoctopic}


% -------------------- %


\begin{bdoctopic}{Structurer}[t]
    \begin{itemize}
        \item \bdocmacro{signprewhy} :
              suppression de cette macro
              \footnote{
                  La fonctionnalité que proposait cette macro sera disponible via le package \bdocpack{macroenvsign}.
              }.

        \item \bdocmacro{bdocexa} et \bdocmacro{bdocexa*} :
              un éventuel titre s'indique via un argument optionnel maintenant.
              Il n'est donc plus besoin de toujours mettre des accolades.

        \item \bdocenv{bdocremark}
              est à utiliser à la place de l'ancien
              \bdocenv{remark}.

        \item \bdocmacro{bdocpack} :
              nouvelle macro pour indiquer un package.

        \item \bdocmacro{bdocquote} :
              nouvelle macro mettant entre des guillemets.
    \end{itemize}
\end{bdoctopic}

