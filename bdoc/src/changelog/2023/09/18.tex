\begin{bdoctopic}[Contenus mis en avant]
    la sémantique s'affine grâce à l'ajout de deux environnements.
    \bdocenv{bdocwarning} permet d'avertir,
    et
    \bdocenv{bdocinfo} sert à donner des informations utiles à connaître.
\end{bdoctopic}


% -------------------- %


\begin{bdoctopic}<Journal des changements>
    \begin{itemize}
        \item La façon d'utiliser \bdocenv{bdoctopic} a été simplifiée. Suivant le type de délimiteurs employés pour encadrer le titre, on obtient un retour à la ligne ou non.

        \item La macro \bdocmacro{bdocnew} a été renommée \bdocmacro{bdocdate}.
    \end{itemize}
\end{bdoctopic}


% -------------------- %


\begin{bdoctopic}[Rendu seul d'un code \LaTeX]
    le nouvel environnement \bdocenv{bdocshowcase*} n'imprime pas une bande colorée en arrière-plan.
\end{bdoctopic}


% -------------------- %


\begin{bdoctopic}<Codes \LaTeX\ et leur rendu>
    \begin{itemize}
        \item Il suffit maintenant d'utiliser au choix \bdocenv{bdoclatex} ou \bdocmacro{bdoclatexinput} pour afficher du code avec éventuellement son rendu ; le type de mise en forme s'indique via une simple option.

        \item Le rendu de l'environnement \bdocenv{bdoclatexshow} se personnalise via des options avec des délimiteurs spéciaux.

        \item Le nouvel environnement \bdocenv{bdoclatexshow*} n'imprime pas une bande colorée en arrière-plan pour le rendu réel.
    \end{itemize}
\end{bdoctopic}


% -------------------- %


\begin{bdoctopic}<Modifications de la macro \bdocmacro{bdocprewhy}>
    \begin{itemize}
        \item La syntaxe \bdocinlatex+\bdocprewhy{pre.fix}+ avec un point séparateur remplace l'obsolète \bdocinlatex+\bdocprewhy{pre}{fix}+.

        \item La mise en forme s'appuie dorénavant sur \bdocmacro{textperiodcentered}.
    \end{itemize}
\end{bdoctopic}


% -------------------- %


\begin{bdoctopic}<Paramètres locaux>
    \begin{itemize}
        \item Il faut employer les conventions \bdocquote{longues} du package \bdocpack{babel}.
        Par exemple, \bdocinlatex+lang = french+ remplace l'ancien \bdocinlatex+lang = FR+.

        \item Si le package \bdocpack{babel} est employé avec la langue \bdocinlatex{french} comme langue principale, l'espacement autour des doubles-points est celui proposé par \bdocpack{babel}.

        \item Les guillemets sont gérés via le package \bdocpack{csquotes} pour un résultat \bdocquote{professionnel}.
    \end{itemize}
\end{bdoctopic}
