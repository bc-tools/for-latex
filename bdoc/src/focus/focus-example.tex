\documentclass[12pt, a4paper]{article}

\usepackage[utf8]{inputenc}
\usepackage[T1]{fontenc}

\usepackage[french]{babel, varioref}
\frenchsetup{StandardItemLabels=true}

\usepackage{enumitem}

\usepackage{multicol}

\newcommand\thispack{\tdocpack{tutodoc}}


\usepackage[lang = french]{../main/main}
\usepackage{../macroenv/macroenv}
\usepackage{../showcase/showcase}
\usepackage{../listing/listing}
\usepackage{../inenglish/inenglish}

% TESTING LOCAL IMPLEMENTATION %

\usepackage{focus}


\begin{document}

\section{Mettre en avant du contenu}

\begin{bdocnote}
    Les environnements présentés dans cette section
    \footnote{
        La mise en forme provient du package \bdocpack{amsthm}.
    }
    ajoutent un court titre indiquant le type d'informations fournies.
    Ce court texte sera toujours traduit dans la langue indiquée lors du chargement du package \bdocpack{bdoc}.
\end{bdocnote}


% ------------------ %


\subsection{Des exemples}

Des exemples numérotés, ou non, s'indiquent via l'environnement \bdocenv{bdocexa} qui propose deux arguments optionnels.

\begin{enumerate}
    \item Le 1\ier{} argument entre chevrons \verb#<...># peut prendre au choix les valeurs \verb#nb# pour numéroter, valeur par défaut, \verb#nonb# pour ne pas numéroter.

    \item Le 2\ieme{} argument entre crochets \verb#[...]# sert à ajouter un mini-titre.
\end{enumerate}


Voici différents emplois possibles.

\bdoclatexinput[sbs]{examples/focus/exa.tex}


% ------------------ %


\begin{bdocimportant}
    La numérotation des exemples est remise à zéro dès qu'une section  de niveau au moins égale à une \bdocinlatex|\subsubsection| est ouverte.
\end{bdocimportant}


% ------------------ %


\begin{bdoctip}
    Il peut parfois être utile de revenir à la ligne dès le début du contenu. Voici comment faire (ce tour de passe-passe reste valable pour les environnements présentés dans les sous-sections suivantes). Noter au passage que la numérotation suit celle de l'exemple précédent comme souhaité.

    \bdoclatexinput[sbs]{examples/focus/exa-leavevmode.tex}
\end{bdoctip}

\end{document}
