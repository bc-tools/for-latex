\documentclass[12pt, a4paper]{article}

\usepackage[utf8]{inputenc}
\usepackage[T1]{fontenc}

\usepackage[french]{babel, varioref}

\usepackage{enumitem}
\frenchsetup{StandardItemLabels=true}


\usepackage[lang = french]{../main/main}
\usepackage{../macroenv/macroenv}
\usepackage{../showcase/showcase}
\usepackage{../listing/listing}
\usepackage{../inenglish/inenglish}

% TESTING LOCAL IMPLEMENTATION %

\usepackage{focus}


\begin{document}

\section{Mettre en avant du contenu}

\begin{bdocnote}
	Les environnements présentés dans cette section
	\footnote{
		La mise en forme provient du package \bdocpack{amsthm}.
	}
	ajoutent un court titre indiquant le type d'informations fournies.
	Ce court texte sera toujours traduit dans la langue indiquée lors du chargement du package \bdocpack{bdoc}.
\end{bdocnote}	


% ------------------ %


\subsection{Des exemples}

Des exemples numérotés, ou non, s'indiquent via les environnements \bdocenv{bdocexa} et  \bdocenv{bdocexa*} respectivement. Voici différents emplois possibles.

\bdoclatexinput[sbs]{examples/focus/exa.tex}


% ------------------ %


Il peut parfois être utile de revenir à la ligne dès le début du contenu. Voici comment faire (cette astuce reste valable pour les environnements présentés dans les sous-sections suivantes). Noter au passage que la numérotation suit celle de l'exemple précédent comme souhaité.

\bdoclatexinput[sbs]{examples/focus/exa-leavevmode.tex}


% ------------------ %


\begin{bdocnote}
	La numérotation des exemples est remise à zéro dès qu'une section  de niveau au moins égale à une \bdocinlatex|\subsubsection| est ouverte.
\end{bdocnote}

\end{document}
