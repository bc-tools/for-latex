\input{../doc-header.txt}

\usepackage[lang = french]{../main/main}
\usepackage{../macroenv/macroenv}
\usepackage{../inenglish/inenglish}
\usepackage{../showcase/showcase}
\usepackage{../rem-exa/rem-exa}

\usepackage{listing}

% WARNING! 
% 
% We don't need to include the lines below because this is done before
% by the file ``listing-direct.tex``.

\newtcblisting{bdoc-doc-latex-alone}{%
    minted language = latex,
    breakable,
% Code and output
    colback = yellow!5,
% Frame
    colframe = darkgray,
    shadow   = {.75mm}{-.75mm}{0mm}{black!30},
%        sharp corners, % OLD STYLE !
    arc    = .75mm,
    left   = 1mm, right = 1mm,
    bottom = 1mm, top   = 1mm,
% Separating line
    enhanced,
    segmentation style = {
        gray,
        dash pattern = {on 5pt off 2.5pt},
        line width   = 1.25pt
    },
    listing only
}


% == NEEDED == %

% Source.
%    * https://tex.stackexchange.com/a/604698/6880

\NewDocumentCommand{ \bdocdocextraruler }{ m }{%
    \par
    {
        \centering
        \color{green!50!black}%
        \leavevmode
        \kern.075\linewidth
        \leaders\hrule height3.25pt\hfill\kern0pt
        \footnotesize\itshape\bfseries\space\ignorespaces#1\unskip\space
        \leaders\hrule height3.25pt\hfill\kern0pt
        \kern.075\linewidth
        \par
    }
}

\NewDocumentEnvironment{ bdoc-doc-showcase }
                       { O{ Début du rendu dans cette doc. }
                         O{ Fin du rendu dans cette doc. } }{
    \begin{colorstrip}{green!5}
        \bdocdocextraruler{#1}
        \smallskip
}{
        \smallskip
        \bdocdocextraruler{#2}
    \end{colorstrip}
}


\begin{document}

%\section{Cas d'utilisation en \LaTeX}

\subsection{Codes importés}

Pour les codes suivants, on considère un fichier \verb+examples/listing/xyz.tex+ dont le chemin est donné relativement au document présent.
Le contenu de ce fichier se réduit à l'unique ligne \bdocinlatex|$x y z = 1$| .

\medskip

Noter ci-après que la macro \bdocmacro{bdoclatexinput} s'utilise de façon semblable à l'environnement \bdocenv{bdoclatex} excepté que l'on fournit le chemin d'un fichier.


% ------------------ %


\begin{bdocexa}[Face à face]
	\leavevmode

	\begin{bdoc-doc-latex-alone}
\bdoclatexinput[sbs]{examples/listing/xyz.tex}
	\end{bdoc-doc-latex-alone}

	Ceci produit la mise en forme suivante.

	\bdoclatexinput[sbs]{examples/listing/xyz.tex}
\end{bdocexa}


% ------------------ %


\begin{bdocexa}[À la suite]
	\leavevmode

	\begin{bdoc-doc-latex-alone}
\bdoclatexinput{examples/listing/xyz.tex}
	\end{bdoc-doc-latex-alone}

	Ceci produit la mise en forme suivante où l'option employée par défaut est \bdocinlatex#std#.

	\bdoclatexinput{examples/listing/xyz.tex}
\end{bdocexa}


% ------------------ %


\begin{bdocexa}[Juste le code]
	\leavevmode

	\begin{bdoc-doc-latex-alone}
\bdoclatexinput[code]{examples/listing/xyz.tex}
	\end{bdoc-doc-latex-alone}

	Ceci produit la mise en forme suivante.

	\bdoclatexinput[code]{examples/listing/xyz.tex}
\end{bdocexa}


% ------------------ %


\subsection{Codes importés mis en situation}

\begin{bdocexa}[Code suivi du rendu centré]
	Le rendu suivant est obtenu en utilisant \bdocinlatex+\bdoclatexshow{examples/listing/xyz.tex}+ où une option est employée par défaut, à savoir \bdocinlatex+co+ pour \bdocquote{\bdocprewhy{c}{ode} and \bdocprewhy{o}{utput}} soit \bdocinEN{code et production}.

	\medskip

	\begin{bdocshowcase}
    	\bdoclatexshow{examples/listing/xyz.tex}
	\end{bdocshowcase}

	\bigskip

	Il est possible de changer le texte entre le code et son rendu via un argument optionnel entre chevrons.
	Ainsi \bdocinlatex|\bdoclatexshow<Voici ce que cela donne.>{...}| aboutit au résultat suivant.

	\medskip

	\begin{bdocshowcase}<Début du rendu réel \bdocquote{personnalisé}>%
    	                <Fin du rendu réel \bdocquote{personnalisé}>
    	\bdoclatexshow<Voici ce que cela donne.>{examples/listing/xyz.tex}
	\end{bdocshowcase}
\end{bdocexa}


% ------------------ %


\begin{bdocexa}[Rendu centré suivi du code] \label{output-centered}
 	Le rendu suivant
	\footnote{
    	Il faut savoir que le 1\ier{} espace vertical disgracieux vient de l'emploi de \bdocenv{center} en coulisse.
	},
	similaire au précédent, est obtenu via \bdocinlatex+\bdoclatexshow[oc]{examples/listing/xyz.tex}+.

	\medskip

	\begin{bdocshowcase}
    	\bdoclatexshow[oc]{examples/listing/xyz.tex}
	\end{bdocshowcase}

	\bigskip

	Via \bdocinlatex|\bdoclatexshow[oc]<Cette formule se tape comme suit.>{...}| , on obtient le résultat ci-après.

	\medskip

	\begin{bdocshowcase}<Début du rendu réel \bdocquote{personnalisé}>%
        	            <Fin du rendu réel \bdocquote{personnalisé}>
    	\bdoclatexshow[oc]<Cette formule se tape comme suit.>{examples/listing/xyz.tex}
	\end{bdocshowcase}
\end{bdocexa}


% ------------------ %


\begin{bdocexa}[Code importé et son rendu réel]
	Pour un code et son rendu réel non centré, on utilisera \bdocinlatex+\bdoclatexshow[real]{examples/listing/xyz.tex}+ qui délimite mieux ce que produit le code (c'est très utile pour un document qui ne sera pas imprimé).

	\medskip

	\begin{bdoc-doc-showcase}
    	\bdoclatexshow[real]{examples/listing/xyz.tex}
	\end{bdoc-doc-showcase}

	\bigskip

	Là aussi le texte par défaut peut être \bdocquote{personnalisé}. Par exemple, ce qui suit a été obtenu via \bdocinlatex|\bdoclatexshow[real]<On obtient le rendu réel ci-après.>{...}| .

	\medskip

	\begin{bdoc-doc-showcase}[Début du rendu \bdocquote{personnalisé} dans cette doc.]%
    	                     [Fin du rendu \bdocquote{personnalisé} dans cette doc.]
    	\bdoclatexshow[real]<On obtient le rendu réel ci-après.>{examples/listing/xyz.tex}
	\end{bdoc-doc-showcase}
\end{bdocexa}
%
%
%\begin{bdocrem}
%    Toutes les macros imprimant automatiquement du texte tiennent compte de la langue choisie lors du chargement du package \bdocpack{bdoc}.
%\end{bdocrem}

\end{document}
