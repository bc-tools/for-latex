\input{../doc-header.txt}

\usepackage[lang = french]{../main/main}
\usepackage{../macroenv/macroenv}
\usepackage{../inenglish/inenglish}
\usepackage{../showcase/showcase}
\usepackage{../rem-exa/rem-exa}

\usepackage{listing}


% == NEEDED == %

\newtcblisting{bdoc-doc-latex-alone}{%
    minted language = latex,
    breakable,
% Code and output
    colback = yellow!5,
% Frame
    colframe = darkgray,
    shadow   = {.75mm}{-.75mm}{0mm}{black!30},
%        sharp corners, % OLD STYLE !
    arc    = .75mm,
    left   = 1mm, right = 1mm,
    bottom = 1mm, top   = 1mm,
% Separating line
    enhanced,
    segmentation style = {
        gray,
        dash pattern = {on 5pt off 2.5pt},
        line width   = 1.25pt
    },
    listing only
}


\begin{document}

\section{Cas d'utilisation en \LaTeX}

\subsection{Codes \bdocquote{en ligne}}

La macro \bdocmacro{bdocinlatex} permet de taper du code en ligne via un usage similaire à \bdocmacro{verb}.
Voici deux exemples d'utilisation.

\begin{enumerate}
    \item \bdocinlatex+\bdocinlatex|$a^b = c$|+ produit \bdocinlatex|$a^b = c$| .

    \item \bdocinlatex+\bdocinlatex|$a^b = c$|+ vient de \bdocinlatex#\bdocinlatex+\bdocinlatex|$a^b = c$|+# obtenu via \bdocinlatex@\bdocinlatex#\bdocinlatex+\bdocinlatex|$a^b = c$|+#@ ... etc.
\end{enumerate}


\begin{bdocrem}
    Le nom de la macro \bdocmacro{bdocinlatex} vient de \bdocquote{\bdocprewhy{in}{line} \LaTeX} soit \bdocinEN{\LaTeX\ en ligne}.
\end{bdocrem}


% ------------------ %


\subsection{Codes tapés directement}


% ------------------ %


\begin{bdocexa}[Face à face]
	Dans le code suivant, l'option \bdocinlatex#sbs# est pour \bdocquote{\bdocprewhy{s}{ide} \bdocprewhy{b}{y} \bdocprewhy{s}{ide}} soit \bdocinEN{côte à côte}.

	\begin{bdoc-doc-latex-alone}
\begin{bdoclatex}[sbGs]
    $A = B + C$
\end{bdoclatex}
	\end{bdoc-doc-latex-alone}

	Ceci donne :

	\begin{bdoclatex}[sbs]
		$A = B + C$
	\end{bdoclatex}
\end{bdocexa}


% ------------------ %


\begin{bdocexa}[À la suite]
	L'environnement \bdocenv{bdoclatex} choisit par défaut l'option \bdocinlatex#std#
	\footnote{
    	\bdocinlatex{std} fait référence au comportement \bdocquote{standard} de \bdocpack{tcolorbox} vis à vis de la librairie \bdocpack{minted}.
	}
	et produit le résultat suivant.

	\begin{bdoclatex}
		$A = B + C$
	\end{bdoclatex}
\end{bdocexa}	


% ------------------ %


\begin{bdocexa}[Juste le code]
	Afficher le code seul comme ci-après s'obtient via l'option \bdocinlatex#code# de l'environnement \bdocenv{bdoclatex}.

	\begin{bdoclatex}[code]
    	$A = B + C$
	\end{bdoclatex}
\end{bdocexa}


% ------------------ %


\begin{bdocrem}	
	Si le code commence par un crochet ouvrant, dans le cas de l'option par défaut, il faudra l'indiquer. Voici un cas d'usage.

	\begin{bdoc-doc-latex-alone}
\begin{bdoclatex}[std]
    [étrange...]
\end{bdoclatex}
	\end{bdoc-doc-latex-alone}

	Ceci donne :
	
    \begin{bdoclatex}[std]
        [étrange...]
    \end{bdoclatex}
\end{bdocrem}

\end{document}
