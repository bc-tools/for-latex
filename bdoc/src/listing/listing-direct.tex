\documentclass[12pt, a4paper]{article}

\usepackage[utf8]{inputenc}
\usepackage[T1]{fontenc}

\usepackage[french]{babel, varioref}

\usepackage{enumitem}
\frenchsetup{StandardItemLabels=true}


\usepackage[lang = french]{../main/main}
\usepackage{../macroenv/macroenv}
\usepackage{../inenglish/inenglish}
\usepackage{../showcase/showcase}
\usepackage{../focus/focus}

% TESTING LOCAL IMPLEMENTATION %

\usepackage{listing}


\begin{document}

\section{Cas d'utilisation en \LaTeX}

\subsection{Codes \bdocquote{en ligne}}

La macro \bdocmacro{bdocinlatex} permet de taper du code en ligne via un usage similaire à \bdocmacro{verb}.
Voici deux exemples d'utilisation.

\begin{enumerate}
    \item \bdocinlatex+\bdocinlatex|$a^b = c$|+ produit \bdocinlatex|$a^b = c$| .

    \item \bdocinlatex+\bdocinlatex|$a^b = c$|+ vient de \bdocinlatex#\bdocinlatex+\bdocinlatex|$a^b = c$|+# obtenu via \bdocinlatex@\bdocinlatex#\bdocinlatex+\bdocinlatex|$a^b = c$|+#@ ... etc.
\end{enumerate}


\begin{bdocinfo}
    La macro \bdocmacro{bdocinlatex} est utilisable dans une note de pied de page. En voici la preuve
    \footnote{
    	\bdocinlatex+$minted = TOP$+ a été tapé \bdocinlatex|\bdocinlatex+$minted = TOP$+| dans cette note de bas de page..
    }.
\end{bdocinfo}


\begin{bdocrem}
    Le nom de la macro \bdocmacro{bdocinlatex} vient de \bdocquote{\bdocprewhy{in.line} \LaTeX} soit \bdocinEN{\LaTeX\ en ligne}.
\end{bdocrem}


% ------------------ %


\subsection{Codes tapés directement}

\begin{bdocexa}[Face à face]
	Dans le code suivant, l'option \bdocinlatex#sbs# est pour \bdocquote{\bdocprewhy{s.ide} \bdocprewhy{b.y} \bdocprewhy{s.ide}} soit \bdocinEN{côte à côte}.
	
	\bdoclatexshow{examples/listing/ABC.tex}
\end{bdocexa}


% ------------------ %


\begin{bdocexa}[À la suite]
	L'environnement \bdocenv{bdoclatex} choisit par défaut l'option \bdocinlatex#std#
	\footnote{
    	\bdocinlatex{std} fait référence au comportement \bdocquote{standard} de \bdocpack{tcolorbox} vis à vis de la librairie \bdocpack{minted}.
	}
	et produit le résultat suivant.

	\begin{bdoclatex}
		$A = B + C$
	\end{bdoclatex}
\end{bdocexa}	


% ------------------ %


\begin{bdocexa}[Juste le code]
	Afficher le code seul comme ci-après s'obtient via l'option \bdocinlatex#code# de l'environnement \bdocenv{bdoclatex}.

	\begin{bdoclatex}[code]
    	$A = B + C$
	\end{bdoclatex}
\end{bdocexa}


% ------------------ %


\begin{bdocwarning}	
	Avec la mise en forme par défaut, si le code commence par un crochet ouvrant, il faudra indiquer explicitement l'option par défaut. Voici un cas d'usage.
	
	\bdoclatexshow*<Ceci permet d'obtenir ce qui est attendu :>{examples/listing/strange.tex}
\end{bdocwarning}

\end{document}
