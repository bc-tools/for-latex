\documentclass[12pt, a4paper]{article}

\usepackage[utf8]{inputenc}
\usepackage[T1]{fontenc}

\usepackage[french]{babel, varioref}

\usepackage{enumitem}
\frenchsetup{StandardItemLabels=true}


\usepackage[lang = FR]{../main/main}
\usepackage{../macroenv/macroenv}
\usepackage{../inenglish/inenglish}
\usepackage{../showcase/showcase}
\usepackage{../rem-exa/rem-exa}

\usepackage{listing}


% == NEEDED == %

% Source.
%    * https://tex.stackexchange.com/a/604698/6880

\NewDocumentCommand{ \bdocextraruler }{ m }{%
    \par
    {
        \centering
        \color{green!50!black}%
        \leavevmode
        \kern.075\linewidth
        \leaders\hrule height3.25pt\hfill\kern0pt
        \footnotesize\itshape\bfseries\space\ignorespaces#1\unskip\space
        \leaders\hrule height3.25pt\hfill\kern0pt
        \kern.075\linewidth
        \par
    }
}

\NewDocumentEnvironment{ bdoc-extra-showcase }
                       { O{ Début du rendu dans cette doc. }
                         O{ Fin du rendu dans cette doc. } }{
    \begin{colorstrip}{green!5}
        \bdocextraruler{#1}
        \smallskip
}{
        \smallskip
        \bdocextraruler{#2}
    \end{colorstrip}
}


\begin{document}

\section{Cas d'utilisation en \LaTeX}

\subsection{Codes \bdocquote{en ligne}}

La macro \bdocmacro{bdocinlatex} permet de taper du code en ligne via un usage similaire à \bdocmacro{verb}.
Voici deux exemples d'utilisation.

\begin{enumerate}
    \item \bdocinlatex+\bdocinlatex|$a^b = c$|+ produit \bdocinlatex|$a^b = c$| .

    \item \bdocinlatex+\bdocinlatex|$a^b = c$|+ vient de \bdocinlatex#\bdocinlatex+\bdocinlatex|$a^b = c$|+# lui-même obtenu via \bdocinlatex@\bdocinlatex#\bdocinlatex+\bdocinlatex|$a^b = c$|+#@ , etc.
\end{enumerate}


\begin{bdocrem}
    Le nom de la macro \bdocmacro{bdocinlatex} vient de \bdocquote{\bdocprewhy{in}{line} \LaTeX} soit \bdocinEN{\LaTeX\ en ligne}.
\end{bdocrem}


% ------------------ %


\subsection{Codes tapés directement}


% ------------------ %


\bdocexa[Face à face]

\begin{bdoclatex-alone}
\begin{bdoclatex}
    $A = B + C$
\end{bdoclatex}
\end{bdoclatex-alone}

Ceci donne :

\begin{bdoclatex}
    $A = B + C$
\end{bdoclatex}


% ------------------ %


\bdocexa[À la suite]

Via \bdocenv{bdoclatex-flat}, on obtient un code à plat
\footnote{
    Le suffixe \bdocpre{flat} signifie \bdocinEN{plat}.
}
comme ci-dessous.

\begin{bdoclatex-flat}
    $A = B + C$
\end{bdoclatex-flat}


% ------------------ %


\bdocexa[Juste le code]

\bdocenv{bdoclatex-alone} affiche le code seul
\footnote{
    Le suffixe \bdocpre{alone} signifie \bdocinEN{seul}.
}
comme ci-après.

\begin{bdoclatex-alone}
    $A = B + C$
\end{bdoclatex-alone}

\end{document}
