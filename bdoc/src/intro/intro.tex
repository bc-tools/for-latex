\documentclass[12pt, a4paper]{article}

\usepackage[utf8]{inputenc}
\usepackage[T1]{fontenc}

\usepackage[french]{babel, varioref}

\usepackage{enumitem}
\frenchsetup{StandardItemLabels=true}


\usepackage[lang = french]{../main/main}
\usepackage{../macroenv/macroenv}
\usepackage{../inenglish/inenglish}


\begin{document}

\subsection{Introduction}

Le package \bdocpack{bdoc}
\footnote{
    Le nom vient de \bdocquote{\bdocprewhy{B}{asic} \bdocprewhy{Doc}{umentation}} qui ne nécessite aucune traduction.
}
facilite la saisie sémantique de documentations de packages \LaTeX\ avec un rendu sobre pour une lecture sur écran
\footnote{
    L'idée est de produire un fichier \texttt{PDF} efficace à parcourir pour des besoins ponctuels. C'est généralement ce que l'on attend d'une documentation liée au codage.
},
mais pratique pour les utilisateurs (ce package propose, ou impose, un style de mise en forme, mais il n'a pas pour vocation à être transformé en une classe).

\end{document}
