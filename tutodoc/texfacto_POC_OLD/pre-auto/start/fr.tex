\title{Le package \texttt{tutodoc} - Documentation de type tutoriel}
\author{Christophe BAL}
\date{23 Août 2024 - Version 1.2.0-a}

\maketitle

\begin{abstract}
Le package \tdocpack{tutodoc}
\footnote{
    Le nom vient de \tdocquote{\tdocprewhy{tuto.rial-type} \tdocprewhy{doc.umentation}} se traduit en \tdocquote{documentation de type tutoriel}.
}
est utilisé par son auteur pour produire de façon sémantique des documentations de packages et de classes \LaTeX\ dans un style de type tutoriel
\footnote{
    L'idée est de produire un fichier \texttt{PDF} efficace à parcourir pour des besoins ponctuels. C'est généralement ce que l'on attend d'une documentation liée au codage.
},
et avec un rendu sobre pour une lecture sur écran.


\begin{tdocnote}
     Ce package impose un style de mise en forme.
    Dans un avenir plus ou moins proche, \tdocpack{tutodoc} sera sûrement éclaté en une classe et un package.
\end{tdocnote}
\end{abstract}
