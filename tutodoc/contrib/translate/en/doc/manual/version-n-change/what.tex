\documentclass{tutodoc}

\usepackage[utf8]{inputenc}
\usepackage[T1]{fontenc}

\usepackage[french]{babel, varioref}

\usepackage{enumitem}
\frenchsetup{StandardItemLabels=true}

\usepackage{tabularray}

\usepackage[lang = french]{tutodoc}



\begin{document}

\subsection{What's new?}

\thisproj\ offers the macro \tdocmacro{tdocstartproj} and different environments to indicate quickly and clearly what has been done during the changes made, or to come.%
\footnote{
    The user doesn't need all the technical details.
}


\begin{tdocnote}
    For icons, see the note at the beginning of the section \ref{tutodoc-admonitions} on page \pageref{tutodoc-admonitions}.
\end{tdocnote}


\subsubsection{Sobriety first}

\foreach \exatitle/\filename in {
    {Just for the very first version}/%<-- Translate me!
        first,
    {For new features}/% <-- Translate me!
        new,
    {For updates}/% <-- Translate me!
        update,
    {For breaks}/% <-- Translate me!
        break,
    {For problems}/% <-- Translate me!
        pb,
    {For fixes}/% <-- Translate me!
        fix,
    {Roadmap}/% <-- Translate me!
        todo,
    {Technical information}/% <-- Translate me!
        tech,
    {Selectable themes with an icon}/% <-- Translate me!
        user-choice-icon,
    {Selectable themes without icons}/%<-- Translate me!
        user-choice%
} {
    \begin{tdocexa}[\exatitle]
        \leavevmode

        \tdoclatexinput[\tdoctcb{sbs}]{examples/version-n-change/chges-\filename.tex}
    \end{tdocexa}
}


\subsubsection{De la couleur si besoin}

\subsubsection{Color if necessary}

It may be useful to highlight certain changes: this can only be done by modifying the content color.

\foreach \exatitle/\filename in {
    {A flashy first version}/%<-- Translate me!
        first,
    {Outstanding fixes}/% <-- Translate me!
        fix%
} {
    \begin{tdocexa}[\exatitle]
        \leavevmode

        \tdoclatexinput[\tdoctcb{sbs}]{examples/version-n-change/color-chges-\filename.tex}
    \end{tdocexa}
}

\end{document}
