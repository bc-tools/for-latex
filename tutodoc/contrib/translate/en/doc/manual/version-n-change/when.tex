\documentclass[10pt, a4paper]{tutodoc}

\usepackage[utf8]{inputenc}
\usepackage[T1]{fontenc}

\usepackage[french]{babel, varioref}

\usepackage{enumitem}
\frenchsetup{StandardItemLabels=true}

\usepackage{tabularray}

\usepackage[lang = french]{tutodoc}



\begin{document}

\section{Indicate changes}

To make it easier to monitor a project, it is essential to provide a history indicating the changes made when a new version is published.



\subsection{When?}
\label{tutodoc-changes-when}

You can either date something, or version it, in which case the version number can be dated.


% ------------------ %


\begin{tdocexa}[Dating new products]
    The \tdocmacro{tdocdate} macro is used to indicate a date in the margin, as in the following example.

    \tdoclatexshow{examples/version-n-change/dating.tex}
\end{tdocexa}


% ------------------ %


\begin{tdocexa}[Versioning new features, possibly with a date]
    Associating a version number with a new feature is done using the \tdocmacro{tdocversion} macro, with the colour and date being optional arguments.

    \tdoclatexshow{examples/version-n-change/versioning.tex}
\end{tdocexa}


\begin{tdocimp}
    \begin{enumerate}[wide]
        \item The \tdocmacro{tdocdate} and \tdocmacro{tdocversion} macros require two compilations.

        \item The final rendering of the dates takes into account the language detected by \thisproj{}: for example, if French is selected, the dates will be displayed in the format \texttt{DD/MM/YYYY}.
    \end{enumerate}
\end{tdocimp}


\begin{tdoccaut}
    Only the use of the digital format \tdocinlatex+YYYY-MM-DD+ is verified,
    \footnote{
        Technically, checking the validity of a date using \LaTeX3 presents no difficulty.
    }
    and this is a choice! Why? Quite simply because dating and versioning explanations should be done semi-automatically to avoid any human bugs.
\end{tdoccaut}


\begin{tdocwarn}
    Behind the scenes, most of the work is done by the \tdocmacro{marginnote} macro in the eponymous package.
    The \tdocmacro{tdocdate} and \tdocmacro{tdocversion} macros therefore suffer from the same side effects as in the following example.
    However, these problems can be solved using an optional argument \tdocinlatex{<vertical-displacement>}, as at the end of the following example. It's ugly, but functional.

    \tdoclatexshow[nostripe, colstripe=red]{examples/version-n-change/side-effect.tex}
\end{tdocwarn}

\end{document}
