\documentclass[theme = color]{tutodoc}

\usepackage{multicol}
\usepackage{lastpage}

\renewcommand*\pagemark{%
  \usekomafont{pagenumber}{\thepage\kern1pt/\kern1pt\pageref*{LastPage}}%
}


\newcommand\thisstyle{color}


\newcommand\myexrmktext{
    \tdocversion{1.7.0}[2024-12-04]
    In the flow of the text, it is always useful to be able to provide examples and comments to supplement the main content.
}


\newcommand\myadmotext{
    Depending on the context of use, it is sometimes necessary to be able to highlight content by indicating its degree of importance.
}

\newcommand\myhighlightedtextnonote{
    What to say?
    I don't know, but it's nice. No ?
}

\newcommand\myhighlightedtext{
    What to say\,%
    \footnote{
        Let's not forget the footnotes...
    }?
    I don't know, but it's nice. No ?
}


\newcommand\mychgestext{
    In a change log, it is important to visualise the types of changes clearly. This makes it easier for the user to read!
}


\ExplSyntaxOn

\int_new:N \g__tutodoc_for_doc_int

\ExplSyntaxOff



\begin{document}

\textsf{\Huge\bfseries The \texttt{"\thisstyle"} theme}

\section{Liens}

{\Large\bfseries \href{https://github.com/bc-tools/for-latex/tree/main/tutodoc}{A very big link}}, but at least we can see it.



\section{\LaTeX\ listings}

Typing inline code such as \tdoclatexin{E = m c^2 \neq \pi \neq \frac{3}{14}} is useful, as is showing use cases such as the following one.

\begin{tdoclatex}
Formatted \LaTeX\ code is great: $E = m c^2$ or $pi \neq \frac{3}{14}$.
\end{tdoclatex}


There's also a less invasive side-by-side mode. Nice! No ?

\begin{tdoclatex}<\tdoctcb{sbs}>
Formatted \LaTeX\ code is great:       \\
$E = m c^2$ or $\pi \neq \frac{3}{14}$.
\end{tdoclatex}



\section{Highlighting, versioning and dating}

\subsection{tdocexa, tdocrem}

\myexrmktext

\ExplSyntaxOn

\seq_map_inline:Nn \g__tutodoc_focus_std_seq {
    \begin{tdoc#1}
        \myhighlightedtext
    \end{tdoc#1}
}

\ExplSyntaxOff



\subsection{tdocnote, tdoctip...}

\myadmotext

\ExplSyntaxOn

\int_set:Nn \g__tutodoc_for_doc_int { 0 }

\ifcsundef{g__tutodoc_focus_color_seq}{
    \prop_map_inline:Nn \g__tutodoc_focus_color_prop {
        \int_gincr:N \g__tutodoc_for_doc_int

        \begin{tdoc#1}
      	  	\int_compare:nTF
	    		{\g__tutodoc_for_doc_int = 1 }
	    		{ \myhighlightedtext }
	    		{ \myhighlightedtextnonote }
        \end{tdoc#1}
    }
} {
    \seq_map_inline:Nn \g__tutodoc_focus_color_seq {
        \int_gincr:N \g__tutodoc_for_doc_int

        \begin{tdoc#1}
      	  	\int_compare:nTF
	    		{\g__tutodoc_for_doc_int = 1 }
	    		{ \myhighlightedtext }
	    		{ \myhighlightedtextnonote }
        \end{tdoc#1}
    }
}

\ExplSyntaxOff



\subsection{tdocbreak, tdocfix...}

\tdocstartproj{A new demonstration section...}

\begin{tdoctodo}
	\item A gallery would be welcome...
\end{tdoctodo}

\ifcsundef{g__tutodoc_topic_change_seq}{
 	\medskip
}{}

\mychgestext

\ifcsundef{g__tutodoc_topic_change_seq}{
	\medskip
}{}

\ExplSyntaxOn

\int_set:Nn \g__tutodoc_for_doc_int { 0 }

\begin{tabular}{%
	@{\hskip 0pt}p{.26\linewidth}%
	*{3}{@{\hskip 7pt}p{.23\linewidth}}@{\hskip 0pt}%
}
	\ifcsundef{g__tutodoc_topic_change_seq}{
    	\prop_map_inline:Nn \g__tutodoc_topic_change_prop {
          	\int_gincr:N \g__tutodoc_for_doc_int

          	\vspace{-12pt}

    	  	\begin{tdoc#1}
           		\item Infos...
    	  	\end{tdoc#1}

    	  	\int_compare:nTF
    	    	{\g__tutodoc_for_doc_int = 4 }
    	    	{ \\ }
    	    	{ & }
    	}
	}{
    	\seq_map_inline:Nn \g__tutodoc_topic_change_seq {
          	\int_gincr:N \g__tutodoc_for_doc_int

          	\vspace{-5pt}

    	  	\begin{tdoc#1}
           		\item Infos...
    	  	\end{tdoc#1}

    	  	\int_compare:nTF
    	    	{\g__tutodoc_for_doc_int = 4 }
    	    	{ \\ }
    	    	{ & }
    	}
	}
\end{tabular}

\ExplSyntaxOff

\end{document}
