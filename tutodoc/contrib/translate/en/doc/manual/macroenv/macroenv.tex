\documentclass{tutodoc}

\usepackage[utf8]{inputenc}
\usepackage[T1]{fontenc}

\usepackage[french]{babel, varioref}
\frenchsetup{StandardItemLabels=true}

\usepackage{enumitem}

\usepackage{multicol}

\newcommand\thispack{\tdocpack{tutodoc}}



\begin{document}

\section{Specify packages, classes, macros or environments}

Here's what you can type semantically.


\begin{tdoclatex}[\tdocstyle{sbs}]
\tdoccls{myclass} is for...              \\
\tdocpack{mypackage} is for...           \\
\tdocmacro{onemacro} is for...           \\
\tdocenv{env} produces...                \\
\tdocenv[{[opt1]<opt2>}]{env}            \\
Just \tdocenv*{env}...                   \\
Finally \tdocenv*[{[opt1]<opt2>}]{env}...
        % For copy and paste.
\end{tdoclatex}


\begin{tdocrem}
    Unlike \tdocmacro{tdoclatexin}, \tdocmacro{tdocenv} and \tdocmacro{tdocenv*} macros don't color the text they produce.
    In addition, \tdoclatexin{\tdocenv{monenv}} produces \tdocenv{monenv} with spaces to allow line breaks if required.
\end{tdocrem}


\begin{tdocwarn}
    The optional argument of the \tdocmacro{tdocenv} macro is copied and pasted
    \footnote{
        Remember that almost anything is possible from now on.
    }
    when rendering. This may sometimes require the use of protective braces, as in the example above.
\end{tdocwarn}



\section{Origin of a prefix or suffix}

To explain the names chosen, there is nothing like indicating and explaining the short prefixes and suffixes used. This is easily done as follows.


\begin{tdoclatex}[\tdocstyle{sbs}]
\tdocpre{sup} relates to...      \\
\tdocprewhy{sup.erbe} means...   \\
\emph{\tdocprewhy{sup.er} for...}
\end{tdoclatex}


\begin{tdocrem}
    The choice of a full stop to split a word allows words with a hyphen to be used, as in \tdoclatexin+\tdocprewhy{bric.k-breaker}+ which gives \tdocprewhy{bric.k-breaker}.
\end{tdocrem}

\end{document}
