\documentclass[10pt, a4paper]{tutodoc}

\usepackage[utf8]{inputenc}
\usepackage[T1]{fontenc}

\usepackage[french]{babel, varioref}

\usepackage{enumitem}
\frenchsetup{StandardItemLabels=true}

\usepackage{tabularray}

\usepackage[lang = french]{tutodoc}



\begin{document}

\section{Highlighting content}

\begin{tdocnote}
    The environments presented in this section
    \footnote{
        The formatting comes from the \tdocpack{keytheorems} package.
    }
    add a short title indicating the type of information provided.
    This short text will always be translated into the language detected by the \thisproj\ class.
\end{tdocnote}


\subsection{Content in the reading flow}


% ------------------ %


\begin{tdocimp}
    All the environments presented in this section share the same counter, which will be reset to zero as soon as a section with a level at least equal to a \tdocinlatex|\section| is opened.
\end{tdocimp}


% ------------------ %


\subsubsection{Examples}

Numbered examples, if required, are indicated via \tdocenv{tdocexa}, which offers an optional argument for adding a mini-title.
Here are two possible uses.

\tdoclatexinput[sbs]{examples/admonitions/exa.tex}


% ------------------ %


\begin{tdoctip}
    It can sometimes be useful to return to the line at the start of the content. The code below shows how to proceed (this trick also applies to the \verb#tdocrem# environment presented next). Note in passing that the numbering follows that of the previous example as desired.
\end{tdoctip}

\tdoclatexinput[sbs]{examples/admonitions/exa-leavevmode.tex}

\end{document}
