\documentclass[10pt, a4paper]{tutodoc}

\usepackage[utf8]{inputenc}
\usepackage[T1]{fontenc}

\usepackage[french]{babel, varioref}

\usepackage{enumitem}
\frenchsetup{StandardItemLabels=true}

\usepackage{tabularray}

\usepackage[lang = french]{tutodoc}



\begin{document}

\subsection{Flashy content}
\label{tutodoc-admonitions}

\begin{tdocnote}
	The formatting proposed here is the default one, but others are possible by changing the theme: see the gallery of use cases in the appendix page \pageref{tutodoc-theme-gallery}.
	As for the icons, they are obtained via the \tdocpack{fontawesome5} package, and the \tdocmacro{tdocicon} macro manages the spacing in relation to the text.
	\footnote{
		For example,
		\tdocinlatex|\fbox{tdocicon{faBed}{Fatigued}}|
		produces
		\fbox{\tdocicon{\faBed}{Fatigued}},.
	}
\end{tdocnote}


\subsubsection{A tip}

The \tdocenv*{tdoctip} environment is used to give tips. Here's how to use it.

\tdoclatexinput[sbs]{examples/admonitions/tip.tex}


\smallskip


\begin{tdoctip}
    Sometimes, highlighted content can be reduced to a list. In this case, the formatting can be improved as follows where we use the \tdocinlatex{wide} option from the \tdocpack{enumitem} package.

	\tdoclatexinput[sbs]{examples/admonitions/leavevmode-items.tex}
\end{tdoctip}


\foreach \sectitle/\desc/\filename in {
    {Informative note}/%<-- Translate me!
    {The \tdocenv*{tdocnote} environment is used to highlight useful information. Here's how to use it.}/%<-- Translate me!
    note,
    %
    {Something important}/%<-- Translate me!
    {The \tdocenv*{tdocimp} environment is used to indicate something important but harmless.}/%<-- Translate me!
    important,
    %
    {Caution about a delicate point}/%<-- Translate me!
    {The \tdocenv*{tdoccaut} environment is used to indicate a delicate point to the user. Here's how to use it.}/% <-- Translate me!
    caution,
    %
    {Warning of danger}/%<-- Translate me!
    {The \tdocenv*{tdocwarn} environment is used to warn the user of a trap to avoid. Here's how to use it.}/%<-- Translate me!
    warn%
} {
    \subsubsection{\sectitle}

    \desc

    \tdoclatexinput[sbs]{examples/admonitions/\filename.tex}
}

\end{document}
