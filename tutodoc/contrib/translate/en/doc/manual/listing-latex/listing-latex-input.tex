\documentclass{tutodoc}

\usepackage[utf8]{inputenc}
\usepackage[T1]{fontenc}

\usepackage[french]{babel, varioref}

\usepackage{enumitem}
\frenchsetup{StandardItemLabels=true}

\usepackage{tabularray}

\usepackage[lang = french]{tutodoc}



\begin{document}

\subsection{Imported codes}

For the following codes, consider a file with the relative path \verb+examples-listing-xyz.tex+, and with the following contents.

\tdoclatexinput<\tdoctcb{code}>{examples/listing-latex/xyz.tex}

\medskip

The \tdocmacro{tdoclatexinput} macro, shown below, expects the path of a file and offers the same system of options between square brackets, or rafters, as the environment \tdocenv*{tdoclatex}.


% ------------------ %


\begin{tdocexa}[Side by side]
    \leavevmode

    \begin{tdoclatex}<\tdoctcb{code}>
\tdoclatexinput<\tdoctcb{sbs}>{examples/listing-latex/xyz.tex}
    \end{tdoclatex}

    This produces the following layout.

    \tdoclatexinput<\tdoctcb{sbs}>{examples/listing-latex/xyz.tex}
\end{tdocexa}


% ------------------ %


\begin{tdocexa}[Following]
    \leavevmode

    \begin{tdoclatex}<\tdoctcb{code}>
\tdoclatexinput{examples/listing-latex/xyz.tex}
    \end{tdoclatex}

    This produces the following formatting, which also corresponds to the option \tdoclatexin#tdoctcb{std}#\,.

    \tdoclatexinput{examples/listing-latex/xyz.tex}
\end{tdocexa}


% ------------------ %


\begin{tdocexa}[Just the code]
    \leavevmode

    \begin{tdoclatex}<\tdoctcb{code}>
\tdoclatexinput<\tdoctcb{code}>{examples/listing-latex/xyz.tex}
    \end{tdoclatex}

    This produces the following layout.

    \tdoclatexinput<\tdoctcb{code}>{examples/listing-latex/xyz.tex}
\end{tdocexa}


\begin{tdocexa}[Customise]
    \leavevmode

    \begin{tdoclatex}<\tdoctcb{code}>
\tdoclatexinput[style=igor, showspaces]<\tdoctcb{code}>{examples/listing-latex/xyz.tex}
    \end{tdoclatex}

    This produces the following layout.

	\tdoclatexinput[style = igor, showspaces]<\tdoctcb{code}>{examples/listing-latex/xyz.tex}
\end{tdocexa}


% ------------------ %


\subsection{Imported codes put into practice}
\label{tutodoc-listing-latexshow}

\begin{tdocexa}[Showcase]
    The following comes from \tdoclatexin+\tdoclatexshow{examples-listing-xyz.tex}+.

    \smallskip

    \tdoclatexshow{examples/listing-latex/xyz.tex}
\end{tdocexa}


\begin{tdocnote}
    The default texts take into account the language detected by \thisproj.
\end{tdocnote}


% ------------------ %


\begin{tdocexa}[Changing the explanatory text]
    Using the key \tdoclatexin|explain|, you can use custom text. Thus, \tdoclatexin|tdoclatexshow[explain = Here is the rendering.]{examples-listing-xyz.tex}| will give the following.

    \smallskip

    \tdoclatexshow[explain = Here is the rendering.]{examples/listing-latex/xyz.tex}
\end{tdocexa}


% ------------------ %


\begin{tdocexa}[The options available]
    In addition to the explanatory text, it is also possible to use all the options of \tdocenv*{tdocshowcase} environment, see \ref{tutodoc-showcase} on page \pageref{tutodoc-showcase}.
    Here is an example to illustrate this.

    \medskip

    \tdoclatexinput<\tdoctcb{code}>{examples/listing-latex/latexshow-options.tex}

    \smallskip

    This will produce the following.

    \smallskip

    \tdoclatexshow[style      = stripe,
               col-stripe = orange,
               col-text   = blue!70!black,
               before     = Rendu ci-après.,
               explain    = Ce qui vient est coloré...,
               after      = Rendu fini.,]
               {examples/listing-latex/xyz.tex}

\end{tdocexa}

\end{document}
