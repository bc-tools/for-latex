\documentclass{tutodoc}

\usepackage[utf8]{inputenc}
\usepackage[T1]{fontenc}

\usepackage[french]{babel, varioref}

\usepackage{enumitem}
\frenchsetup{StandardItemLabels=true}

\usepackage{tabularray}

\usepackage[lang = french]{tutodoc}



\begin{document}

\section{Use cases in \LaTeX}
\label{tutodoc-listing-latex}

Documenting a package, or class, is best done through use cases showing both the code and the corresponding result.%
\footnote{
    Code is formatted using the \tdocpack{minted} and \tdocpack{tcolorbox} packages.
}


\subsection{\tdocquote{Inline} codes}
\label{tutodoc-listing-latex-inline}

\begin{tdocexa}[Standard use]
    The \tdocmacro{tdoclatexin} macro\,%
    \footnote{
        The name of the macro \tdocmacro{tdoclatexin} comes from \tdocquote{\tdocprewhy{in.line} \LaTeX}\,.
    }
    can be used to type code in line in a similar way to \tdocmacro{verb}, or as a standard macro (see the handling of braces in the latter case below).
    Here are some examples of use.%
    \footnote{
    	A background color is deliberately used to subtly highlight the \tdoclatexin{\LaTeX} codes.
    }

    \begin{tdoclatex}<\tdoctcb{sbs}>
1: \tdoclatexin|$a^b = c$|               \\
2: \tdoclatexin+\tdoclatexin|$a^b = c$|+ \\
3: \tdoclatexin{\tdoclatexin{$a^b = c$}}
	\end{tdoclatex}
\end{tdocexa}


\begin{tdocexa}[Possible options]
    As the \tdocmacro{tdoclatexin} macro is based on \tdocpack{minted}, you can use all the options taken into account by \tdocpack{minted}.
    Here are some examples.

    \begin{tdoclatex}<\tdoctcb{sbs}>
1: \tdoclatexin[style = bw]{$a^b = c$} \\
2: \tdoclatexin[style = igor,
                showspaces]{$a^b = c$}
	\end{tdoclatex}
\end{tdocexa}


\begin{tdocnote}
	The \tdocmacro{tdoclatexin} macro can be used in a footnote as shown below.%
    \footnote{
        \tdoclatexin+$minted = TOP$+ has been typed \tdoclatexin|\tdoclatexin+$minted = TOP$+| in this footnote.
    }
\end{tdocnote}


\subsection{Directly typed codes}

\begin{tdocexa}[Side by side]
    Displaying a code and its rendering side by side is done as follows where the macro \tdocmacro{tdoctcb} allows you to just type \tdoclatexin{tdoctcb{sbs}} instead of \tdoclatexin{listing side text} (\tdoclatexin#sbs# is for \tdocquote{\tdocprewhy{s.ide} \tdocprewhy{b.y} \tdocprewhy{s.ide}}, while \tdoclatexin#tcb# is the standard abbreviation for \texttt{tcolorbox}). Note the use of rafters, not square brackets (more on this later).

    \tdoclatexshow{examples/listing-latex/ABC.tex}
\end{tdocexa}


\begin{tdocexa}[Following]
    \tdocenv{tdoclatex} produces the following result (this default setting is also obtained by using \tdoclatexin#\tdoctcb{std}#).%
    \footnote{
        \tdoclatexin{std} refers to the \tdocquote{standard} behaviour of \tdocpack{tcolorbox} in relation to the \tdocpack{minted} library.
    }

    \begin{tdoclatex}
$A = B + C$
    \end{tdoclatex}
\end{tdocexa}


\begin{tdocexa}[Just the code]
    Via \tdoclatexin#\tdoctcb{code}#, we'll just get the code as below.

    \begin{tdoclatex}<\tdoctcb{code}>
$A = B + C$
    \end{tdoclatex}
\end{tdocexa}


\begin{tdocexa}[Customise]
	The \tdocenv*{tdoclatex} environment accepts two types of optional argument.
	%
	\begin{enumerate}
		\item Between classic square brackets, you can use any option taken into account by \tdocpack{minted}.

		\item Between rafters, you can use any option taken into account by the environments obtained via \tdocpack{tcolorbox}.
	\end{enumerate}

	For example, the following modifications can be made if required.%
	\footnote{
		This documentation uses the options between rafters to obtain correct rendering of code producing shaded frames: see the section \ref{tutodoc-admonitions} on page \pageref{tutodoc-admonitions}.
	}

    \tdoclatexshow{examples/listing-latex/ABC-custom.tex}
\end{tdocexa}

\medskip

\begin{tdocwarn}
    To obtain the default formatting for a code beginning with a bracket or a rafter, you'll need to do a bit of fiddling, as shown below.
    \tdoclatexshow{examples/listing-latex/strange.tex}

    \smallskip

    Another method is to use the \tdocmacro{string} primitive, as shown below.
    \tdoclatexshow{examples/listing-latex/strange-bis.tex}
\end{tdocwarn}

\end{document}
