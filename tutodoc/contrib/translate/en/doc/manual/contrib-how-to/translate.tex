\documentclass{tutodoc}

\usepackage[utf8]{inputenc}
\usepackage[T1]{fontenc}

\usepackage[french]{babel, varioref}

\usepackage{enumitem}
\frenchsetup{StandardItemLabels=true}

\usepackage{tabularray}

\usepackage[lang = french]{tutodoc}


% == FORDOC == %

\NewDocumentCommand{\mailsubject}{m}%  <-- Translate me!
  {subject \tdocquote{\texttt{tutodoc - CONTRIB - #1}}}

% Source: https://tex.stackexchange.com/a/424061/6880

\newcommand{\FTdirO}{}
\def\FTdirO(#1,#2,#3){%
  \FTfile(#1,{\color{blue!40!black}\faFolderOpen\hspace{-.35pt}}{\hspace{0.2em}#3})
  (tmp.west)++(0.8em,-0.4em)node(#2){}
  (tmp.west)++(1.5em,0)
  ++(0,-1.3em)
}

\newcommand{\FTdirC}{}
\def\FTdirC(#1,#2,#3){%
  \FTfile(#1,{\color{blue!40!black}\faFolder\hspace{.75pt}}{\hspace{0.2em}#3})
  (tmp.west)++(0.8em,-0.4em)node(#2){}
  (tmp.west)++(1.5em,0)
  ++(0,-1.3em)
}

\newcommand{\FTfile}{}
\def\FTfile(#1,#2){%
  node(tmp){}
  (#1|-tmp)++(0.6em,0)
  node(tmp)[anchor=west,black]{\tt #2}
  (#1)|-(tmp.west)
  ++(0,-1.2em)
}

\newcommand{\FTroot}{}
\def\FTroot{tmp.west}

\newcommand\contribtranslatedirtree{
  \begin{tikzpicture}%
    \draw[color=black, thick]
% en        : parent = \FTroot
% normal dir: (parentID, currentID, label)
% file      :       (parentID, label)
      \FTdirO(\FTroot,root,translate){
        \FTdirC(root,changes,changes){
        }
        \FTdirO(root,en,en) {
          \FTdirC(en,api,api) {
          }
          \FTdirC(en,doc,doc) {
          }
        }
        \FTdirC(root,fr,fr){
        }
        \FTdirC(root,status,status){
          \FTdirO(status,en,en) {
            \FTfile(en,api.yaml)
            \FTfile(en,manual.yaml)
          }
          \FTdirC(status,fr,fr){
          }
        }
        \FTfile(root,README.md)
        \FTfile(root,LICENSE.txt)
      };
  \end{tikzpicture}
}


\begin{document}

%\section{Contribute}

\subsection{Complete the translations}

\begin{tdocnote}
    The author of \thisproj\ manages the French and English versions of the translations.
\end{tdocnote}


\begin{tdoccaut}
    Although we're going to explain how to translate the documentation, it doesn't seem relevant to do so, as English should suffice these days.%
    \footnote{
      The existence of a French version is simply a consequence of the native language of the author of \thisproj.
    }
\end{tdoccaut}


\begin{figure}[ht]
    \centering
    \contribtranslatedirtree\
    \caption{Simplified view of the translation folder}
    \label{tutodoc-contrib-translate-dir}
\end{figure}


The translations are roughly organized as in figure \ref{tutodoc-contrib-translate-dir} where only the folders important for the translations have been \tdocquote{opened}\,.%}.
\footnote{
    This was the organization on October 5, 2024, but it's still relevant today.
}
\textbf{A little further down, the section \ref{tutodoc-contrib-translate} explains how to add new translations}.


\subsubsection{The \texttt{fr} and \texttt{en} folders}

These two folders, managed by the author of \thisproj, have the same organization; they contain files that are easy to translate even if you're not a coder.


\subsubsection{The \texttt{changes} folder}

This folder is a communication tool where important changes are indicated without dwelling on minor modifications specific to one or more translations.


\subsubsection{The \texttt{status} folder}

This folder is used to keep track of translations from the project's point of view. Everything is done via well-commented \verb#YAML# files, readable by a non-coder.


\subsubsection{The \texttt{README.md} and \texttt{LICENSE.txt} files}

The \texttt{LICENSE.txt} file is aptly named, while the \texttt{README.md} file takes up in English the important points of what is said in this section about new translations.


\subsubsection{New translations}
\label{tutodoc-contrib-translate}

\begin{tdocimp}
    The \verb#api# folder contains translations relating to the functionalities of \thisproj.
    Here you'll find \verb#TXT# files for editing with a text or code editor, but not with a word processor.
    The content of these files uses commented lines in English to explain what \thisproj\ will do; these lines begin with \verb#//#\,. Here's an extract from such a file, where translations are made after each \,$=$\ sign, without touching the preceding, as this initial piece is used internally by the \thisproj code.

    \tdocsep
    \vspace{-10pt}
    \begin{verbatim}
    // #1: year  in format YYYY like 2023.
    // #2: month in format MM   like 04.
    // #3: day   in format DD   like 29.
    date = #1-#2-#3

    // #1: the idea is to produce one text like
    //     "this word means #1 in English".
    in_EN = #1 in english\end{verbatim}
\end{tdocimp}


\begin{tdocnote}
    The \verb#doc# folder is reserved for documentation. It contains files of type \verb#TEX# that can be compiled directly for real-time validation of translations.
\end{tdocnote}


\begin{tdocwarn}
    Only start from one of the \verb#fr# and \verb#en# folders, as these are the responsibility of the \thisproj\ author.
\end{tdocwarn}


\medskip


\emph{\textbf{Let's say you want to add support for Italian from files written in English.}}%
\footnote{
    As mentioned above, there is no real need for the \texttt{doc} documentation folder.
}


\paragraph{Method 1 : \git.}

\begin{enumerate}
      \item Obtain the entire project folder via \thisrepo\,.
    Do not use the \verb#main# branch, which is used to freeze the latest stable versions of projects in the single \thismonorepo\ repository,.

      \item In the \verb#tutodoc/contrib/translate# folder, create a \verb#it# copy of the \verb#en# folder, with the short name of the language documented in
      \href{https://en.wikipedia.org/wiki/IETF_language_tag#List_of_common_primary_language_subtags}%
           {the page \tdocquote{IIETF language tag}}
      from \texttt{Wikipedia}.

      \item Once the translation is complete in the \verb#it# folder, you'll need to communicate it via \thisrepo\ using a classic \verb#git push#\,.
\end{enumerate}


\paragraph{Method 2 : communicate by e-mail.}

\begin{enumerate}
      \item By e-mail with the \mailsubject{en FOR italian}, request a version of the English translations (note the use of the English name for the new language).
    Be sure to respect the subject of the e-mail, as the author of \thisproj\ automates the pre-processing of this type of e-mail.

      \item You will receive a folder named \verb#italian# containing the English version of the latest translations.
    This folder will be the place for your contribution.

      \item Once the translation is complete, you will need to compress your \verb#italian# file in \verb#zip# or \verb#rar# format before sending it by e-mail with the \mailsubject{italian}\,.
\end{enumerate}

\end{document}
