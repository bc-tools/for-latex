\documentclass{tutodoc}

\usepackage[utf8]{inputenc}
\usepackage[T1]{fontenc}

\usepackage[french]{babel, varioref}
\frenchsetup{StandardItemLabels=true}

\usepackage{enumitem}

\usepackage{multicol}

\newcommand\thispack{\tdocpack{tutodoc}}



\begin{document}

\subsection{With framing lines}

To make the formatted \LaTeX\ code more visible, you can use the \tdoclatexin{rule} style, as in the following examples.


\begin{tdocexa}
	The option \tdoclatexin{style = rule} provides the following where the automatically added texts will adapt to the language found by \thisproj.

	\begin{tdocshowcase}[style = rule]
    Bla, bla, bla, bla, bla, bla, bla, bla, bla, bla, bla, bla, bla...
\end{tdocshowcase}

\end{tdocexa}


\begin{tdocexa}[Editable text and colours]
	You can easily obtain the following horror.

	\begin{tdocshowcase}[style      = rule,
                     col-stripe = red,
                     col-text   = orange!75!black,
                     before     = My beginning,
                     after      = My end]
    Bla, bla, bla, bla, bla, bla, bla, bla, bla, bla, bla, bla, bla...
\end{tdocshowcase}


	Here's the code that was used.%
	\footnote{
		The next section will justify the a priori strange choice of \tdoclatexin{col-stripe} instead of \tdoclatexin{col-rule}\,.
	}

	\tdoclatexinput<\tdoctcb{code}>{examples/showcase/rule-custom.tex}
\end{tdocexa}


\begin{tdocnote}
    In the previous example, the text uses the proposed darkened orange. However, the red is used as a base to obtain the colors used for the framing lines: the transformations used depend on the theme chosen.%
    \footnote{
        For example, the themes \tdoclatexin{bw} and \tdoclatexin{draft} ignore the key \tdoclatexin{col-stripe}!
    }
    You should also be aware that behind the scenes, the macro \tdocmacro{tdocruler} is used, it works as follows.

    \begin{tdoclatex}<\tdoctcb{std}>
\tdocruler[red]{A decorated pseudo-title}
    \end{tdoclatex}
\end{tdocnote}

\end{document}
