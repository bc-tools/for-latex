\documentclass{tutodoc}

\usepackage[utf8]{inputenc}
\usepackage[T1]{fontenc}

\usepackage[french]{babel, varioref}

\usepackage{enumitem}
\frenchsetup{StandardItemLabels=true}

\usepackage{tabularray}

\usepackage[lang = french]{tutodoc}



\begin{document}

\section{Du code dans d'autres langages de programmation}

Certains packages proposent des outils utilisables via le langage \lua\ depuis un document \LaTeX.%
\footnote{
	Pour les mathématiques, on peut citer \tdocpack{luacas} et \tdocpack{tkz-éléments}.
}
Pour ce type de projet, il est utile de pouvoir présenter des lignes de code \lua, mais comme \tdocpack{minted} est chargé en coulisse par \thisproj, il est facile de proposer des macros et des environnements pour la mise en forme de codes dans tous les langages supportés par \href{https://pygments.org/}{Pygments}.%
\footnote{
	\tdocpack{minted} s'appuie le projet \trademark{Python} \trademark{Pygments} pour la coloration des codes informatiques.
}



\subsection{Codes \tdocquote{en ligne}}

ZZZ

\end{document}
