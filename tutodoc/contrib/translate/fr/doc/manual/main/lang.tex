\documentclass[12pt, a4paper]{tutodoc}

\usepackage[utf8]{inputenc}
\usepackage[T1]{fontenc}

\usepackage[french]{babel, varioref}

\usepackage{enumitem}
\frenchsetup{StandardItemLabels=true}

\usepackage{tabularray}

\usepackage[lang = french]{tutodoc}



\begin{document}

\section{Quelle langue est utilisée par la classe \thisproj\ ?}

Cette documentation charge le package \tdocpack{babel} via \tdocinlatex|\usepackage[french]{babel}|\,.
Dès lors, la classe \thisproj\ repère \tdocinlatex|fr| comme langue principale utilisée par \tdocpack{babel}.%
\footnote{
	Techniquement, on utilise \tdocinlatex|\BCPdata{language}| qui renvoie une langue au format court.
}
Comme cette langue fait partie de la liste des langues prises en compte, voir ci-dessous, la classe \thisproj\ produira les effets attendus.
% Do not touch the following placeholder.
<<API-LANGS>>


\begin{tdoccaut}
	Si le choix de la langue principale n'est pas faite dans le préambule, le mécanisme employé échouera avec des effets de bord non voulus (voir l'avertissement qui suit).
\end{tdoccaut}


\begin{tdocwarn}
    Lorsqu'une langue n'est pas prise en compte par \thisproj, un message d'avertissement est émis, et l'anglais est alors choisi comme langue vis-à-vis de \thisproj.
\end{tdocwarn}


\begin{tdocnote}
    Le mécanisme utilisé devrait être compatible avec le package \tdocpack{polyglossia}.
\end{tdocnote}

\end{document}
