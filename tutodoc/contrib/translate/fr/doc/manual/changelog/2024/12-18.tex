\documentclass{tutodoc}

\usepackage[utf8]{inputenc}
\usepackage[T1]{fontenc}

\usepackage[french]{babel, varioref}

\usepackage{enumitem}
\frenchsetup{StandardItemLabels=true}

\usepackage{tabularray}

\usepackage[lang = french]{tutodoc}



\begin{document}

\begin{tdocfix}
	\item Documentation: les références aux outils pour indiquer des changements ont été incorrectement écrites comme caractéristiques des contenus colorés mis en avant.
\end{tdocfix}


\begin{tdocbreak}
	\item La macro \tdocmacro{tdocenv} et sa version étoilée ne proposent plus d'option.

	\item Cas d'utilisation \LaTeX: la présentation par défaut est plus sobre, et des options permettent d'ajouter les lignes cadrantes, ou la bande colorée. Voir juste après.
\end{tdocbreak}


\begin{tdocnew}
	\item Mise en forme de codes informatiques en plus de ceux spécifiquement en \LaTeX.
	%
	\begin{enumerate}
		\item Création de \tdocenv{tdoccode} et de \tdocmacro{tdoccodein}.

		\item Pour les macros pour du code en ligne, et les environnements pour des blocs de code, des options de type \tdocpack{minted} s'indiquent à l'intérieur de crochet de manière traditionnelle: \tdoclatexin{[options minted]}\,.

		\item Pour les environnements pour des blocs de code, des options de type \tdocpack{tcolorbox} s'indiquent à l'intérieur de chevrons: \tdoclatexin{<options tcolorbox>}\,.

		\item La nouvelle macro \tdocmacro{tdoctcb} permet d'utiliser des raccourcis pour les styles \tdocpack{tcolorbox} régulièrement utilisés.
	\end{enumerate}

	\item Documentation: une nouvelle section présente les outils de mise en forme de codes informatiques autres que les cas d'utilisation de \LaTeX.
\end{tdocnew}


\begin{tdocupdate}
	\item Les sous-sous-sections sont numérotées en lettres minuscules.

	\item Thèmes.
	%
	\begin{enumerate}
		\item Moins d'espace consommé.

		\item Les ombres ont une meilleure coloration.

		\item Pour tous les thèmes sauf le \tdoclatexin{draft}, le rayon des arcs des coins des cadres passe de \tdoclatexin{.75mm} à \tdoclatexin{.2pt}\,.

 		\item Cas d'utilisation de \LaTeX: avec le thème \tdoclatexin{color}, la couleur du fond passe de \tdoclatexin[bgcolor = yellow!4]{yellow!4} à \tdoclatexin{gray!5}.

		\item Dernières modifications: avec le thème \tdoclatexin{dark}, le texte \tdoclatexin{[Init]} produit par la macro \tdocmacro{tdocstartproj} utilise la même police que celle des titres des environnements pour indiquer des changements.
	\end{enumerate}
\end{tdocupdate}

\end{document}
