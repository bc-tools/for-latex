\documentclass[10pt, a4paper]{tutodoc}

\usepackage[utf8]{inputenc}
\usepackage[T1]{fontenc}

\usepackage[french]{babel, varioref}

\usepackage{enumitem}
\frenchsetup{StandardItemLabels=true}

\usepackage{tabularray}

\usepackage[lang = french]{tutodoc}



\begin{document}

\section{Indiquer des packages, des classes, des macros ou des environnements}

Voici ce qu'il est possible de taper de façon sémantique.


\begin{tdoclatex}[sbs]
\tdoccls{maclasse} sert à...           \\
\tdocpack{monpackage} est pour...      \\
\tdocmacro{unemacro} permet de...      \\
\tdocenv{env} produit...               \\
\tdocenv[{[opt1]<opt2>}]{env}          \\
Juste \tdocenv*{env}...                \\
Enfin \tdocenv*[{[opt1]<opt2>}]{env}...
\end{tdoclatex}


\begin{tdocrem}
	Contrairement à \tdocmacro{tdocinlatex}, les macros \tdocmacro{tdocenv} et \tdocmacro{tdocenv*} ne colorent pas le texte produit.
    De plus, \tdocinlatex{\tdocenv{monenv}} produit \tdocenv{monenv} avec des espaces afin d'autoriser des retours à la ligne si besoin.
\end{tdocrem}


\begin{tdocwarn}
	L'argument optionnel de la macro \tdocmacro{tdocenv} est copié-collé
    \footnote{
        Se souvenir que tout est possible ou presque dorénavant.
    }
    lors du rendu. Ceci peut donc parfois nécessiter d'utiliser des accolades protectrices comme dans l'exemple ci-dessus.
\end{tdocwarn}



\section{Origine d'un préfixe ou d'un suffixe}

Pour expliquer les noms retenus, rien de tel que d'indiquer et expliciter les courts préfixes et suffixes employés. Ceci se fait facilement comme suit.


\begin{tdoclatex}[sbs]
\tdocpre{sup} est relatif à...    \\
\tdocprewhy{sup.erbe} signifie... \\
\emph{\tdocprewhy{sup.er} pour...}
\end{tdoclatex}


\begin{tdocrem}
    Le choix du point pour scinder un mot permet d'utiliser des mots avec un tiret comme dans \tdocinlatex+\tdocprewhy{ca.sse-brique}+ qui donne \tdocprewhy{ca.sse-brique}.
\end{tdocrem}

\end{document}
