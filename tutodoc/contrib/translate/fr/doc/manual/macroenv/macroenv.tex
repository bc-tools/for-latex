\documentclass{tutodoc}

\usepackage[utf8]{inputenc}
\usepackage[T1]{fontenc}

\usepackage[french]{babel, varioref}

\usepackage{enumitem}
\frenchsetup{StandardItemLabels=true}

\usepackage{tabularray}

\usepackage[lang = french]{tutodoc}



\begin{document}

\section{Indiquer des packages, des classes, des macros ou des environnements}

Voici ce qu'il est possible de taper de façon sémantique.


\begin{tdoclatex}<\tdoctcb{sbs}>
\tdoccls{maclasse} sert à...      \\
\tdocpack{monpackage} est pour... \\
\tdocmacro{unemacro} permet de... \\
\tdocenv{env} produit...          \\
Juste \tdocenv*{env}...
\end{tdoclatex}


\begin{tdocrem}
    Contrairement à \tdocmacro{tdoclatexin}, les macros \tdocmacro{tdocmacro}, \tdocmacro{tdocenv} et \tdocmacro{tdocenv*} ne colorent pas le texte produit.
    De plus, \tdoclatexin{\tdocenv{monenv}} produit \tdocenv{monenv} avec des espaces sécables afin d'autoriser des retours à la ligne si besoin.
\end{tdocrem}



\section{Origine d'un préfixe ou d'un suffixe}

Pour expliquer les noms retenus, rien de tel que d'indiquer et expliciter les courts préfixes et suffixes employés. Ceci se fait facilement comme suit.


\begin{tdoclatex}<\tdoctcb{sbs}>
\tdocpre{sup} est relatif à...     \\
\tdocprewhy{sup.erbe} signifie...  \\
\emph{\tdocprewhy{sup.er} pour...}
\end{tdoclatex}


\begin{tdocrem}
    Le choix du point pour scinder un mot permet d'utiliser des mots avec un tiret comme dans \tdoclatexin+\tdocprewhy{ca.sse-brique}+ qui donne \tdocprewhy{ca.sse-brique}.
\end{tdocrem}

\end{document}
