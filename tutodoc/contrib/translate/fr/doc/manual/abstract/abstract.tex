\documentclass[10pt, a4paper, theme = color]{tutodoc}

\usepackage[utf8]{inputenc}
\usepackage[T1]{fontenc}

\usepackage[french]{babel, varioref}

\usepackage{enumitem}
\frenchsetup{StandardItemLabels=true}

\usepackage{tabularray}

\usepackage[lang = french]{tutodoc}



\begin{document}

\title{La classe \texttt{tutodoc} - Documentation de type tutoriel}
\author{<<AUTHOR>>}
\date{<<DATE-N-VERSION>>}

\maketitle


\begin{abstract}
    La classe \thisproj{}\,%
    \footnote{
        Le nom vient de \tdocquote{\tdocprewhy{tuto.rial-type} \tdocprewhy{doc.umentation}} qui se traduit en \tdocquote{documentation de type tutoriel}.
    }
    est utilisée par son auteur pour produire de façon sémantique des documentations de packages et de classes \LaTeX\ dans un style de type tutoriel\,%
    \footnote{
        L'idée est de produire un fichier \texttt{PDF} efficace à parcourir pour des besoins ponctuels. C'est généralement ce que l'on attend d'une documentation liée au codage.
    }
    via un rendu sobre pour une lecture sur écran.

    \smallskip

    \noindent
    \emph{\textbf{Remarque :} cette documentation est aussi disponible en <<DOC-LANGS>>.}
\end{abstract}

\end{document}
