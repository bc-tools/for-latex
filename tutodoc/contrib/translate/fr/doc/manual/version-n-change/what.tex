\documentclass{tutodoc}

\usepackage[utf8]{inputenc}
\usepackage[T1]{fontenc}

\usepackage[french]{babel, varioref}

\usepackage{enumitem}
\frenchsetup{StandardItemLabels=true}

\usepackage{tabularray}

\usepackage[lang = french]{tutodoc}



\begin{document}

\subsection{Quoi de neuf ?}

\thisproj\ propose la macro \tdocmacro{tdocstartproj} et différents environnements pour indiquer rapidement et clairement ce qui a été fait lors des changements faits, ou à venir.%
\footnote{
    L'utilisateur n'a pas besoin de tous les détails techniques.
}


\begin{tdocnote}
	Concernant les icônes, voir la note au début de la section \ref{tutodoc-admonitions} page \pageref{tutodoc-admonitions}.
\end{tdocnote}


\subsubsection{La sobriété avant tout}

\foreach \exatitle/\filename in {
    {Juste pour la toute première version}/%<-- Translate me!
        first,
    {Pour les nouveautés}/% <-- Translate me!
        new,
    {Pour les mises à jour}/% <-- Translate me!
        update,
    {Pour les bifurcations}/% <-- Translate me!
        break,
    {Pour les problèmes}/% <-- Translate me!
        pb,
    {Pour les réparations}/% <-- Translate me!
        fix,
    {Feuille de route}/% <-- Translate me!
        todo,
    {Informations techniques}/% <-- Translate me!
        tech,
    %
    {Thématiques aux choix avec une icône}/%<-- Translate me!
        user-choice-icon,
    {Thématiques aux choix sans icône}/% <-- Translate me!
        user-choice%
} {
    \begin{tdocexa}[\exatitle]
        \leavevmode

        \tdoclatexinput[sbs]{examples/version-n-change/chges-\filename.tex}
    \end{tdocexa}
}


\subsubsection{De la couleur si besoin}

Il peut être utile de mettre en avant certains changements : ceci n'est faisable qu'en modifiant la couleur du contenu.

\foreach \exatitle/\filename in {
    {Une première version tape-à-l'oeil}/%<-- Translate me!
        first,
    {Des réparations exceptionnelles}/% <-- Translate me!
        fix%
} {
    \begin{tdocexa}[\exatitle]
        \leavevmode

        \tdoclatexinput[sbs]{examples/version-n-change/color-chges-\filename.tex}
    \end{tdocexa}
}

\end{document}
