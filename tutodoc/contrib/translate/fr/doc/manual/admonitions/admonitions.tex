\documentclass[12pt, a4paper]{article}

\usepackage[utf8]{inputenc}
\usepackage[T1]{fontenc}

\usepackage[french]{babel, varioref}

\usepackage{enumitem}
\frenchsetup{StandardItemLabels=true}

\usepackage{tabularray}

\usepackage[lang = french]{tutodoc}



\begin{document}

\subsection{Du contenu tape-à-l'oeil} \label{tdoc-admonitions}

\begin{tdocnote}
    Les icônes sont obtenues via le package \tdocpack{fontawesome5}, et la gestion de l'espacement avec le texte est faite par la macro \tdocmacro{tdocicon}.
    \footnote{
        Par exemple,
        \tdocinlatex|\fbox{\tdocicon{\faBed}{Fatigué}}|
        produit\,
        \fbox{\tdocicon{\faBed}{Fatigué}}\,.
    }
\end{tdocnote}



\subsubsection{Une astuce}

L'environnement \tdocenv*{tdoctip} sert à donner des astuces. Voici comment l'employer.

\tdoclatexinput[sbs]{examples/admonitions/tip.tex}


\smallskip


\begin{tdocnote}
    Les couleurs sont obtenues via les macros développables \tdocmacro{tdocbackcolor} et \tdocmacro{tdocdarkcolor}.
    Pour des informations complémentaires à ce sujet, se reporter à la fin de la section \ref{tdoc-color-macros} page \pageref{tdoc-color-macros}.
\end{tdocnote}


\foreach \sectitle/\desc/\filename in {
	{Note informative}/% <-- Translate me!
	{L'environnement \tdocenv*{tdocnote} sert à mettre en avant des informations utiles. Voici comment l'utiliser.}/% <-- Translate me!
	note,
	%
	{Un truc important}/% <-- Translate me!
	{L'environnement \tdocenv*{tdocimp} permet d'indiquer quelque chose d'important mais sans danger.}/% <-- Translate me!
	important,
	%
	{Avertir d'un point très délicat}/% <-- Translate me!
	{L'environnement \tdocenv*{tdoccaut} sert à indiquer un point délicat à  l'utilisateur. Voici comment l'employer.}/%<-- Translate me!
	caution,
	%
	{Avertir d'un danger}/% <-- Translate me!
	{L'environnement \tdocenv*{tdocwarn} sert à avertir l'utilisateur d'un piège à éviter. Voici comment l'employer.}/% <-- Translate me!
	warn%
} {
	\subsubsection{\sectitle}

	\desc

	\tdoclatexinput[sbs]{examples/admonitions/\filename.tex}
}

\end{document}
