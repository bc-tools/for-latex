\documentclass{tutodoc}

\usepackage[utf8]{inputenc}
\usepackage[T1]{fontenc}

\usepackage[french]{babel, varioref}

\usepackage{enumitem}
\frenchsetup{StandardItemLabels=true}

\usepackage{tabularray}

\usepackage[lang = french]{tutodoc}



\begin{document}

\subsection{Du contenu tape-à-l'oeil}
\label{tutodoc-admonitions}

\begin{tdocnote}
    La mise en forme proposée ici est celle par défaut, mais d'autres sont possible en changeant de thème : voir la galerie de cas d'utilisation dans l'annexe page \pageref{tutodoc-theme-gallery}.
    Quant aux icônes, elles sont obtenues via le package \tdocpack{fontawesome5}, et la macro \tdocmacro{tdocicon} gère l'espacement vis-à-vis du texte.
    \footnote{
        Par exemple,
        \tdocinlatex|\fbox{\tdocicon{\faBed}{Fatigué}}|
        produit\,
        \fbox{\tdocicon{\faBed}{Fatigué}}\,.
    }
\end{tdocnote}



\subsubsection{Une astuce}

L'environnement \tdocenv*{tdoctip} sert à donner des astuces. Voici comment l'employer.

\tdoclatexinput[sbs]{examples/admonitions/tip.tex}


\smallskip


\begin{tdoctip}
    Quelque fois, un contenu mis en avant peut se réduire à une liste. Dans ce cas, la mise en forme peut être améliorée comme suit où nous utilisons l'option \tdocinlatex{wide} du package \tdocpack{enumitem}.

    \tdoclatexinput[sbs]{examples/admonitions/leavevmode-items.tex}
\end{tdoctip}


\foreach \sectitle/\desc/\filename in {
    {Note informative}/% <-- Translate me!
    {L'environnement \tdocenv*{tdocnote} sert à mettre en avant des informations utiles. Voici comment l'utiliser.}/% <-- Translate me!
    note,
    %
    {Un truc important}/% <-- Translate me!
    {L'environnement \tdocenv*{tdocimp} permet d'indiquer quelque chose d'important mais sans danger.}/% <-- Translate me!
    important,
    %
    {Avertir d'un point très délicat}/% <-- Translate me!
    {L'environnement \tdocenv*{tdoccaut} sert à indiquer un point délicat à  l'utilisateur. Voici comment l'employer.}/%<-- Translate me!
    caution,
    %
    {Avertir d'un danger}/% <-- Translate me!
    {L'environnement \tdocenv*{tdocwarn} sert à avertir l'utilisateur d'un piège à éviter. Voici comment l'employer.}/% <-- Translate me!
    warn%
} {
    \subsubsection{\sectitle}

    \desc

    \tdoclatexinput[sbs]{examples/admonitions/\filename.tex}
}

\end{document}
