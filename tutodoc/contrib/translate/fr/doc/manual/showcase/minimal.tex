\documentclass{tutodoc}

\usepackage[utf8]{inputenc}
\usepackage[T1]{fontenc}

\usepackage[french]{babel, varioref}
\frenchsetup{StandardItemLabels=true}

\usepackage{enumitem}

\usepackage{multicol}

\newcommand\thispack{\tdocpack{tutodoc}}



\begin{document}

\section{Un rendu en situation réelle}
\label{tutodoc-showcase}

Il est parfois utile d'obtenir directement le rendu d'un code dans la documentation. Ceci nécessite que ce type de rendu soit dissociable du texte donnant des explications.



\subsection{Un rendu minimaliste par défaut}

\begin{tdocexa}[Avec les textes par défaut]
    Il peut être utile de montrer un rendu réel directement dans un document.%
    \footnote{
        Typiquement lorsque l'on fait une démo.
    }
    Ceci se tape via \tdocenv{tdocshowcase} comme suit.

    \tdoclatexshow[explain = {On obtient alors le rendu suivant qui est juste la combinaison d'un faible espacement vertical et d'une simple importation.}]{examples/showcase/default.tex}
\end{tdocexa}


\begin{tdocrem}
    La section \ref{tutodoc-listing-latexshow} page \pageref{tutodoc-listing-latexshow} explique comment obtenir, via la macro \tdocmacro{tdoclatexshow}, un code suivi de son rendu réel comme dans l'exemple précédent.
\end{tdocrem}


\begin{tdocwarn}
    Avec les paramètres par défaut, si le code à formater commence par un crochet ouvrant, utilisez l'une des astuces suivantes.

    \tdoclatexshow[explain = {Ceci produira ce qui suit.}]{examples/showcase/hook.tex}
\end{tdocwarn}

\end{document}
