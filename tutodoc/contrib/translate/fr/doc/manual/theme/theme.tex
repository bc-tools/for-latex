\documentclass[10pt, a4paper]{tutodoc}

\usepackage[utf8]{inputenc}
\usepackage[T1]{fontenc}

\usepackage[french]{babel, varioref}

\usepackage{enumitem}
\frenchsetup{StandardItemLabels=true}

\usepackage{tabularray}

\usepackage[lang = french]{tutodoc}



\begin{document}

\section{Choisir son thème}

Pour modifier la mise en forme générale, la classe \thisproj\ propose l'option \tdocinlatex{theme = <choix>} où \tdocinlatex{<choix>} peut prendre les valeurs suivantes.

\begin{itemize}
    \item \tdocinlatex|bw| :
    ce thème est de type noir-et-blanc avec certaines nuances de gris.

    \item \tdocinlatex|color| :
    ce thème est coloré, \emph{c'est la valeur par défaut}.

    \item \tdocinlatex|dark| :
    ce thème est sombre, idéal pour se reposer les yeux.

    \item \tdocinlatex|draft| :
    ce thème est juste pour une impression papier à la recherche d'erreurs de contenu pas forcément simples à débusquer devant son écran.
\end{itemize}


\begin{tdocnote}
	A la fin de ce document, après l'historique, vous trouverez une galerie de cas d'utilisation de ces différents thèmes.
\end{tdocnote}

\end{document}
