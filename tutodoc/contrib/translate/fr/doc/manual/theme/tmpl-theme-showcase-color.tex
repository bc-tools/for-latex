\documentclass[theme = color]{tutodoc}

\usepackage[utf8]{inputenc}
\usepackage[T1]{fontenc}

\usepackage[french]{babel, varioref}

\usepackage{enumitem}
\frenchsetup{StandardItemLabels=true}

\usepackage{tabularray}

\usepackage[lang = french]{tutodoc}


% -- FORDOC -- %

\usepackage[french]{babel}
\frenchsetup{StandardItemLabels=true}

\usepackage{multicol}
\usepackage{lastpage}

\renewcommand*\pagemark{%
  \usekomafont{pagenumber}{\thepage\kern1pt/\kern1pt\pageref{LastPage}}%
}


\newcommand\thisstyle{color}


\newcommand\myexrmktext{
    \tdocversion{1.7.0}[2024-12-04]
    Dans le flot du texte, il est toujours utile de pouvoir indiquer des exemples et des remarques qui viennent compléter le contenu principal.
}


\newcommand\myadmotext{
    Suivant le contexte d'utilisation, il est parfois nécessaire de pouvoir mettre en avant des contenus en indiquant leur degré d'importance.
}

\newcommand\myhighlightedtextnonote{
    Que dire ?
    Je ne sais pas, mais c'est sympathique. Non ?
}

\newcommand\myhighlightedtext{
    Que dire\,%
    \footnote{
        N'oublions pas les notes de bas de page...
    }?
    Je ne sais pas, mais c'est sympathique. Non ?
}


\newcommand\mychgestext{
    Dans un journal de bord, il est important de bien visualiser les types de changements. Ceci rend plus efficace la lecture côté utilisateur !
}


\ExplSyntaxOn

\int_new:N \g__tutodoc_for_doc_int

\ExplSyntaxOff


\begin{document}

\textsf{\Huge\bfseries Le thème \texttt{"\thisstyle"}}

\section{Liens}

{\Large\bfseries \href{https://github.com/bc-tools/for-latex/tree/main/tutodoc}{Un lien très gros}}, mais au moins on le voit.



\section{Des codes \LaTeX}

Taper du code \LaTeX\ en ligne comme \tdoclatexin{E = m c^2 \neq \pi \neq \frac{3}{14}} est utile, tout comme montrer des cas d'utilisation comme le suivant.

\begin{tdoclatex}
Du code \LaTeX\ mis en forme, c'est top : $E = m c^2$ ou $\pi \neq \frac{3}{14}$.
\end{tdoclatex}


On dispose aussi d'un mode côte-à-côte moins envahissant. Sympa ! Non ?

\begin{tdoclatex}[sbs]
Du code \LaTeX\ mis en forme, c'est top : \\
$E = m c^2$ ou $\pi \neq \frac{3}{14}$.
\end{tdoclatex}



\section{Mettre en avant, versionner et dater}

\subsection{tdocexa, tdocrem}

\myexrmktext

\ExplSyntaxOn

\seq_map_inline:Nn \g__tutodoc_focus_std_seq {
    \begin{tdoc#1}
        \myhighlightedtext
    \end{tdoc#1}
}

\ExplSyntaxOff



\subsection{tdocnote, tdoctip...}

\myadmotext

\ExplSyntaxOn

\int_set:Nn \g__tutodoc_for_doc_int { 0 }

\ifcsundef{g__tutodoc_focus_color_seq}{
    \prop_map_inline:Nn \g__tutodoc_focus_color_prop {
        \int_gincr:N \g__tutodoc_for_doc_int
        
        \begin{tdoc#1}
      	  	\int_compare:nTF 
	    		{\g__tutodoc_for_doc_int = 1 }
	    		{ \myhighlightedtext }
	    		{ \myhighlightedtextnonote }
        \end{tdoc#1}
    }
} {
    \seq_map_inline:Nn \g__tutodoc_focus_color_seq {
        \int_gincr:N \g__tutodoc_for_doc_int
        
        \begin{tdoc#1}
      	  	\int_compare:nTF 
	    		{\g__tutodoc_for_doc_int = 1 }
	    		{ \myhighlightedtext }
	    		{ \myhighlightedtextnonote }
        \end{tdoc#1}
    }
}

\ExplSyntaxOff



\subsection{tdocbreak, tdocfix...}

\tdocstartproj{Une nouvelle section démonstrative...}

\begin{tdoctodo}
	\item Une galerie serait bienvenue...
\end{tdoctodo}

\ifcsundef{g__tutodoc_topic_change_seq}{
 	\medskip
}{}

\mychgestext

\ifcsundef{g__tutodoc_topic_change_seq}{
	\medskip
}{}

\ExplSyntaxOn

\int_set:Nn \g__tutodoc_for_doc_int { 0 }

\begin{tabular}{%
	@{\hskip 0pt}p{.26\linewidth}%
	*{3}{@{\hskip 7pt}p{.23\linewidth}}@{\hskip 0pt}%
}
	\ifcsundef{g__tutodoc_topic_change_seq}{
    	\prop_map_inline:Nn \g__tutodoc_topic_change_prop {
          	\int_gincr:N \g__tutodoc_for_doc_int
    
          	\vspace{-15pt}
    
    	  	\begin{tdoc#1}
           		\item Infos... 
    	  	\end{tdoc#1}
    	    
    	  	\int_compare:nTF 
    	    	{\g__tutodoc_for_doc_int = 4 }
    	    	{ \\ }
    	    	{ & }
    	}
	}{
    	\seq_map_inline:Nn \g__tutodoc_topic_change_seq {
          	\int_gincr:N \g__tutodoc_for_doc_int
    
          	\vspace{-5pt}
    
    	  	\begin{tdoc#1}
           		\item Infos... 
    	  	\end{tdoc#1}
    	    
    	  	\int_compare:nTF 
    	    	{\g__tutodoc_for_doc_int = 4 }
    	    	{ \\ }
    	    	{ & }
    	}
	}
\end{tabular}

\ExplSyntaxOff

\end{document}
