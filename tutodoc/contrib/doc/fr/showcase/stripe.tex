\documentclass[12pt, a4paper]{article}

\usepackage[utf8]{inputenc}
\usepackage[T1]{fontenc}

\usepackage[french]{babel, varioref}

\usepackage{enumitem}
\frenchsetup{StandardItemLabels=true}

\usepackage{tabularray}

\usepackage[lang = french]{tutodoc}



\begin{document}


\section{Un rendu en situation réelle} \label{tdoc-showcase}

Il est parfois utile d'obtenir directement le rendu d'un code dans la documentation. Ceci nécessite que ce type de rendu soit dissociable du texte donnant des explications.


% ------------------ %


\subsection{Avec une bande colorée}

\begin{tdocexa}[Avec les textes par défaut]
    Il peut être utile de montrer un rendu réel directement dans un document
    \footnote{
        Typiquement lorsque l'on fait une démo.
    }.
    Ceci se tape via \tdocenv{tdocshowcase} comme suit.

    \tdoclatexinput[code]{examples/showcase/default.tex}

    On obtient alors le rendu suivant
    \footnote{
        En coulisse, la bande est créée sans effort grâce au package \tdocpack{clrstrip}.
    }.

    \medskip

    \begin{bdocshowcase}
    \bfseries Un peu de code \LaTeX.

    \bigskip

    \emph{\large Fin de l'affreuse démo.}
\end{bdocshowcase}
\end{tdocexa}


\begin{tdocrem}
    Voir la section \ref{tdoc-latexshow} page \pageref{tdoc-latexshow} pour obtenir facilement un code suivi de son rendu réel comme dans l'exemple précédent.
\end{tdocrem}


\begin{tdocnote}
    Les textes explicatifs s'adaptent à la langue choisie lors du chargement de \tdocpack{tutodoc}.
\end{tdocnote}


% ------------------ %


\begin{tdocexa}[Changer la couleur et/ou les textes par défaut]
    \leavevmode

    \tdoclatexinput[code]{examples/showcase/customized.tex}

    Ceci produira ce qui suit.

    \medskip

    
\begin{bdocshowcase}[before = Mon début,   
                     after  = Ma fin à moi,
                     color  = red]
    Bla, bla, bla, bla, bla, bla, bla, bla, bla, bla, bla, bla, bla...
\end{bdocshowcase}
\end{tdocexa}


\begin{tdocnote}
    Vous avez sûrement noté que l'on n'obtient pas un rouge pur : en coulisse les macros développables \tdocmacro{tdocbackcolor} et \tdocmacro{tdocdarkcolor} sont utilisées pour créer celles du fond et des titres respectivement à partir de la couleur proposée à \tdocenv{tdocshowcase}.
    Ces macros à un seul argument, la couleur choisie, admettent les codes suivants.

    \begin{tdoclatex}[code]
\NewExpandableDocumentCommand{\tdocbackcolor}{m}{#1!5}
\NewExpandableDocumentCommand{\tdocdarkcolor}{m}{#1!50!black}
    \end{tdoclatex}
\end{tdocnote}


% ------------------ %


\begin{tdocwarn}
    Avec les réglages par défaut, si le code \LaTeX\ à mettre en forme commence par un crochet ouvrant, il faudra indiquer explicitement une option vide comme dans l'exemple suivant.

    \tdoclatexinput[code]{examples/showcase/hook.tex}

    Ceci produira ce qui suit.

    \medskip

    \begin{tdocshowcase}
    \string[Cela fonctionne...]
\end{tdocshowcase}

\end{tdocwarn}


\begin{tdocnote}
    Il faut savoir qu'en coulisse la macro \tdocmacro{tdocruler} est utilisée.

    \begin{tdoclatex}[std]
        \tdocruler{Un pseudo-titre décoré}{red}
    \end{tdoclatex}
\end{tdocnote}

\end{document}


\end{document}
    