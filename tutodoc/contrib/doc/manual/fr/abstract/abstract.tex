\documentclass[12pt, a4paper]{article}

\usepackage[utf8]{inputenc}
\usepackage[T1]{fontenc}

\usepackage[french]{babel, varioref}

\usepackage{enumitem}
\frenchsetup{StandardItemLabels=true}

\usepackage{tabularray}

\usepackage[lang = french]{tutodoc}



\begin{document}


\title{Le package \texttt{tutodoc} - Documentation de type tutoriel}
\author{Christophe BAL}
\date{AUTO-DATE-N-VERSION}

\maketitle

Le package \tdocpack{tutodoc}
\footnote{
    Le nom vient de \tdocquote{\tdocprewhy{tuto.rial-type} \tdocprewhy{doc.umentation}} qui se traduit en \tdocquote{documentation de type tutoriel}.
}
est utilisé par son auteur pour produire de façon sémantique des documentations de packages et de classes \LaTeX\ dans un style de type tutoriel
\footnote{
    L'idée est de produire un fichier \texttt{PDF} efficace à parcourir pour des besoins ponctuels. C'est généralement ce que l'on attend d'une documentation liée au codage.
},
et avec un rendu sobre pour une lecture sur écran.

\medskip

\noindent
Deux points importants à noter.
\begin{itemize}
    \item Ce package impose un style de mise en forme. Dans un avenir plus ou moins proche, \tdocpack{tutodoc} sera sûrement éclaté en une classe et un package.

    \item Cette documentation est aussi disponible en anglais.
\end{itemize}

\end{document}


\end{document}
    