\documentclass[12pt, a4paper]{article}

\usepackage[utf8]{inputenc}
\usepackage[T1]{fontenc}

\usepackage[french]{babel, varioref}

\usepackage{enumitem}
\frenchsetup{StandardItemLabels=true}

\usepackage{tabularray}

\usepackage[lang = french]{tutodoc}



\begin{document}

\subsection{Du contenu tape-à-l'oeil}  \label{tdoc-colorful-focus}

\begin{tdocnote}
    Les icônes sont obtenues via le package \tdocpack{fontawesome5}, et la gestion de l'espacement avec le texte est faite par la macro \tdocmacro{tdocicon}.
    \footnote{
        Par exemple,
        \tdocinlatex|\tdocicon{\faBed}{Fatigué}|
        produit\,
        \tdocicon{\faBed}{Fatigué}.
    }
\end{tdocnote}


\subsubsection{Une astuce}

L'environnement \tdocenv*{tdoctip} sert à donner des astuces. Voici comment l'employer.

\tdoclatexinput[sbs]{examples/focus/tip.tex}


\smallskip

\begin{tdocnote}
    Les couleurs sont fournies par les macros développables \tdocmacro{tdocbackcolor} et \tdocmacro{tdocdarkcolor} qui admettent les codes suivants et attendent une couleur au format \tdocpack{xcolor} comme argument.

    \begin{tdoclatex}[code]
\NewExpandableDocumentCommand{\tdocbackcolor}{m}{#1!5}
\NewExpandableDocumentCommand{\tdocdarkcolor}{m}{#1!50!black}
    \end{tdoclatex}
\end{tdocnote}

\end{document}
