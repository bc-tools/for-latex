\documentclass[10pt, a4paper]{article}

\usepackage[utf8]{inputenc}
\usepackage[T1]{fontenc}

\usepackage[french]{babel, varioref}
\frenchsetup{StandardItemLabels=true}

\usepackage{enumitem}

\usepackage{multicol}

\newcommand\thispack{\tdocpack{tutodoc}}



\begin{document}

\section{Indicate changes}

To make it easier to monitor a package, it is essential to provide a history indicating the changes made when a new version is published.


% ------------------ %


\subsection{When?}

You can either date something, or version it, in which case the version number can be dated.


% ------------------ %


\begin{tdocexa}[Dating new products]
    The \tdocmacro{tdocdate} macro is used to indicate a date in the margin, as in the following example.

    \tdoclatexshow{examples/version-n-change/dating.tex}
\end{tdocexa}


% ------------------ %


\begin{tdocexa}[Versioning new features, possibly with a date]
    Associating a version number with a new feature is done using the \tdocmacro{tdocversion} macro, with the colour and date being optional arguments.

    \tdoclatexshow{examples/version-n-change/versioning.tex}
\end{tdocexa}


\begin{tdocimportant}
    \leavevmode

    \begin{enumerate}
        \item The \tdocmacro{tdocdate} and \tdocmacro{tdocversion} macros require two compilations.

        \item The final rendering of the dates takes into account the language specified when loading the package \tdocpack{tutodoc}: for example, if French is selected, the dates will be displayed in the format \texttt{DD/MM/YYYY}.
    \end{enumerate}
\end{tdocimportant}


\begin{tdocwarn}
    Only the use of the digital format \tdocinlatex+YYYY-MM-DD+ is verified.
    \footnote{
        Technically, checking the validity of a date using \LaTeX3 presents no difficulty.
    },
    and this is a choice! Why? Quite simply because dating and versioning explanations should be done semi-automatically to avoid any human bugs.
\end{tdocwarn}

\end{document}
