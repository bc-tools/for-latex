\documentclass[10pt, a4paper]{article}

\usepackage[utf8]{inputenc}
\usepackage[T1]{fontenc}

\usepackage[french]{babel, varioref}
\frenchsetup{StandardItemLabels=true}

\usepackage{enumitem}

\usepackage{multicol}

\newcommand\thispack{\tdocpack{tutodoc}}



\begin{document}

\subsection{What's new?}

\thispack{} offers the macro \tdocmacro{tdocstartproj} and different environments to indicate quickly and clearly what has been done during the latest changes.%
\footnote{
    The user doesn't need all the technical details.
}


\begin{tdocnote}
    Icons are obtained via the \tdocpack{fontawesome5} package, and text spacing is managed by the \tdocmacro{tdocicon} macro.
\end{tdocnote}


% ------------------ %


\begin{tdocexa}[Just for the very first version]
    \leavevmode

    \tdoclatexinput[sbs]{examples/version-n-change/first.tex}
\end{tdocexa}


% ------------------ %


\begin{tdocexa}[For new features]
    \leavevmode

    \tdoclatexinput[sbs]{examples/version-n-change/new.tex}
\end{tdocexa}


% ------------------ %


\begin{tdocexa}[For updates]
    \leavevmode

    \tdoclatexinput[sbs]{examples/version-n-change/update.tex}
\end{tdocexa}


% ------------------ %


\begin{tdocexa}[For breaks]
    \leavevmode

    \tdoclatexinput[sbs]{examples/version-n-change/break.tex}
\end{tdocexa}


% ------------------ %


\begin{tdocexa}[For problems]
    \leavevmode

    \tdoclatexinput[sbs]{examples/version-n-change/pb.tex}
\end{tdocexa}


% ------------------ %


\begin{tdocexa}[For fixes]
    \leavevmode

    \tdoclatexinput[sbs]{examples/version-n-change/fix.tex}
\end{tdocexa}


% ------------------ %


\begin{tdocexa}[Selectable themes with an icon]
    \leavevmode

    \tdoclatexinput[sbs]{examples/version-n-change/user-choice-icon.tex}
\end{tdocexa}


% ------------------ %


\begin{tdocexa}[Selectable themes without icons]
    \leavevmode

    \tdoclatexinput[sbs]{examples/version-n-change/user-choice.tex}
\end{tdocexa}


% ------------------ %


\begin{tdocnote}
    The use of an icon followed by the correct spacing just before the text is done via the \tdocmacro{tdocicon} macro.
    For example,
    \tdocinlatex|tdocicon{faBed}{Fatigued}|
    product\,
    \tdocicon{\faBed}{Fatigued}.
\end{tdocnote}

\end{document}
