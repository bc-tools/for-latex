\documentclass[10pt, a4paper]{article}

\usepackage[utf8]{inputenc}
\usepackage[T1]{fontenc}

\usepackage[french]{babel, varioref}

\usepackage{enumitem}
\frenchsetup{StandardItemLabels=true}

\usepackage{tabularray}

\usepackage[lang = french]{tutodoc}



% == FORDOC == %

% Source.
%    * https://tex.stackexchange.com/a/604698/6880

\NewDocumentCommand{ \tdocdocbasicinput }{ m }{%
    Consider the following code.

    \tdoclatexinput[code]{#1}

    This will produce the following.

    \input{#1}
}


\begin{document}

\section{Use cases in \LaTeX}

Documenting a package or a class is done efficiently using use cases showing both the code and the corresponding result.




\subsection{\tdocquote{Inline} codes} \label{tdoc-listing-inline}

The \tdocmacro{tdocinlatex} macro
\footnote{
    The name of the macro \tdocmacro{tdocinlatex} comes from \tdocquote{\tdocprewhy{in.line} \LaTeX}.
}
can be used to type inline code in a similar way to \tdocmacro{verb}.
Here are some examples.


\begin{tdoclatex}[sbs]
    1: \tdocinlatex|$a^b = c$|

    2: \tdocinlatex+\tdocinlatex|$a^b = c$|+
\end{tdoclatex}


\begin{tdocnote}
    The \tdocmacro{tdocinlatex} macro can be used in a footnote: see below.
    \footnote{
        \tdocinlatex+$minted = TOP$+ has been typed \tdocinlatex|\tdocinlatex+$minted = TOP$+| in this footnote...
    }.
    In addition, a background color is deliberately used to subtly highlight the codes \tdocinlatex#\LaTeX#\,.
\end{tdocnote}


% ------------------ %


\subsection{Directly typed codes}

\begin{tdocexa}[Side by side]
    Using \tdocenv[{[sbs]}]{tdoclatex}, we can display a code and its rendering side by side.
    \tdocdocbasicinput{examples/listing/ABC.tex}
\end{tdocexa}


% ------------------ %


\begin{tdocexa}[Following]
    \tdocenv{tdoclatex} produces the following result, which corresponds to the default option \tdocinlatex#std#
    \footnote{
        \tdocinlatex{std} refers to the \tdocquote{standard} behaviour of \tdocpack{tcolorbox} in relation to the \tdocpack{minted} library.
    }.

    \begin{tdoclatex}
        $A = B + C$
    \end{tdoclatex}
\end{tdocexa}


% ------------------ %


\begin{tdocexa}[Just the code]
    Via \tdocenv[{[code]}]{tdoclatex}, we'll just get the code as shown below.

    \begin{tdoclatex}[code]
        $A = B + C$
    \end{tdoclatex}
\end{tdocexa}


% ------------------ %


\begin{tdocwarn}
    With default formatting, if the code begins with an opening bracket, the default option must be explicitly indicated.
    \tdocdocbasicinput{examples/listing/strange.tex}
    
    \smallskip
    
    Another method is to use the \tdocmacro{string} primitive.
    \tdocdocbasicinput{examples/listing/strange-bis.tex}
\end{tdocwarn}

\end{document}
