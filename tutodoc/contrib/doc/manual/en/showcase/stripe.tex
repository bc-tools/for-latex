\documentclass[10pt, a4paper]{article}

\usepackage[utf8]{inputenc}
\usepackage[T1]{fontenc}

\usepackage[french]{babel, varioref}

\usepackage{enumitem}
\frenchsetup{StandardItemLabels=true}

\usepackage{tabularray}

\usepackage[lang = french]{tutodoc}



\begin{document}

\section{A real-life rendering} \label{tdoc-showcase}

It is sometimes useful to render code directly in the documentation. This type of rendering must be dissociable from the explanatory text.


% ------------------ %


\subsection{With a coloured stripe}

\begin{tdocexa} [With default text]
    It can be useful to show a real rendering directly in a document
    \footnote{
        Typically when making a demo.
    }.
    This is done via \tdocenv{tdocshowcase} as follows.

    \tdoclatexinput[code]{examples/showcase/default.tex}

    The result is the following rendering
    \footnote{
        Behind the scenes, the strip is created effortlessly using the \tdocpack{clrstrip} package.
    }.

    \medskip

    \begin{bdocshowcase}
    \bfseries Un peu de code \LaTeX.

    \bigskip

    \emph{\large Fin de l'affreuse démo.}
\end{bdocshowcase}
\end{tdocexa}


\begin{tdocrem}
    See the section \ref{tdoc-latexshow} on page \pageref{tdoc-latexshow} to easily obtain code followed by its actual rendering as in the previous example.
\end{tdocrem}


\begin{tdocnote}
    The explanatory texts adapt to the language chosen when \tdocpack{tutodoc} is loaded.
\end{tdocnote}


% ------------------ %


\begin{tdocexa}[Change the default colour and/or text]
    \leavevmode

    \tdoclatexinput[code]{examples/showcase/customized.tex}

    This will produce the following.

    \medskip

    
\begin{bdocshowcase}[before = Mon début,   
                     after  = Ma fin à moi,
                     color  = red]
    Bla, bla, bla, bla, bla, bla, bla, bla, bla, bla, bla, bla, bla...
\end{bdocshowcase}
\end{tdocexa}


\begin{tdocnote}
    You will have noticed that we don't obtain a pure red: behind the scenes, the expandable macros \tdocmacro{tdocbackcolor} and \tdocmacro{tdocdarkcolor} are used to create the background and title colours respectively from the ones proposed in \tdocenv{tdocshowcase}.
    These macros have a single argument, the chosen colour, and accept the following codes.

    \begin{tdoclatex}[code]
\NewExpandableDocumentCommand{\tdocbackcolor}{m}{#1!5}
\NewExpandableDocumentCommand{\tdocdarkcolor}{m}{#1!50!black}
    \end{tdoclatex}
\end{tdocnote}


% ------------------ %


\begin{tdocwarn}
    With the default settings, if the code to be formatted begins with an opening bracket, use \tdocmacro{string} as in the following example.

    \tdoclatexinput[code]{examples/showcase/hook.tex}

    This will produce the following.

    \medskip

    \begin{tdocshowcase}
    \string[Cela fonctionne...]
\end{tdocshowcase}

\end{tdocwarn}


\begin{tdocnote}
    Behind the scenes, the \tdocmacro{tdocruler} macro is used.

    \begin{tdoclatex}[std]
        \tdocruler{Un pseudo-titre décoré}{red}
    \end{tdoclatex}
\end{tdocnote}

\end{document}
