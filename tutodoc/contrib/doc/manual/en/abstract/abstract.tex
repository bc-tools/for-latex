\documentclass[10pt, a4paper]{article}

\usepackage[utf8]{inputenc}
\usepackage[T1]{fontenc}

\usepackage[french]{babel, varioref}

\usepackage{enumitem}
\frenchsetup{StandardItemLabels=true}

\usepackage{tabularray}

\usepackage[lang = french]{tutodoc}



\begin{document}

\title{The \texttt{tutodoc} package - Tutorial-style documentation}
\author{<<AUTHOR>>}
\date{<<DATE-N-VERSION>>}

\maketitle

\begin{abstract}
    The \thispack{} package\,%
    \footnote{
        The name comes from \tdocquote{\tdocprewhy{tuto.rial-type} \tdocprewhy{doc.umentation}}.
    }
    is used by its author to semantically produce documentation of \LaTeX\ packages and classes in a tutorial style,%
    \footnote{
        The idea is to produce an efficient \texttt{PDF} file that can be browsed for one-off needs. This is generally what is expected of coding documentation.
    }
    and with a sober rendering for reading on screen.

    \medskip

    Two important points to note.
    \begin{itemize}
        \item This package imposes a formatting style. In the not-too-distant future, \thispack{} will probably be split into a class and a package.

        \item This documentation is also available in <<OTHER-LANGS>>.
    \end{itemize}
\end{abstract}

\end{document}
