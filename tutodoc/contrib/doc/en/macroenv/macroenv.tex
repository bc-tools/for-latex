\documentclass[10pt, a4paper]{article}

\usepackage[utf8]{inputenc}
\usepackage[T1]{fontenc}

\usepackage[french]{babel, varioref}

\usepackage{enumitem}
\frenchsetup{StandardItemLabels=true}

\usepackage{tabularray}

\usepackage[lang = french]{tutodoc}



\begin{document}

\section{Specify packages, classes, macros or environments}

Here's what you can type semantically.

\begin{tdoclatex}[sbs]
\tdoccls{myclass} is for...

\tdocpack{mypackage} is for...

\tdocmacro{onemacro} is for...

\tdocenv{env} produces...

We also have :

\tdocenv[{[opt1]<opt2>}]{env}
\end{tdoclatex}


\begin{tdocrem}
    The advantage of the previous macros over the use of \tdocmacro{tdocinlatex}, see the section \ref{tdoc-listing-inline} page \pageref{tdoc-listing-inline}, is the absence of colouring.
    Furthermore, the \tdocmacro{tdocenv} macro simply asks you to type the name of the environment
    \footnote{
        In addition, \tdocinlatex{\tdocenv{monenv}} produces \tdocenv{monenv} with spaces to allow line breaks if necessary.
    }
    with any options by typing the correct delimiters
    \footnote{
        Remember that almost anything is possible from now on.
    }
    by hand.
\end{tdocrem}


\begin{tdocwarn}
    The optional argument to the \tdocmacro{tdocenv} macro is copied and pasted during rendering. This can sometimes require the use of protective braces, as in the previous example.
\end{tdocwarn}


% -------------------- %


\section{Origin of a prefix or suffix}

To explain the names chosen, there is nothing like indicating and explaining the short prefixes and suffixes used. This is easily done as follows.

\begin{tdoclatex}[sbs]
\tdocpre{sup} relates to...

\tdocprewhy{sup.erbe} means...

\emph{\tdocprewhy{sup.er} for...}
\end{tdoclatex}


\begin{tdocrem}
    Le choix du point pour scinder un mot permet d'utiliser des mots avec un tiret comme dans \tdocinlatex+\tdocprewhy{ca.sse-brique}+ qui donne \tdocprewhy{ca.sse-brique}.
    The choice of a full stop to split a word allows words with a hyphen to be used, as in \tdocinlatex+\tdocprewhy{bric.k-breaker}+ which gives \tdocprewhy{bric.k-breaker}.
\end{tdocrem}

\end{document}
