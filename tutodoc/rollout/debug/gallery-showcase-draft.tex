\documentclass[10pt, a4paper]{tutodoc}


\begin{document}

{\Huge\bfseries Le thème \texttt{"\thisstyle"}}

\section{Mettre en avant, versionner et dater}

\ExplSyntaxOn

\seq_map_inline:Nn \__g_tutodoc_focus_std_seq {
    \subsection*{tdoc#1}

    \myexrmktext

    \begin{tdoc#1}
        \myhighlightedtext
    \end{tdoc#1}

    \myexrmktext
}

\ifcsundef{__g_tutodoc_focus_color_seq}{
    \prop_map_inline:Nn \__g_tutodoc_focus_color_prop {
        \subsection*{tdoc#1}

        \myadmotext

        \begin{tdoc#1}
            \myhighlightedtext
        \end{tdoc#1}

        \myadmotext
    }
} {
    \seq_map_inline:Nn \__g_tutodoc_focus_color_seq {
        \subsection*{tdoc#1}

        \myadmotext

        \begin{tdoc#1}
            \myhighlightedtext
        \end{tdoc#1}

        \myadmotext
    }
}

\ExplSyntaxOff

\section{Des codes \LaTeX}

Il est indispensable de pouvoir montrer des cas d'utilisation en \LaTeX.

\begin{tdoclatex}
Voir du code \LaTeX\ mis en forme, c'est sympa : $E = m c^2$ ou $\pi \neq \frac{3}{14}$.
\end{tdoclatex}


On dispose aussi d'un mode côte-à-côte moins envahissant. Sympa ! Non ?

\begin{tdoclatex}[sbs]
Voir du code \LaTeX\ mis en forme,
c'est sympa : $E = m c^2$ ou
$\pi \neq \frac{3}{14}$.
\end{tdoclatex}

\end{document}

