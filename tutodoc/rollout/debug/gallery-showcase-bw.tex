\documentclass[theme = bw]{tutodoc}

% -- FORDOC -- %

\newcommand\thisstyle{bw}

\newcommand\myexrmktext{
    \tdocversion{1.7.0}[2024-12-04]
    Dans le flot du texte, il est toujours utile de pouvoir indiquer des exemples et des remarques qui viennent compléter le contenu principal.
}

\newcommand\myadmotext{
    Suivant le contexte d'utilisation, il est parfois nécessaire de pouvoir mettre en avant des contenus en indiquant leur degré d'importance.
}

\newcommand\myhighlightedtext{
    Que dire\,%
    \footnote{
        N'oublions pas les notes de bas de page...
    }?
    Je ne sais pas, mais c'est sympathique. Non ?
}


\begin{document}

\textsf{\Huge\bfseries Le thème \texttt{"\thisstyle"}}


\section{Liens}

{\Large\bfseries \href{https://github.com/bc-tools/for-latex/tree/main/tutodoc}{Un lien très gros}}, mais au moins on le voit.


\section{Mettre en avant, versionner et dater}

\subsection{tdocexa, tdocrem}

\myexrmktext

\ExplSyntaxOn

\seq_map_inline:Nn \g__tutodoc_focus_std_seq {
    \begin{tdoc#1}
        \myhighlightedtext
    \end{tdoc#1}
}

\ExplSyntaxOff

\myexrmktext


\subsection{tdocnote, tdoctip...}

\myadmotext

\ExplSyntaxOn

\ifcsundef{g__tutodoc_focus_color_seq}{
    \prop_map_inline:Nn \g__tutodoc_focus_color_prop {
        \medskip

        \begin{tdoc#1}
            \myhighlightedtext
        \end{tdoc#1}
    }
} {
    \seq_map_inline:Nn \g__tutodoc_focus_color_seq {
        \medskip

        \begin{tdoc#1}
            \myhighlightedtext
        \end{tdoc#1}
    }
}

\ExplSyntaxOff


\subsection{tdocbreak, tdocfix...}

\myexrmktext

\ExplSyntaxOn

\begin{multicols}{2}

\prop_map_inline:Nn \g__tutodoc_topic_change_prop {
    \begin{tdoc#1}
        \item Infos...
    \end{tdoc#1}
}

\vfill\null

\end{multicols}

\ExplSyntaxOff


\section{Des codes \LaTeX}

Taper du code \LaTeX\ en ligne comme \tdocinlatex{E = m c^2 \neq \pi \neq \frac{3}{14}} est utile, tout comme montrer des cas d'utilisation comme le suivant.

\begin{tdoclatex}
Voir du code \LaTeX\ mis en forme, c'est sympa : $E = m c^2$ ou $\pi \neq \frac{3}{14}$.
\end{tdoclatex}


On dispose aussi d'un mode côte-à-côte moins envahissant. Sympa ! Non ?

\begin{tdoclatex}[sbs]
Voir du code \LaTeX\ mis en forme,
c'est sympa : $E = m c^2$ ou
$\pi \neq \frac{3}{14}$.
\end{tdoclatex}

\end{document}

