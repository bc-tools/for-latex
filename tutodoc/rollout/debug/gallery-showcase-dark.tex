\documentclass[theme = dark]{tutodoc}

\newcommand\thisstyle{dark}

\newcommand\myexrmktext{
    In the flow of the text, it is always useful to be able to indicate examples and comments to supplement the main content.
}

\newcommand\myadmotext{
    Depending on the context of use, it is sometimes necessary to be able to highlight content by indicating its degree of importance.
}

\newcommand\myhighlightedtext{
    What to say
    \footnote{
        Let's not forget the footnotes...
    }?
    I don't know, but it's nice. No ?
}



\begin{document}

\textsf{\Huge\bfseries The theme \texttt{"\thisstyle"}}


\section{Links}

{\Large\bfseries \href{https://github.com/bc-tools/for-latex/tree/main/tutodoc}{A very large link}}, but at least you can see it.



\section{Highlight, version and date}

\subsection{tdocexa, tdocrem}

\myexrmktext

\ExplSyntaxOn

\seq_map_inline:Nn \g__tutodoc_focus_std_seq {
    \begin{tdoc#1}
        \myhighlightedtext
    \end{tdoc#1}
}

\ExplSyntaxOff

\myexrmktext


\subsection{tdocnote, tdoctip...}

\myadmotext

\ExplSyntaxOn

\ifcsundef{g__tutodoc_focus_color_seq}{
    \prop_map_inline:Nn \g__tutodoc_focus_color_prop {
        \medskip

        \begin{tdoc#1}
            \myhighlightedtext
        \end{tdoc#1}
    }
} {
    \seq_map_inline:Nn \g__tutodoc_focus_color_seq {
        \medskip

        \begin{tdoc#1}
            \myhighlightedtext
        \end{tdoc#1}
    }
}

\ExplSyntaxOff


\subsection{tdocbreak, tdocfix...}

\myexrmktext

\ExplSyntaxOn

\prop_map_inline:Nn \g__tutodoc_topic_change_prop {
    \begin{tdoc#1}
        \item Infos...
    \end{tdoc#1}
}

\ExplSyntaxOff


\section{LaTeX codes}

Typing an inline code such as \tdocinlatex{E = m c^2 \neq \pi \neq \frac{3}{14}} is useful, as is demonstrating use cases such as the following one.

\begin{tdoclatex}
Seeing some \LaTeX code formatted is nice: $E = m c^2$ or $\pi \neq \frac{3}{14}$.
\end{tdoclatex}


There's also a less intrusive side-by-side mode. Nice! No ?

\begin{tdoclatex}[sbs]
View formatted code,
is nice: $E = m c^2$ or
$\pi \neq \frac{3}{14}$.
\end{tdoclatex}

\end{document}

