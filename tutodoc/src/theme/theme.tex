\documentclass{../main/main}

\usepackage[utf8]{inputenc}
\usepackage[T1]{fontenc}

\usepackage[french]{babel, varioref}

\usepackage{enumitem}
\frenchsetup{StandardItemLabels=true}

\usepackage{tabularray}

\usepackage[lang = french]{tutodoc}


\usepackage{../admonitions/admonitions.cls}
\usepackage{../listing/listing.cls}
\usepackage{../macroenv/macroenv.cls}

\begin{document}

\section{Choisir son thème}

Pour modifier la mise en forme générale, la classe \thisproj\ propose l'option \tdocinlatex{theme = <choix>} où \tdocinlatex{<choix>} peut prendre les valeurs suivantes.

\begin{center}
    \begin{minipage}{.925\linewidth}
        \begin{description}[wide]
            \item[bw]
            Un thème de type noir-et-blanc avec certaines nuances de gris.
        
            \item[color]
            Un thème coloré : c'est \emph{la valeur par défaut}.
        
            \item[dark]
            Un thème sombre idéal pour se reposer les yeux.
        
            \item[draft]
            Un thème pour une impression papier à la recherche d'erreurs de contenu pas forcément simples à débusquer devant un écran.
        \end{description}
    \end{minipage}
\end{center}


\begin{tdocnote}
    A la fin de ce document, après l'historique des changements, vous trouverez une galerie de cas d'utilisation de ces différents thèmes : se rendre à l'annexe page \pageref{tutodoc-theme-gallery}.
\end{tdocnote}

\end{document}
