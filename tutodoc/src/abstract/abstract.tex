\documentclass[12pt, a4paper]{article}

\usepackage[utf8]{inputenc}
\usepackage[T1]{fontenc}

\usepackage[french]{babel, varioref}

\usepackage{enumitem}
\frenchsetup{StandardItemLabels=true}

\usepackage{tabularray}

\usepackage[lang = french]{tutodoc}


\usepackage[lang = french]{../main/main}
\usepackage{../macroenv/macroenv}
\usepackage{../inenglish/inenglish}
\usepackage{../focus/focus}
\usepackage{../deco/deco}


\begin{document}

Le package \tdocpack{tutodoc}
\footnote{
    Le nom vient de \tdocquote{\tdocprewhy{tuto.rial-type} \tdocprewhy{doc.umentation}} se traduit en \tdocquote{documentation de type tutoriel}.
}
est utilisé par son auteur pour produire de façon sémantique des documentations de packages et de classes \LaTeX\ dans un style de type tutoriel
\footnote{
    L'idée est de produire un fichier \texttt{PDF} efficace à parcourir pour des besoins ponctuels. C'est généralement ce que l'on attend d'une documentation liée au codage.
},
et avec un rendu sobre pour une lecture sur écran.

\medskip

Deux points importants à noter.
\begin{itemize}
    \item Ce package impose un style de mise en forme. Dans un avenir plus ou moins proche, \tdocpack{tutodoc} sera sûrement éclaté en une classe et un package.

    \item Cette documentation est aussi disponible en anglais.
\end{itemize}


% ------------------ %


\tdocsep

{\small\itshape
\textbf{Abstract.}
The \tdocpack{tutodoc} package
\footnote{
    The name comes from \tdocquote{\tdocprewhy{tuto.rial-type} \tdocprewhy{doc.umentation}}.
}
is used by its author to semantically produce documentation of \LaTeX\ packages and classes in a tutorial style
\footnote{
    The idea is to produce an efficient \texttt{PDF} file that can be browsed for one-off needs. This is generally what is expected of coding documentation.
},
and with a sober rendering for reading on screen.

\medskip

Two important points to note.
\begin{itemize}
    \item This package imposes a formatting style. In the not-too-distant future, \tdocpack{tutodoc} will probably be split into a class and a package.

    \item This documentation is also available in French.
\end{itemize}
}

\end{document}
