\documentclass[10pt, a4paper]{../main/main}

\usepackage[utf8]{inputenc}
\usepackage[T1]{fontenc}

\usepackage[french]{babel, varioref}

\usepackage{enumitem}
\frenchsetup{StandardItemLabels=true}

\usepackage{tabularray}

\usepackage[lang = french]{tutodoc}


\usepackage{../focus/focus.cls}
\usepackage{../inenglish/inenglish.cls}
\usepackage{../macroenv/macroenv.cls}

%\geometry{showframe}


% == FORDOC == %

% Source: https://tex.stackexchange.com/a/424061/6880

\newcommand{\FTdirO}{}
\def\FTdirO(#1,#2,#3){%
  \FTfile(#1,{\color{blue!40!black}\faFolderOpen\hspace{-.35pt}}{\hspace{0.2em}#3})
  (tmp.west)++(0.8em,-0.4em)node(#2){}
  (tmp.west)++(1.5em,0)
  ++(0,-1.3em) 
}

\newcommand{\FTdirC}{}
\def\FTdirC(#1,#2,#3){%
  \FTfile(#1,{\color{blue!40!black}\faFolder\hspace{.75pt}}{\hspace{0.2em}#3})
  (tmp.west)++(0.8em,-0.4em)node(#2){}
  (tmp.west)++(1.5em,0)
  ++(0,-1.3em) 
}

\newcommand{\FTfile}{}
\def\FTfile(#1,#2){%
  node(tmp){}
  (#1|-tmp)++(0.6em,0)
  node(tmp)[anchor=west,black]{\tt #2}
  (#1)|-(tmp.west)
  ++(0,-1.2em) 
}

\newcommand{\FTroot}{}
\def\FTroot{tmp.west}

\newcommand\contribtranslatedirtree{
  \begin{tikzpicture}%
    \draw[color=black, thick]
% en        : parent = \FTroot
% normal dir: (parentID, currentID, label)
% file      :       (parentID, label)
      \FTdirO(\FTroot,root,translate){
        \FTdirC(root,changes,changes){
        }
        \FTdirO(root,en,en) {    
          \FTdirC(en,api,api) {
          } 
          \FTdirC(en,doc,doc) {
          } 
        } 
        \FTdirC(root,fr,fr){
        }
        \FTdirC(root,status,status){
          \FTdirO(status,en,en) {
            \FTfile(en,api.yaml)
            \FTfile(en,manual.yaml)
          } 
          \FTdirC(status,fr,fr){
          }
        }
        \FTfile(root,README.md)
        \FTfile(root,LICENSE.txt)
      };
  \end{tikzpicture}
}

\begin{document}

%\section{Contribuer}

\subsection{Compléter les traductions}

\begin{tdocnote}
	L'auteur de \thisproj\ se charge des versions françaises et anglaises des traductions.
\end{tdocnote}


\begin{tdoccaut}
	Même si nous allons expliquer comment traduire les documentations, il semble peu pertinent de le faire, car l'anglais devrait suffire de nos jours.%
	\footnote{
		L'existence d'une version anglaise et d'une française vient juste de la langue maternelle de l'auteur de \thisproj, le français.
	}
\end{tdoccaut}


\begin{wrapfigure}{R}%[0pt]
	\contribtranslatedirtree\
	
	\caption{Vue simplifiée du dossier des traductions}
	\label{tdoc-contrib-tranlsate-dir}
\end{wrapfigure}{}


Les traductions sont organisées grosso-modo comme dans la figure \ref{tdoc-contrib-tranlsate-dir} où seuls les dossiers importants pour de nouvelles traductions ont été \tdocquote{ouverts}\,.%
\footnote{
	Cette organisation était celle du 5 octobre 2024, mais elle reste d'actualité.
}



\subsubsection{Les dossiers \texttt{fr} et \texttt{en}}

Ces deux dossiers, d'organisation identique, contiennent des fichiers faciles à traduire même si l'on n'est pas un codeur. Imaginons que vous souhaitiez ajouter le support de l'espagnol. Vous pouvez le faire de deux façons.
%
\begin{enumerate}
	\item Le plus important est de s'intéresser aux dossiers  suivants utiles au fonctionnement de \thisproj.
	\begin{itemize}
		\item \verb#api# pour les traductions relatives aux fonctionnalités de \thisproj
		
		\item \verb#tools# pour les outils automatisant diverses traductions.
	\end{itemize}
	
	Ces deux dossiers ne contiennent que des dossiers avec uniquement des fichiers de type \verb#TXT# à modifier via un éditeur de texte, ou de code, mais non via un traitement de texte.
	Les contenus utilisent des lignes commentées en anglais pour expliquer ce que fera \thisproj\ : ces lignes commencent par \verb#//# comme dans l'extrait suivant.
\end{enumerate}
	
\begin{verbatim}
     // #1: year  in format YYYY like 2023.
     // #2: month in format MM   like 04.
     // #3: day   in format DD   like 29.
     date = #1-#2-#3
\end{verbatim}


\begin{enumerate}[resume]
	\item Le dossier \verb#doc# est réservé aux documentations. Il contient des dossiers avec uniquement des fichiers de type \verb#TEX# compilables directement pour une validation en temps réel des traductions faites.
\end{enumerate}



\subsubsection{Le dossier \texttt{changes}}

Ce dossier est un outil de communication où sont indiqués les changements importants sans s'attarder sur les modifications mineures propres à une ou plusieurs traductions.



\subsubsection{Le dossier \texttt{status}}

Ce dossier permet de savoir où en sont les traductions du point de vue du projet. Tout se passe via des fichiers \verb#YAML# bien commentés.



\subsubsection{Les fichiers \texttt{README.md} et \texttt{LICENSE.txt}}

Le fichier \texttt{LICENSE.txt} est bien nommé, tandis que le fichier \texttt{README.md} reprend en anglais ce qui a été dit dans cette section sur les traductions.

\end{document}
