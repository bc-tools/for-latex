\documentclass[10pt, a4paper]{../main/main}

\usepackage[utf8]{inputenc}
\usepackage[T1]{fontenc}

\usepackage[french]{babel, varioref}

\usepackage{enumitem}
\frenchsetup{StandardItemLabels=true}

\usepackage{tabularray}

\usepackage[lang = french]{tutodoc}


\usepackage{../highlight/highlight.cls}
\usepackage{../listing/listing.cls}
\usepackage{../macroenv/macroenv.cls}

% TESTING LOCAL IMPLEMENTATION %

\usepackage{\jobname.cls}


\begin{document}

\section{Cela veut dire quoi en \tdocquote{anglais}}

Penser aux non-anglophones est bien, même si ces derniers se font de plus en plus rares.


\begin{tdoclatex}
Cool et top signifient \tdocinEN*{cool} et \tdocinEN{top}.
\end{tdoclatex}


La macro \tdocmacro{tdocinEN} et sa version étoilée s'appuient sur \tdocmacro{tdocquote} : par exemple, \tdocquote{sémantique} s'obtient via \tdocinlatex|\tdocquote{sémantique}| .


\begin{tdocnote}
    Le texte \tdocquote{en anglais} est traduit dans la langue détectée par \thisproj.
\end{tdocnote}

\end{document}
