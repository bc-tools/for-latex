\documentclass[10pt, a4paper]{article}

\usepackage[utf8]{inputenc}
\usepackage[T1]{fontenc}

\usepackage[french]{babel, varioref}

\usepackage{enumitem}
\frenchsetup{StandardItemLabels=true}

\usepackage{tabularray}

\usepackage[lang = french]{tutodoc}


\usepackage[lang = french]{../main/main}
\usepackage{../macroenv/macroenv}
\usepackage{../inenglish/inenglish}
\usepackage{../showcase/showcase}
\usepackage{../focus/focus}

% TESTING LOCAL IMPLEMENTATION %

\usepackage{listing}


% == FORDOC == %

% Source.
%    * https://tex.stackexchange.com/a/604698/6880

\NewDocumentCommand{ \tdocdocextraruler }{ m }{%
    \par
    {
        \centering
        \color{green!50!black}%
        \leavevmode
        \kern.075\linewidth
        \leaders\hrule height3.25pt\hfill\kern0pt
        \footnotesize\itshape\bfseries\space\ignorespaces#1\unskip\space
        \leaders\hrule height3.25pt\hfill\kern0pt
        \kern.075\linewidth
        \par
    }
}

\NewDocumentEnvironment{ tdoc-doc-showcase }
                       { O{ Début du rendu dans cette doc. }
                         O{ Fin du rendu dans cette doc. } }{
        \tdocdocextraruler{#1}
        \nopagebreak\smallskip\nopagebreak
}{
        \nopagebreak\smallskip\nopagebreak
        \tdocdocextraruler{#2}
}


\begin{document}

\subsection{Codes importés}

Pour les codes suivants, on considère un fichier de chemin relatif \verb+examples-listing-xyz.tex+, et ayant le contenu suivant.

\tdoclatexinput[code]{examples/listing/xyz.tex}

\medskip

La macro \tdocmacro{tdoclatexinput}, présentée ci-dessous, attend le chemin d'un fichier et propose les mêmes options que l'environnement \tdocenv{tdoclatex}.


% ------------------ %


\begin{tdocexa}[Face à face]
    \leavevmode

    \begin{tdoclatex}[code]
\tdoclatexinput[sbs]{examples/listing/xyz.tex}
    \end{tdoclatex}

    Ceci produit la mise en forme suivante.

    \tdoclatexinput[sbs]{examples/listing/xyz.tex}
\end{tdocexa}


% ------------------ %


\begin{tdocexa}[À la suite]
    \leavevmode

    \begin{tdoclatex}[code]
\tdoclatexinput{examples/listing/xyz.tex}
    \end{tdoclatex}

    Ceci produit la mise en forme suivante où l'option employée par défaut est \tdocinlatex#std#.

    \tdoclatexinput{examples/listing/xyz.tex}
\end{tdocexa}


% ------------------ %


\begin{tdocexa}[Juste le code]
    \leavevmode

    \begin{tdoclatex}[code]
\tdoclatexinput[code]{examples/listing/xyz.tex}
    \end{tdoclatex}

    Ceci produit la mise en forme suivante.

    \tdoclatexinput[code]{examples/listing/xyz.tex}
\end{tdocexa}


% ------------------ %


\subsection{Codes importés et mis en situation} \label{tdoc-latexshow}

\begin{tdocexa}[Showcase]
    Ce qui suit s'obtient via \tdocinlatex+\tdoclatexshow{examples-listing-xyz.tex}+.

    \medskip

    \begin{tdoc-doc-showcase}
        \tdoclatexshow{examples/listing/xyz.tex}
    \end{tdoc-doc-showcase}
\end{tdocexa}


\begin{tdocnote}
    Les textes par défaut tiennent compte de la langue choisie lors du chargement du package \tdocpack{tutodoc}.
\end{tdocnote}


% ------------------ %


\begin{tdocexa}[Changer le texte explicatif]
    Via la clé \tdocinlatex|explain|, on peut utiliser un texte personnalisé. Ainsi, \tdocinlatex|\tdoclatexshow[explain = Voici le rendu réel.]{examples-listing-xyz.tex}| produira ce qui suit.

    \medskip

    \begin{tdoc-doc-showcase}
        \tdoclatexshow[explain = Voici le rendu réel.]{examples/listing/xyz.tex}
    \end{tdoc-doc-showcase}
\end{tdocexa}


% ------------------ %


\begin{tdocexa}[Les options disponibles]
    En plus du texte explicatif, il est aussi possible d'utiliser toutes les options de \tdocenv{tdocshowcase}, voir \ref{tdoc-showcase} page \pageref{tdoc-showcase}.
    Voici un exemple illustrant ceci.

    \medskip

    \tdoclatexinput[code]{examples/listing/latexshow-options.tex}

    \medskip

    Ceci va produire ce qui suit.

    \medskip

    \begin{tdoc-doc-showcase}
        \tdoclatexshow[style      = stripe,
               col-stripe = orange,
               col-text   = blue!70!black,
               before     = Rendu ci-après.,
               explain    = Ce qui vient est coloré...,
               after      = Rendu fini.,]
               {examples/listing-latex/xyz.tex}

    \end{tdoc-doc-showcase}
\end{tdocexa}

\end{document}
