\documentclass[10pt, a4paper]{article}

\usepackage[utf8]{inputenc}
\usepackage[T1]{fontenc}

\usepackage[french]{babel, varioref}

\usepackage{enumitem}
\frenchsetup{StandardItemLabels=true}

\usepackage{tabularray}

\usepackage[lang = french]{tutodoc}


\usepackage[lang = french]{../main/main}
\usepackage{../macroenv/macroenv}
\usepackage{../inenglish/inenglish}
\usepackage{../showcase/showcase}
\usepackage{../focus/focus}

% TESTING LOCAL IMPLEMENTATION %

\usepackage{listing}


\begin{document}

\section{Cas d'utilisation en \LaTeX}

\subsection{Codes \bdocquote{en ligne}} \label{bdoc-listing-inline}

La macro \bdocmacro{bdocinlatex}
\footnote{
	Le nom de la macro \bdocmacro{bdocinlatex} vient de \bdocquote{\bdocprewhy{in.line} \LaTeX} soit \bdocinEN{\LaTeX\ en ligne}.
}
permet de taper du code en ligne via un usage similaire à \bdocmacro{verb}.
Voici des exemples d'utilisation.
\begin{itemize}
    \item \bdocinlatex|$a^b = c$| s'obtient via le code suivant.
		  \begin{center}
		  		\bdocinlatex+\bdocinlatex|$a^b = c$|+
		  \end{center}


    \item \bdocinlatex+\bdocinlatex|$a^b = c$|+ s'obtient via le code suivant.
		  \begin{center}
		  		\bdocinlatex#\bdocinlatex+\bdocinlatex|$a^b = c$|+#
		  \end{center}


    \item \bdocinlatex#\bdocinlatex+\bdocinlatex|$a^b = c$|+# s'obtient via le code suivant.
		  \begin{center}
		  		\bdocinlatex£\bdocinlatex#\bdocinlatex+\bdocinlatex|$a^b = c$|+#£

				\medskip

				... \emph{etc.}
		  \end{center}
\end{itemize}


\begin{bdocnote}
    La macro \bdocmacro{bdocinlatex} est utilisable dans une note de pied de page : voir le bas de cette page
    \footnote{
        \bdocinlatex+$minted = TOP$+ a été tapé \bdocinlatex|\bdocinlatex+$minted = TOP$+| dans cette note de bas de page..
    }.
\end{bdocnote}


% ------------------ %


\subsection{Codes tapés directement}

\begin{bdocexa}[Face à face]
    Dans le code suivant, l'option \bdocinlatex#sbs# est pour \bdocquote{\bdocprewhy{s.ide} \bdocprewhy{b.y} \bdocprewhy{s.ide}} soit \bdocinEN{côte à côte}.

    \bdoclatexshow{examples/listing/ABC.tex}
\end{bdocexa}


% ------------------ %


\begin{bdocexa}[À la suite]
    \bdocenv{bdoclatex} produit le résultat suivant qui correspond à l'option par défaut \bdocinlatex#std#
    \footnote{
        \bdocinlatex{std} fait référence au comportement \bdocquote{standard} de \bdocpack{tcolorbox} vis à vis de la librairie \bdocpack{minted}.
    }.

    \begin{bdoclatex}
        $A = B + C$
    \end{bdoclatex}
\end{bdocexa}


% ------------------ %


\begin{bdocexa}[Juste le code]
    Afficher juste le code comme ci-après s'obtient via l'option \bdocinlatex#code#, donc \bdocenv[{[code]}]{bdoclatex} donnera juste ce qui suit.

    \begin{bdoclatex}[code]
        $A = B + C$
    \end{bdoclatex}
\end{bdocexa}


% ------------------ %


\begin{bdocwarn}
    Avec la mise en forme par défaut, si le code commence par un crochet ouvrant, il faudra indiquer explicitement l'option par défaut. Voici un cas d'usage.

    \bdoclatexshow[explain = Ceci permet d'obtenir ce qui est attendu :]{examples/listing/strange.tex}
\end{bdocwarn}

\end{document}
