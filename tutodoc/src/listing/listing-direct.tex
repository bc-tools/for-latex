\documentclass[10pt, a4paper]{article}

\usepackage[utf8]{inputenc}
\usepackage[T1]{fontenc}

\usepackage[french]{babel, varioref}

\usepackage{enumitem}
\frenchsetup{StandardItemLabels=true}

\usepackage{tabularray}

\usepackage[lang = french]{tutodoc}


\usepackage[lang = french]{../main/main}
\usepackage{../macroenv/macroenv}
\usepackage{../inenglish/inenglish}
\usepackage{../showcase/showcase}
\usepackage{../focus/focus}

% TESTING LOCAL IMPLEMENTATION %

\usepackage{listing}


\begin{document}

\section{Cas d'utilisation en \LaTeX}

\subsection{Codes \tdocquote{en ligne}} \label{tdoc-listing-inline}

La macro \tdocmacro{tdocinlatex}
\footnote{
	Le nom de la macro \tdocmacro{tdocinlatex} vient de \tdocquote{\tdocprewhy{in.line} \LaTeX} soit \tdocinEN{\LaTeX\ en ligne}.
}
permet de taper du code en ligne via un usage similaire à \tdocmacro{verb}.
Voici des exemples d'utilisation.
\begin{itemize}
    \item \tdocinlatex|$a^b = c$| s'obtient via le code suivant.
		  \begin{center}
		  		\tdocinlatex+\tdocinlatex|$a^b = c$|+
		  \end{center}


    \item \tdocinlatex+\tdocinlatex|$a^b = c$|+ s'obtient via le code suivant.
		  \begin{center}
		  		\tdocinlatex#\tdocinlatex+\tdocinlatex|$a^b = c$|+#
		  \end{center}


    \item \tdocinlatex#\tdocinlatex+\tdocinlatex|$a^b = c$|+# s'obtient via le code suivant.
		  \begin{center}
		  		\tdocinlatex£\tdocinlatex#\tdocinlatex+\tdocinlatex|$a^b = c$|+#£

				\medskip

				... \emph{etc.}
		  \end{center}
\end{itemize}


\begin{tdocnote}
    La macro \tdocmacro{tdocinlatex} est utilisable dans une note de pied de page : voir le bas de cette page
    \footnote{
        \tdocinlatex+$minted = TOP$+ a été tapé \tdocinlatex|\tdocinlatex+$minted = TOP$+| dans cette note de bas de page..
    }.
\end{tdocnote}


% ------------------ %


\subsection{Codes tapés directement}

\begin{tdocexa}[Face à face]
    Dans le code suivant, l'option \tdocinlatex#sbs# est pour \tdocquote{\tdocprewhy{s.ide} \tdocprewhy{b.y} \tdocprewhy{s.ide}} soit \tdocinEN{côte à côte}.

    \tdoclatexshow{examples/listing/ABC.tex}
\end{tdocexa}


% ------------------ %


\begin{tdocexa}[À la suite]
    \tdocenv{tdoclatex} produit le résultat suivant qui correspond à l'option par défaut \tdocinlatex#std#
    \footnote{
        \tdocinlatex{std} fait référence au comportement \tdocquote{standard} de \tdocpack{tcolorbox} vis à vis de la librairie \tdocpack{minted}.
    }.

    \begin{tdoclatex}
        $A = B + C$
    \end{tdoclatex}
\end{tdocexa}


% ------------------ %


\begin{tdocexa}[Juste le code]
    Afficher juste le code comme ci-après s'obtient via l'option \tdocinlatex#code#, donc \tdocenv[{[code]}]{tdoclatex} donnera juste ce qui suit.

    \begin{tdoclatex}[code]
        $A = B + C$
    \end{tdoclatex}
\end{tdocexa}


% ------------------ %


\begin{tdocwarn}
    Avec la mise en forme par défaut, si le code commence par un crochet ouvrant, il faudra indiquer explicitement l'option par défaut. Voici un cas d'usage.

    \tdoclatexshow[explain = Ceci permet d'obtenir ce qui est attendu :]{examples/listing/strange.tex}
\end{tdocwarn}

\end{document}
