\documentclass[10pt, a4paper]{../main/main}

\usepackage[utf8]{inputenc}
\usepackage[T1]{fontenc}

\usepackage[french]{babel, varioref}

\usepackage{enumitem}
\frenchsetup{StandardItemLabels=true}

\usepackage{tabularray}

\usepackage[lang = french]{tutodoc}


\usepackage{../highlight/highlight.cls}
\usepackage{../inenglish/inenglish.cls}
\usepackage{../macroenv/macroenv.cls}

% TESTING LOCAL IMPLEMENTATION %

\usepackage{listing.cls}


% == FORDOC == %

% Source.
%    * https://tex.stackexchange.com/a/604698/6880

\NewDocumentCommand{ \tdocdocbasicinput }{ m }{%
    Considérons le code suivant.

    \tdoclatexinput[code]{#1}

    Ceci produira ce qui suit.

    \input{#1}
}


\begin{document}

\section{Cas d'utilisation en \LaTeX}

Documenter un package ou une classe se fait efficacement via des cas d'utilisation montrant à la fois du code et le résultat correspondant.%
\footnote{
	La mise en forme des codes se fait via le package \tdocpack{minted}.
}


%\begin{tdoccaut}
%	La version 3 de \tdocpack{minted} ne peut pas être prise en compte pour le moment car elle comporte des bugs : voir \url{https://github.com/gpoore/minted/issues/401}. On force donc l'usage de la version 2 de \tdocpack{minted}.
%\end{tdoccaut}



\subsection{Codes \tdocquote{en ligne}} \label{tdoc-listing-inline}

La macro \tdocmacro{tdocinlatex}
\footnote{
    Le nom de la macro \tdocmacro{tdocinlatex} vient de \tdocquote{\tdocprewhy{in.line} \LaTeX} soit \tdocinEN{\LaTeX\ en ligne}.
}
permet de taper du code en ligne via un usage similaire à \tdocmacro{verb} ou bien comme une macro standard (voir la gestion des accolades dans le dernier cas ci-dessous).
Voici des exemples d'utilisation.


\begin{tdoclatex}[sbs]
    1: \tdocinlatex|$a^b = c$|               \\
    2: \tdocinlatex+\tdocinlatex|$a^b = c$|+ \\
    3: \tdocinlatex{\tdocinlatex{$a^b = c$}}
\end{tdoclatex}


\begin{tdocnote}
    La macro \tdocmacro{tdocinlatex} est utilisable dans une note de pied de page : voir ci-dessous.
    \footnote{
        \tdocinlatex+$minted = TOP$+ a été tapé \tdocinlatex|\tdocinlatex+$minted = TOP$+| dans cette note de bas de page..
    }
    De plus, une couleur de fond est volontairement utilisée pour subtilement faire ressortir les codes \tdocinlatex#\LaTeX#\,.
\end{tdocnote}



\subsection{Codes tapés directement}

\begin{tdocexa}[Face à face]
    Via \tdocenv[{[sbs]}]{tdoclatex}, on affichera un code et son rendu côte à côte.
    Indiquons que \tdocinlatex#sbs# est pour \tdocquote{\tdocprewhy{s.ide} \tdocprewhy{b.y} \tdocprewhy{s.ide}} soit \tdocinEN{côte à côte}.
    \tdocdocbasicinput{examples/listing/ABC.tex}
\end{tdocexa}


\begin{tdocexa}[À la suite]
    \tdocenv{tdoclatex} produit le résultat suivant qui correspond à l'option par défaut \tdocinlatex#std#\,.
    \footnote{
        \tdocinlatex{std} fait référence au comportement \tdocquote{standard} de \tdocpack{tcolorbox} vis à vis de la librairie \tdocpack{minted}.
    }

    \begin{tdoclatex}
        $A = B + C$
    \end{tdoclatex}
\end{tdocexa}


\begin{tdocexa}[Juste le code]
    Via \tdocenv[{[code]}]{tdoclatex}, on aura juste le code comme ci-après.

    \begin{tdoclatex}[code]
        $A = B + C$
    \end{tdoclatex}
\end{tdocexa}


\begin{tdocwarn}
    Avec la mise en forme par défaut, si le code commence par un crochet ouvrant, il faudra indiquer explicitement l'option par défaut.
    \tdocdocbasicinput{examples/listing/strange.tex}

    \smallskip

    Une autre méthode consiste à utiliser la primitive \tdocmacro{string}.
    \tdocdocbasicinput{examples/listing/strange-bis.tex}
\end{tdocwarn}

\end{document}
