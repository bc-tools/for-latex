\documentclass[10pt, a4paper]{../main/main}

\usepackage[utf8]{inputenc}
\usepackage[T1]{fontenc}

\usepackage[french]{babel, varioref}

\usepackage{enumitem}
\frenchsetup{StandardItemLabels=true}

\usepackage{tabularray}

\usepackage[lang = french]{tutodoc}


\usepackage{../admonitions/admonitions.cls}
\usepackage{../listing/listing.cls}
\usepackage{../macroenv/macroenv.cls}

% TESTING LOCAL IMPLEMENTATION %

\usepackage{showcase.cls}


\begin{document}

\section{Un rendu en situation réelle}
\label{tutodoc-showcase}

Il est parfois utile d'obtenir directement le rendu d'un code dans la documentation. Ceci nécessite que ce type de rendu soit dissociable du texte donnant des explications.



\subsection{Avec une bande colorée}
\label{tutodoc-color-macros}

\begin{tdocexa}[Avec les textes par défaut]
    Il peut être utile de montrer un rendu réel directement dans un document.
    \footnote{
        Typiquement lorsque l'on fait une démo.
    }
    Ceci se tape via \tdocenv{tdocshowcase} comme suit.

    \tdoclatexinput[code]{examples/showcase/default.tex}

    On obtient alors le rendu suivant.
    \footnote{
        En coulisse, la bande est créée sans effort grâce au package \tdocpack{clrstrip}.
    }

    \medskip

    \begin{bdocshowcase}
    \bfseries Un peu de code \LaTeX.

    \bigskip

    \emph{\large Fin de l'affreuse démo.}
\end{bdocshowcase}
\end{tdocexa}


\begin{tdocrem}
    Voir la section \ref{tutodoc-latexshow} page \pageref{tutodoc-latexshow} pour obtenir facilement un code suivi de son rendu réel comme dans l'exemple précédent.
\end{tdocrem}


\begin{tdocnote}
    Les textes explicatifs s'adaptent à la langue détectée par \thisproj.
\end{tdocnote}


\begin{tdocexa}[Changer les couleurs et/ou les textes par défaut]
    \leavevmode

    \tdoclatexinput[code]{examples/showcase/customized.tex}

    Ceci produira ce qui suit.

    \medskip

    
\begin{bdocshowcase}[before = Mon début,   
                     after  = Ma fin à moi,
                     color  = red]
    Bla, bla, bla, bla, bla, bla, bla, bla, bla, bla, bla, bla, bla...
\end{bdocshowcase}
\end{tdocexa}


\begin{tdocnote}
    Dans l'exemple précédent, le texte utilise bien l'orange assombri proposé. Par contre, le rouge sert de base pour obtenir les couleurs utilisées pour la bande : les transformations utilisées dépendent du thème choisi.%
    \footnote{
        Par exemple, les thèmes \tdocinlatex{bw} et \tdocinlatex{draft} ne tiennent pas compte de la clé \tdocinlatex{colstripe} !
    }
    %
    Il faut également savoir qu'en coulisse, la macro \tdocmacro{tdocruler} est employée.

    \begin{tdoclatex}[std]
        \tdocruler[red]{Un pseudo-titre décoré}
    \end{tdoclatex}
\end{tdocnote}


\begin{tdocwarn}
    Avec les réglages par défaut, si le code \LaTeX\ à mettre en forme commence par un crochet ouvrant, il faudra user de \tdocmacro{string} comme dans l'exemple suivant.

    \tdoclatexinput[code]{examples/showcase/hook.tex}

    Ceci produira ce qui suit.
\end{tdocwarn}

\begin{tdocshowcase}
    \string[Cela fonctionne...]
\end{tdocshowcase}


\end{document}
