\documentclass[10pt, a4paper]{article}

\usepackage[utf8]{inputenc}
\usepackage[T1]{fontenc}

\usepackage[french]{babel, varioref}

\usepackage{enumitem}
\frenchsetup{StandardItemLabels=true}

\usepackage{tabularray}

\usepackage[lang = french]{tutodoc}


\usepackage[lang = french]{../main/main}
\usepackage{../macroenv/macroenv}
\usepackage{../listing/listing}
\usepackage{../focus/focus}

% TESTING LOCAL IMPLEMENTATION %

\usepackage{showcase}


\begin{document}

%\section{Un rendu en situation réelle} \label{tdoc-showcase}

\subsection{Sans bande colorée}

Le rendu de \tdocenv{tdocshowcase} avec une bande colorée peut ne pas convenir, ou parfois ne pas être acceptable malgré le travail fait par \tdocpack{clrstrip}.
Il est possible de ne pas utiliser une bande colorée comme nous allons le voir tout de suite.

\begin{tdocexa}
    L'emploi de \tdocenv[{[nostripe]}]{tdocshowcase} demande de ne pas faire appel à \tdocpack{clrstrip}.
    Voici un exemple d'utilisation.

    \tdoclatexinput[code]{examples/showcase/no-clrstrip.tex}

    Ceci produira ce qui suit.

    \medskip

    
\begin{bdocshowcase}[nostripe]
    Bla, bla, bla, bla, bla, bla, bla, bla, bla, bla, bla, bla, bla...
\end{bdocshowcase}
\end{tdocexa}


% ------------------ %


\begin{tdocexa}[Changer la couleur et/ou les textes par défaut]
    \leavevmode

    \tdoclatexinput[code]{examples/showcase/no-clrstrip-customized.tex}

    Ceci produira ce qui suit.

    \medskip

    \begin{tdocshowcase}[nostripe,
                     before = My beginning,
                     after  = My end,
                     colstripe = green,
                     coltext   = purple]
    Bla, bla, bla, bla, bla, bla, bla, bla, bla, bla, bla, bla, bla...
\end{tdocshowcase}

\end{tdocexa}

\end{document}
