\documentclass{../main/main}

\usepackage[utf8]{inputenc}
\usepackage[T1]{fontenc}

\usepackage[french]{babel, varioref}

\usepackage{enumitem}
\frenchsetup{StandardItemLabels=true}

\usepackage{tabularray}

\usepackage[lang = french]{tutodoc}


\usepackage{../admonitions/admonitions.cls}
\usepackage{../listing-latex/listing-latex.cls}
\usepackage{../macroenv/macroenv.cls}

% TESTING LOCAL IMPLEMENTATION %

\usepackage{showcase.cls}


\begin{document}

\subsection{Avec des lignes cadrantes}

Pour rendre plus visible le code \LaTeX\ mis en forme, on peut faire appel au style \tdoclatexin{rule} comme dans les exemple suivants.


\begin{tdocexa}
	L'option \tdoclatexin{style = rule} permet d'obtenir ce qui suit ce qui suit où les textes ajoutés automatiquement s'adapteront à la langue repéré par \thisproj.

	\documentclass{tutodoc}

\usepackage[utf8]{inputenc}
\usepackage[T1]{fontenc}

\usepackage[french]{babel, varioref}

\usepackage{enumitem}
\frenchsetup{StandardItemLabels=true}

\usepackage{tabularray}

\usepackage[lang = french]{tutodoc}



\begin{document}

\subsection{Avec des lignes cadrantes}

Pour rendre plus visible le code \LaTeX\ mis en forme de façon sombre, on peut faire appel au style \tdoclatexin{rule} comme dans les exemple suivants.


\begin{tdocexa}
	L'option \tdoclatexin{style=rule} permet d'obtenir ce qui suit ce qui suit où les textes ajoutés automatiquement s'adapteront à la langue repéré par \thisproj.

	\documentclass{tutodoc}

\usepackage[utf8]{inputenc}
\usepackage[T1]{fontenc}

\usepackage[french]{babel, varioref}

\usepackage{enumitem}
\frenchsetup{StandardItemLabels=true}

\usepackage{tabularray}

\usepackage[lang = french]{tutodoc}



\begin{document}

\subsection{Avec des lignes cadrantes}

Pour rendre plus visible le code \LaTeX\ mis en forme de façon sombre, on peut faire appel au style \tdoclatexin{rule} comme dans les exemple suivants.


\begin{tdocexa}
	L'option \tdoclatexin{style=rule} permet d'obtenir ce qui suit ce qui suit où les textes ajoutés automatiquement s'adapteront à la langue repéré par \thisproj.

	\documentclass{tutodoc}

\input{../preamble.cfg.tex}


\begin{document}

\subsection{Avec des lignes cadrantes}

Pour rendre plus visible le code \LaTeX\ mis en forme de façon sombre, on peut faire appel au style \tdoclatexin{rule} comme dans les exemple suivants.


\begin{tdocexa}
	L'option \tdoclatexin{style=rule} permet d'obtenir ce qui suit ce qui suit où les textes ajoutés automatiquement s'adapteront à la langue repéré par \thisproj.

	\input{examples/showcase/rule.tex}
\end{tdocexa}


\begin{tdocexa}[Du texte et des couleurs modifiables]
	On peut obtenir facilement l'horreur suivante.

	\input{examples/showcase/rule-custom.tex}
	
	Voici le code qui a été employé.%
	\footnote{
		La section suivante va normaliser le choix, a priori étrange, de \tdoclatexin{col-stripe} au lieu de \tdoclatexin{col-rule}\,.
	}

	\tdoclatexinput<\tdoctcb{code}>{examples/showcase/rule-custom.tex}
\end{tdocexa}


\begin{tdocnote}
    Dans l'exemple précédent, le texte utilise bien l'orange assombri proposé. Par contre, le rouge sert de base pour obtenir les couleurs utilisées pour la bande : les transformations utilisées dépendent du thème choisi.%
    \footnote{
        Par exemple, les thèmes \tdoclatexin{bw} et \tdoclatexin{draft} ne tiennent pas compte de la clé \tdoclatexin{col-stripe} !
    }
    %
    Il faut également savoir qu'en coulisse, la macro \tdocmacro{tdocruler} est employée.

    \begin{tdoclatex}<\tdoctcb{std}>
\tdocruler[red]{Un pseudo-titre décoré}
    \end{tdoclatex}
\end{tdocnote}

\end{document}

\end{tdocexa}


\begin{tdocexa}[Du texte et des couleurs modifiables]
	On peut obtenir facilement l'horreur suivante.

	\begin{tdocshowcase}[style      = rule,
                     col-stripe = red,
                     col-text   = orange!75!black,
                     before     = Mon début,
                     after      = Ma fin à moi]
    Bla, bla, bla, bla, bla, bla, bla, bla, bla, bla, bla, bla, bla...
\end{tdocshowcase}

	
	Voici le code qui a été employé.%
	\footnote{
		La section suivante va normaliser le choix, a priori étrange, de \tdoclatexin{col-stripe} au lieu de \tdoclatexin{col-rule}\,.
	}

	\tdoclatexinput<\tdoctcb{code}>{examples/showcase/rule-custom.tex}
\end{tdocexa}


\begin{tdocnote}
    Dans l'exemple précédent, le texte utilise bien l'orange assombri proposé. Par contre, le rouge sert de base pour obtenir les couleurs utilisées pour la bande : les transformations utilisées dépendent du thème choisi.%
    \footnote{
        Par exemple, les thèmes \tdoclatexin{bw} et \tdoclatexin{draft} ne tiennent pas compte de la clé \tdoclatexin{col-stripe} !
    }
    %
    Il faut également savoir qu'en coulisse, la macro \tdocmacro{tdocruler} est employée.

    \begin{tdoclatex}<\tdoctcb{std}>
\tdocruler[red]{Un pseudo-titre décoré}
    \end{tdoclatex}
\end{tdocnote}

\end{document}

\end{tdocexa}


\begin{tdocexa}[Du texte et des couleurs modifiables]
	On peut obtenir facilement l'horreur suivante.

	\begin{tdocshowcase}[style      = rule,
                     col-stripe = red,
                     col-text   = orange!75!black,
                     before     = Mon début,
                     after      = Ma fin à moi]
    Bla, bla, bla, bla, bla, bla, bla, bla, bla, bla, bla, bla, bla...
\end{tdocshowcase}

	
	Voici le code qui a été employé.%
	\footnote{
		La section suivante va normaliser le choix, a priori étrange, de \tdoclatexin{col-stripe} au lieu de \tdoclatexin{col-rule}\,.
	}

	\tdoclatexinput<\tdoctcb{code}>{examples/showcase/rule-custom.tex}
\end{tdocexa}


\begin{tdocnote}
    Dans l'exemple précédent, le texte utilise bien l'orange assombri proposé. Par contre, le rouge sert de base pour obtenir les couleurs utilisées pour la bande : les transformations utilisées dépendent du thème choisi.%
    \footnote{
        Par exemple, les thèmes \tdoclatexin{bw} et \tdoclatexin{draft} ne tiennent pas compte de la clé \tdoclatexin{col-stripe} !
    }
    %
    Il faut également savoir qu'en coulisse, la macro \tdocmacro{tdocruler} est employée.

    \begin{tdoclatex}<\tdoctcb{std}>
\tdocruler[red]{Un pseudo-titre décoré}
    \end{tdoclatex}
\end{tdocnote}

\end{document}

\end{tdocexa}


\begin{tdocexa}[Du texte et des couleurs modifiables]
	On peut obtenir facilement l'horreur suivante.

	\begin{tdocshowcase}[style      = rule,
                     col-stripe = red,
                     col-text   = orange!75!black,
                     before     = Mon début,
                     after      = Ma fin à moi]
    Bla, bla, bla, bla, bla, bla, bla, bla, bla, bla, bla, bla, bla...
\end{tdocshowcase}


	Voici le code qui a été employé.%
	\footnote{
		La section suivante va justifier le choix, a priori étrange, de \tdoclatexin{col-stripe} au lieu de \tdoclatexin{col-rule}\,.
	}

	\tdoclatexinput<\tdoctcb{code}>{examples/showcase/rule-custom.tex}
\end{tdocexa}


\begin{tdocnote}
    Dans l'exemple précédent, le texte utilise bien l'orange assombri proposé. Par contre, le rouge sert de base pour obtenir les couleurs utilisées pour les lignes cadrantes: les transformations utilisées dépendent du thème choisi.%
    \footnote{
        Par exemple, les thèmes \tdoclatexin{bw} et \tdoclatexin{draft} ne tiennent pas compte de la clé \tdoclatexin{col-stripe} !
    }
    %
    Il faut également savoir qu'en coulisse, la macro \tdocmacro{tdocruler} est employée, elle fonctionne comme suit.

    \begin{tdoclatex}<\tdoctcb{std}>
\tdocruler[red]{Un pseudo-titre décoré}
    \end{tdoclatex}
\end{tdocnote}

\end{document}
