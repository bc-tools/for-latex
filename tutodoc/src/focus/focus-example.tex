\documentclass[12pt, a4paper]{article}

\usepackage[utf8]{inputenc}
\usepackage[T1]{fontenc}

\usepackage[french]{babel, varioref}

\usepackage{enumitem}
\frenchsetup{StandardItemLabels=true}

\usepackage{tabularray}

\usepackage[lang = french]{tutodoc}


\usepackage[lang = french]{../main/main}
\usepackage{../macroenv/macroenv}
\usepackage{../showcase/showcase}
\usepackage{../listing/listing}
\usepackage{../inenglish/inenglish}

% TESTING LOCAL IMPLEMENTATION %

\usepackage{focus}


\begin{document}

\section{Mettre en avant du contenu}

\begin{tdocnote}
    Les environnements présentés dans cette section
    \footnote{
        La mise en forme provient du package \tdocpack{amsthm}.
    }
    ajoutent un court titre indiquant le type d'informations fournies.
    Ce court texte sera toujours traduit dans la langue indiquée lors du chargement du package \thispack\.
\end{tdocnote}



\subsection{Des exemples}

Des exemples numérotés, ou non, s'indiquent via l'environnement \tdocenv{tdocexa} qui propose deux arguments optionnels.

\begin{enumerate}
    \item Le 1\ier{} argument entre chevrons \tdocinlatex#<...># peut prendre au choix les valeurs \tdocinlatex#nb# pour numéroter, réglage par défaut, et \tdocinlatex#nonb# pour ne pas numéroter.

    \item Le 2\ieme{} argument entre crochets \tdocinlatex#[...]# sert à ajouter un mini-titre.
\end{enumerate}


Voici différents emplois possibles.

\tdoclatexinput[sbs]{examples/focus/exa.tex}


% ------------------ %


\begin{tdocimportant}
    La numérotation des exemples est remise à zéro dès qu'une section  de niveau au moins égale à une \tdocinlatex|\subsubsection| est ouverte.
\end{tdocimportant}


% ------------------ %


\begin{tdoctip}
    Il peut parfois être utile de revenir à la ligne dès le début du contenu. Voici comment faire (ce tour de passe-passe reste valable pour les environnements présentés dans les sous-sections suivantes). Noter au passage que la numérotation suit celle de l'exemple précédent comme souhaité.

    \tdoclatexinput[sbs]{examples/focus/exa-leavevmode.tex}
\end{tdoctip}

\end{document}
