\documentclass[10pt, a4paper]{../main/main}

\usepackage[utf8]{inputenc}
\usepackage[T1]{fontenc}

\usepackage[french]{babel, varioref}

\usepackage{enumitem}
\frenchsetup{StandardItemLabels=true}

\usepackage{tabularray}

\usepackage[lang = french]{tutodoc}


\usepackage{../macroenv/macroenv.cls}
\usepackage{../listing/listing.cls}

% TESTING LOCAL IMPLEMENTATION %

\usepackage{focus.cls}


\begin{document}

\subsection{Du contenu tape-à-l'oeil} \label{tdoc-colorful-focus}

\begin{tdocnote}
    Les icônes sont obtenues via le package \tdocpack{fontawesome5}, et la gestion de l'espacement avec le texte est faite par la macro \tdocmacro{tdocicon}.
    \footnote{
        Par exemple,
        \tdocinlatex|\tdocicon{\faBed}{Fatigué}|
        produit\,
        \tdocicon{\faBed}{Fatigué}.
    }
\end{tdocnote}


\subsubsection{Une astuce}

L'environnement \tdocenv*{tdoctip} sert à donner des astuces. Voici comment l'employer.

\tdoclatexinput[sbs]{examples/focus/tip.tex}


\smallskip

\begin{tdocnote}
    Les couleurs sont obtenues via les macros développables \tdocmacro{tdocbackcolor} et \tdocmacro{tdocdarkcolor}.
    Pour des informations complémentaires à ce sujet, se reporter à la fin de la section \ref{tdoc-color-macros} page \pageref{tdoc-color-macros}.
\end{tdocnote}

\end{document}
