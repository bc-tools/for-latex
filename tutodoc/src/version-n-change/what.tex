\documentclass[10pt, a4paper]{../main/main}

\usepackage[utf8]{inputenc}
\usepackage[T1]{fontenc}

\usepackage[french]{babel, varioref}

\usepackage{enumitem}
\frenchsetup{StandardItemLabels=true}

\usepackage{tabularray}

\usepackage[lang = french]{tutodoc}


\usepackage{../admonitions/admonitions.cls}
\usepackage{../listing/listing.cls}
\usepackage{../macroenv/macroenv.cls}


% TESTING LOCAL IMPLEMENTATION %

\usepackage{version-n-change.cls}


\begin{document}

\subsection{Quoi de neuf ?}

\thisproj\ propose la macro \tdocmacro{tdocstartproj} et différents environnements pour indiquer rapidement et clairement ce qui a été fait lors des derniers changements.%
\footnote{
    L'utilisateur n'a pas besoin de tous les détails techniques.
}


\begin{tdocnote}
    Concernant les icônes, voir la note au début de la section \ref{tutodoc-admonitions} page \pageref{tutodoc-admonitions}.
\end{tdocnote}


\foreach \exatitle/\filename in {
    {Juste pour la toute première version}/%<-- Translate me!
    	first,
    {Pour les nouveautés}/% <-- Translate me!
    	new,
    {Pour les mises à jour}/% <-- Translate me!
    	update,
    {Pour les bifurcations}/% <-- Translate me!
    	break,
    {Pour les problèmes}/% <-- Translate me!
    	pb,
    {Pour les réparations}/% <-- Translate me!
    	fix,
    {Informations techniques}/% <-- Translate me!
    	tech,
    %
    {Thématiques aux choix avec une icône}/%<-- Translate me!
    	user-choice-icon,
    {Thématiques aux choix sans icône}/% <-- Translate me!
    	user-choice%
} {
    \begin{tdocexa}[\exatitle]
        \leavevmode

        \tdoclatexinput[sbs]{examples/version-n-change/\filename.tex}
    \end{tdocexa}
}

\end{document}
