\documentclass[10pt, a4paper]{article}

\usepackage[utf8]{inputenc}
\usepackage[T1]{fontenc}

\usepackage[french]{babel, varioref}

\usepackage{enumitem}
\frenchsetup{StandardItemLabels=true}

\usepackage{tabularray}

\usepackage[lang = french]{tutodoc}


\usepackage[lang = french]{../main/main}
\usepackage{../macroenv/macroenv}
\usepackage{../inenglish/inenglish}
\usepackage{../showcase/showcase}
\usepackage{../listing/listing}
\usepackage{../focus/focus}

% TESTING LOCAL IMPLEMENTATION %

\usepackage{version-n-change}


\begin{document}

\subsection{Quoi de neuf ?}

\thispack{} propose différents environnements pour indiquer rapidement et clairement ce qui a été fait
\footnote{
    L'utilisateur n'a pas besoin de tous les détails techniques.
}
lors des derniers changements.


\begin{tdocexa}[Pour les nouveautés]
    \leavevmode

    \tdoclatexinput[sbs]{examples/version-n-change/new.tex}
\end{tdocexa}


% ------------------ %


\begin{tdocexa}[Pour les mises à jour]
    \leavevmode

    \tdoclatexinput[sbs]{examples/version-n-change/update.tex}
\end{tdocexa}


% ------------------ %


\begin{tdocexa}[Pour les bifurcations]
    \leavevmode

    \tdoclatexinput[sbs]{examples/version-n-change/break.tex}
\end{tdocexa}


% ------------------ %


\begin{tdocexa}[Pour les problèmes]
    \leavevmode

    \tdoclatexinput[sbs]{examples/version-n-change/pb.tex}
\end{tdocexa}


% ------------------ %


\begin{tdocexa}[Pour les réparations]
    \leavevmode

    \tdoclatexinput[sbs]{examples/version-n-change/fix.tex}
\end{tdocexa}


% ------------------ %


\begin{tdocexa}[Thématiques aux choix]
    \leavevmode

    \tdoclatexinput[sbs]{examples/version-n-change/user-choice.tex}
\end{tdocexa}

\end{document}
