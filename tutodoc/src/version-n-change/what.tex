\documentclass[10pt, a4paper]{article}

\usepackage[utf8]{inputenc}
\usepackage[T1]{fontenc}

\usepackage[french]{babel, varioref}

\usepackage{enumitem}
\frenchsetup{StandardItemLabels=true}

\usepackage{tabularray}

\usepackage[lang = french]{tutodoc}


\usepackage[lang = french]{../main/main}
\usepackage{../macroenv/macroenv}
\usepackage{../inenglish/inenglish}
\usepackage{../showcase/showcase}
\usepackage{../listing/listing}
\usepackage{../focus/focus}


% TESTING LOCAL IMPLEMENTATION %

\usepackage{version-n-change}


\begin{document}

\subsection{Quoi de neuf ?} 

\thispack{} propose la macro \tdocmacro{tdocstartproj} et différents environnements pour indiquer rapidement et clairement ce qui a été fait lors des derniers changements.%
\footnote{
    L'utilisateur n'a pas besoin de tous les détails techniques.
}


\begin{tdocnote}
    Les icônes sont obtenues via le package \tdocpack{fontawesome5}, et la gestion de l'espacement avec le texte est faite par la macro \tdocmacro{tdocicon}.
\end{tdocnote}


% ------------------ %


\begin{tdocexa}[Juste pour la toute première version]
    \leavevmode

    \tdoclatexinput[sbs]{examples/version-n-change/first.tex}
\end{tdocexa}


% ------------------ %


\begin{tdocexa}[Pour les nouveautés]
    \leavevmode

    \tdoclatexinput[sbs]{examples/version-n-change/new.tex}
\end{tdocexa}


% ------------------ %


\begin{tdocexa}[Pour les mises à jour]
    \leavevmode

    \tdoclatexinput[sbs]{examples/version-n-change/update.tex}
\end{tdocexa}


% ------------------ %


\begin{tdocexa}[Pour les bifurcations]
    \leavevmode

    \tdoclatexinput[sbs]{examples/version-n-change/break.tex}
\end{tdocexa}


% ------------------ %


\begin{tdocexa}[Pour les problèmes]
    \leavevmode

    \tdoclatexinput[sbs]{examples/version-n-change/pb.tex}
\end{tdocexa}


% ------------------ %


\begin{tdocexa}[Pour les réparations]
    \leavevmode

    \tdoclatexinput[sbs]{examples/version-n-change/fix.tex}
\end{tdocexa}


% ------------------ %


\begin{tdocexa}[Thématiques aux choix avec une icône]
    \leavevmode

    \tdoclatexinput[sbs]{examples/version-n-change/user-choice-icon.tex}
\end{tdocexa}


% ------------------ %


\begin{tdocexa}[Thématiques aux choix sans icône]
    \leavevmode

    \tdoclatexinput[sbs]{examples/version-n-change/user-choice.tex}
\end{tdocexa}


% ------------------ %


\begin{tdocnote}
    L'utilisation d'une icône suivie du bon espacement juste avant le texte se fait via la macro \tdocmacro{tdocicon}.
    Par exemple,
    \tdocinlatex|\tdocicon{\faBed}{Fatigué}|
    produit\,
    \tdocicon{\faBed}{Fatigué}.
\end{tdocnote}

\end{document}
