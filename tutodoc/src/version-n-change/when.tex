\documentclass[10pt, a4paper]{../main/main}

\usepackage[utf8]{inputenc}
\usepackage[T1]{fontenc}

\usepackage[french]{babel, varioref}

\usepackage{enumitem}
\frenchsetup{StandardItemLabels=true}

\usepackage{tabularray}

\usepackage[lang = french]{tutodoc}


\usepackage{../admonitions/admonitions.cls}
\usepackage{../listing/listing.cls}
\usepackage{../macroenv/macroenv.cls}
\usepackage{../showcase/showcase.cls}


% TESTING LOCAL IMPLEMENTATION %

\usepackage{version-n-change.cls}


\begin{document}

\section{Indiquer les changements}

Afin de faciliter le suivi d'un projet, il est indispensable de fournir un historique indiquant les changements effectués lors de la publication d'une nouvelle version.



\subsection{À quel moment ?}
\label{tutodoc-changes-when}

On peut au choix dater quelque chose, ou bien le versionner, dans ce second cas le numéro de version pourra éventuellement être daté.


\begin{tdocexa}[Dater des nouveautés]
    La macro \tdocmacro{tdocdate} permet d'indiquer une date dans la marge comme dans l'exemple suivant.

    \tdoclatexshow{examples/version-n-change/dating.tex}
\end{tdocexa}


\begin{tdocexa}[Versionner des nouveautés en les datant événtuellement]
    Associer un numéro de version à une nouveauté se fait via la macro \tdocmacro{tdocversion}, la couleur et la date étant des arguments optionnels.

    \tdoclatexshow{examples/version-n-change/versioning.tex}
\end{tdocexa}


\begin{tdocimp}
    \begin{enumerate}[wide]
        \item Les macros \tdocmacro{tdocdate} et \tdocmacro{tdocversion} nécessitent deux compilations.

        \item Comme la langue détectée pour cette documentation est le français, la date dans le rendu final est au format \texttt{JJ/MM/AAAA} alors que dans le code celle-ci devra toujours être saisie au format anglais \texttt{AAAA-MM-JJ}.
    \end{enumerate}
\end{tdocimp}


\begin{tdoccaut}
    Seul l'emploi du format numérique \tdocinlatex+YYYY-MM-DD+ est vérifié,
    \footnote{
        Techniquement, vérifier la validité d'une date, via \LaTeX3, ne présente pas de difficulté.
    }
    et ceci est un choix ! Pourquoi cela ? Tout simplement car dater et versionner des explications devrait se faire de façon semi-automatisée afin d'éviter tout bug humain.
%    \footnote{
%        L'auteur de \thispack{} est entrain de mettre en place un ensemble d'outils permettant une telle semi-automatisation.
%    }.
\end{tdoccaut}


\begin{tdocwarn}
    En coulisse, le gros du travail est fait par la macro \tdocmacro{\marginnote} du package éponyme.
    Les macros \tdocmacro{tdocdate} et \tdocmacro{tdocversion} souffrent donc des même effets de bord comme dans l'exemple suivant.
    On peut tout de même régler ces problèmes via un argument optionnel \tdocinlatex{<déplacement-vertical>} comme à la fin de l'exemple ci-après. C'est moche, mais fonctionnel.

	\tdoclatexshow[nostripe, colstripe=red]{examples/version-n-change/side-effect.tex}
\end{tdocwarn}

\end{document}
