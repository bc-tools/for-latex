\documentclass{../main/main}

\usepackage[utf8]{inputenc}
\usepackage[T1]{fontenc}

\usepackage[french]{babel, varioref}

\usepackage{enumitem}
\frenchsetup{StandardItemLabels=true}

\usepackage{tabularray}

\usepackage[lang = french]{tutodoc}


\usepackage{../listing-latex/listing-latex.cls}
\usepackage{../macroenv/macroenv.cls}

% TESTING LOCAL IMPLEMENTATION %

\usepackage{admonitions.cls}


\begin{document}

\section{Mettre en avant du contenu}

\begin{tdocnote}
    Les environnements présentés dans cette section\,%
    \footnote{
        La mise en forme provient du package \tdocpack{keytheorems}.
    }
    ajoutent un court titre indiquant le type d'informations fournies.
    Ce court texte sera toujours traduit dans la langue repérée par la classe \thisproj.
\end{tdocnote}



\subsection{Du contenu dans le flot de la lecture}

\begin{tdocimp}
    Tous les environnements présentés dans cette section partagent le même compteur qui sera remis à zéro dès qu'une section de niveau au moins égale à une \tdoclatexin|\section| sera ouverte.
\end{tdocimp}



\subsubsection{Des exemples}

Des exemples numérotés, si besoin, s'indiquent via \tdocenv{tdocexa} qui propose un argument optionnel pour ajouter un mini-titre.
Voici deux usages possibles.

\tdoclatexinput<\tdoctcb{sbs}>{examples/admonitions/exa.tex}


\begin{tdoctip}
    Il peut parfois être utile de revenir à la ligne dès le début du contenu. Le code suivant montre comment faire (ce tour de passe-passe reste valable pour l'environnement \verb#tdocrem# présenté juste après). Noter au passage que la numérotation suit celle de l'exemple précédent comme souhaité.
\end{tdoctip}

\tdoclatexinput<\tdoctcb{sbs}>{examples/admonitions/exa-leavevmode.tex}

\end{document}
