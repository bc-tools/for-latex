\documentclass{../main/main}

\usepackage[utf8]{inputenc}
\usepackage[T1]{fontenc}

\usepackage[french]{babel, varioref}
\frenchsetup{StandardItemLabels=true}

\usepackage{enumitem}

\usepackage{multicol}

\newcommand\thispack{\tdocpack{tutodoc}}


\usepackage{../admonitions/admonitions.cls}
\usepackage{../inenglish/inenglish.cls}
\usepackage{../macroenv/macroenv.cls}
\usepackage{../listing-latex/listing-latex.cls}

% TESTING LOCAL IMPLEMENTATION %

\usepackage{listing-full.cls}


\begin{document}

\section{Présenter du code informatique}

Certains packages proposent des outils utilisables via le langage \lua\ depuis un document \LaTeX.%
\footnote{
	Pour les mathématiques, on peut citer \tdocpack{luacas} et \tdocpack{tkz-elements}.
}
Pour ce type de projet, il est utile de pouvoir présenter des lignes de code \lua\ ; \thisproj\ permet de faire cela aisément, et bien plus.%
\footnote{
    La mise en forme des codes étant faite via les packages \tdocpack{minted} et \tdocpack{tcolorbox}, les macros et les environnements présentés dans cette section permettent la mise en forme de codes dans tous les langages supportés par \pygmentsREF, un projet \python\ utilisé en coulisse par \tdocpack{minted}.
}


\begin{tdoccaut}
	Les outils de cette section permettent aussi de présenter du code \LaTeX, mais il ne faut pas les utiliser pour de simples cas d'utilisation.
	Les macros et les environnements présentées juste après servent à étudier du code du point de vue du codeur, et non de celui d'un utilisateur standard : se reporter à la section \ref{tutodoc-listing-latex} page \pageref{tutodoc-listing-latex} pour employer les bons outils pour mettre en forme de simples cas d'utilisation.
\end{tdoccaut}



\subsection{Codes \tdocquote{en ligne}}

ZZZ


\tdocodein{py}{print("Hello!")}


\tdocodeinput{py}{examples/listing-full/hello-you.py}

\end{document}
