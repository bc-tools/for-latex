\documentclass{../main/main}

\usepackage[utf8]{inputenc}
\usepackage[T1]{fontenc}

\usepackage[french]{babel, varioref}

\usepackage{enumitem}
\frenchsetup{StandardItemLabels=true}

\usepackage{tabularray}

\usepackage[lang = french]{tutodoc}


\usepackage{../admonitions/admonitions.cls}
\usepackage{../inenglish/inenglish.cls}
\usepackage{../macroenv/macroenv.cls}
\usepackage{../showcase/showcase.cls}

% TESTING LOCAL IMPLEMENTATION %

\usepackage{listing-latex.cls}


\begin{document}

\subsection{Codes importés}

Pour les codes suivants, on considère un fichier de chemin relatif \verb+examples-listing-xyz.tex+, et ayant le contenu suivant.


\tdoclatexinput<\tdoctcb{code}>{examples/listing-latex/xyz.tex}


\medskip


La macro \tdocmacro{tdoclatexinput}, présentée ci-dessous, attend le chemin d'un fichier et propose le même système d'options entre crochets, ou entre chevrons, que l'environnement \tdocenv*{tdoclatex}.


\foreach \title/\extra/\fname in {%
	{Face à face}/%
		{}/%
		sbs,
	{À la suite}/%
		{ qui correspond aussi à l'option \tdoclatexin{\tdoctcb{std}}\,}/%
		std,%
	{Juste le code}/%
		{}/%
		code,%
	{Personnaliser}/%
		{}/%
		perso%
}{
	\begin{tdocexa}[\title]
    	\leavevmode
		\tdoclatexshow[explain=Ceci produit la mise en forme suivante\extra.]{examples/listing-latex/latexinput-option-\fname.tex}
	\end{tdocexa}
}



\subsection{Codes importés et mis en situation}
\label{tutodoc-listing-latexshow}

\begin{tdocnote}
    Les textes par défaut de la macro \tdocmacro{tdoclatexshow} tiennent compte de la langue détectée par \thisproj.
\end{tdocnote}


\begin{tdocexa}[Mise en situation]
    \tdoclatexin+\tdoclatexshow{examples-listing-xyz.tex}+ produit ce qui suit.

    \smallskip
    
    \tdoclatexshow{examples/listing-latex/xyz.tex}
\end{tdocexa}


\begin{tdocexa}[Changer le texte explicatif]
    Via la clé \tdoclatexin|explain|, on peut utiliser un texte personnalisé. Ainsi, \tdoclatexin|\tdoclatexshow[explain = Voici le rendu réel.]{examples-listing-xyz.tex}| produira ce qui suit.

    \smallskip

    \tdoclatexshow[explain = Voici le rendu réel.]{examples/listing-latex/xyz.tex}
\end{tdocexa}


\begin{tdocexa}[Les options disponibles]
    En plus du texte explicatif, il est aussi possible d'utiliser toutes les options de l'environnement \tdocenv*{tdocshowcase}, voir la section \ref{tutodoc-showcase} page \pageref{tutodoc-showcase}.
    Voici un exemple illustrant ceci.

    \medskip

    \tdoclatexinput<\tdoctcb{code}>{examples/listing-latex/latexshow-options.tex}

    \smallskip

    Ceci va produire ce qui suit.

    \smallskip

    \tdoclatexshow[style      = stripe,
               col-stripe = orange,
               col-text   = blue!70!black,
               before     = Rendu ci-après.,
               explain    = Ce qui vient est coloré...,
               after      = Rendu fini.,]
               {examples/listing-latex/xyz.tex}

\end{tdocexa}

\end{document}
