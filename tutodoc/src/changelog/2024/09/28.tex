\begin{tdocbreak}
	\item L'environnement \tdocenv*{tdoccaution} a été renommé \tdocenv*{tdoccaut} pour une saisie simplifiée.

	\item Mise en avant de contenus : les exemples et remarques, indiqués via les environnements \tdocenv*{tdocexa} et \tdocenv*{tdocrem}, sont toujours numérotés, et ils partagent le même compteur.

	\item La macro inutilisée \tdocmacro{tdocxspace} a été supprimée.
\end{tdocbreak}


\begin{tdocnew}
    \item Journal des changements : la macro \tdocmacro{tdocstartproj} permet de gérer le cas de la première version publique.

    \item Factorisation du code : la macro \tdocmacro{tdocicon} est en charge de l'ajout d'icônes devant du texte.
\end{tdocnew}


\begin{tdocupdate}
	\item Couleurs : les macros \tdocmacro{tdocdarkcolor} et \tdocmacro{tdoclightcolor} proposent un argument facultatif.
	\begin{enumerate}
		\item \tdocmacro{tdocdarkcolor} : la quantité de couleur par rapport au noir peut être définie de manière facultative.

		\item \tdocmacro{tdoclightcolor} : le taux de transparence peut être défini de manière facultative.
	\end{enumerate}

    \item Mise en avant de contenus : réduction de l'espace autour du contenu dans les cadres colorés.

	\item Gestion des versions: un meilleur espacement vertical via \tdocmacro{vphantom}.
\end{tdocupdate}
