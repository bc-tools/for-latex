\begin{tdocbreak}
	\item Mise en forme : la classe \tdoccls{scrartcl} remplace la vénérable \tdoccls{article}. Cela implique un meilleur positionnement des notes de marge avec les options retenues pour charger \tdoccls{scrartcl}.

	\item Code \LaTeX\ : la macro \tdocmacro{tdocinlatex} a été renommée \tdocmacro{tdoclatexin}.

	\item Les noms des clés pour les couleurs utiliseront des traits d'union lorsque cela sera nécessaire : cela implique les changements suivants.
	%
	\begin{enumerate}
		\item Mise en avant colorée de contenus : l'option \tdoclatexin{colchges} des environnements a été renommée \tdoclatexin{col-chges}.

		\item Démonstration de codes \LaTeX\ : pour l'environnement \tdocenv*{tdocshowcase} et la macro \tdocmacro{tdocshowcaseinput}, les options \tdoclatexin{colstripe} et \tdoclatexin{coltext} ont été renommées \tdoclatexin{col-stripe} et \tdoclatexin{col-text}\,.
	\end{enumerate}
\end{tdocbreak}


\begin{tdocfix}
	\item Mise en avant colorée de contenus : pour les \tdocmacro{newkeytheorem} utilisés avec le thème \tdoclatexin{draft}, il a fallu ajouter \tdoclatexin{postheadhook = \leavevmode} (ceci est nécessaire car le contenu peut juste être de type liste).
\end{tdocfix}


\begin{tdocnew}
	\item Documentation : ajout d'une section listant les dépendances.

	\item Options de classe.
	%
	\begin{enumerate}
		\item Les options qui ne sont pas spécifiques à \thisproj\ sont transmises à la classe chargée de la mise en forme générale.

		\item Les options \tdoclatexin{fontsize} et \tdoclatexin{DIV} de la classe \tdoccls{scrartcl} ne peuvent pas être utilisées, car leurs valeurs sont fixées par \thisproj.
	\end{enumerate}

	\item La macro \tdocmacro{tdocinEN} respecte les règles linguistiques anglaises.

	\item Mise en avant colorée de contenus.
	%
	\begin{enumerate}
		\item Un nouvel environnement \tdocenv{tdoctodo} a été ajouté.

		\item Chaque environnement dispose d'une nouvelle option \tdoclatexin{col} pour la couleur du contenu indiquant les changements.
	\end{enumerate}
\end{tdocnew}


\begin{tdocupdate}
	\item Documentation : la galerie des thèmes utilise un meilleur exemple factice.
\end{tdocupdate}



\begin{tdoctech}
	\item Organisation simplifiée des fichiers de configuration dans le projet final.
	%
	\begin{enumerate}
		\item Comme du \texttt{CSS} : emploi d'un fichier par thème avec des noms du type \texttt{tutodoc-bw.css.cls}\,.


		\item Locale : utilisation de noms comme \texttt{tutodoc-fr.loc.cls}\,.
	\end{enumerate}
\end{tdoctech}
