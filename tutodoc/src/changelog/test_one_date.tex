\documentclass{../main/main}


% -------------------------- %
% -- DATE TO TEST - START -- %
% -------------------------- %

\newcommand\docyear {2024}
\newcommand\docmonth{10}
\newcommand\docday  {19}

% Preparing next release
%\renewcommand\docday{0}


% ------------------------ %
% -- DATE TO TEST - END -- %
% ------------------------ %

\usepackage[utf8]{inputenc}
\usepackage[T1]{fontenc}

\usepackage[french]{babel, varioref}

\usepackage{enumitem}
\frenchsetup{StandardItemLabels=true}

\usepackage{tabularray}

\usepackage[lang = french]{tutodoc}


\usepackage{../deco/deco.cls}
\usepackage{../admonitions/admonitions.cls}
\usepackage{../inenglish/inenglish.cls}
\usepackage{../listing/listing.cls}
\usepackage{../macroenv/macroenv.cls}
\usepackage{../showcase/showcase.cls}
\usepackage{../version-n-change/version-n-change.cls}


\ExplSyntaxOn
  \int_compare:nTF { \docday = 0 } {
      \newcommand\when{NEXT ~ (draft)}
      \newcommand\filepath{next}
  } {
      \newcommand\when{\docyear-\docmonth-\docday}
      \newcommand\filepath{\docyear/\docmonth/\docday}
  }
\ExplSyntaxOff


\begin{document}

{
    \color{red!75!black}
    \rule\linewidth{1pt}

    \vspace{-1em}
    \section*{\color{red!75!black} Test: \when}
    \ExplSyntaxOff
    \vspace{-.75em}
    \rule\linewidth{1pt}
}


\small

\bigskip

\begin{tdoctodo}[
    col       = ForestGreen,
    col-chges = Goldenrod,
    date      = 5678-90-12,
    version   = 1.2.3-rc.4
]
    \item Voilà comment sera daté le tout premier type de changements.
\end{tdoctodo}


\input{\filepath}

\end{document}
